% @see : https://coopmaths.fr/alea/?uuid=0e6bd&id=6C10-1&n=10&d=10&s=2-3-4-5-6-7-8-9-10&s2=1&s3=true&alea=tPbO&cd=1&uuid=0e6bd&id=6C10-1&n=10&d=10&s=&s2=2&s3=true&alea=xa3t&cd=1&uuid=ac64a&id=6C10-9&alea=T3F0&uuid=fa2eb&id=6N43-2&n=6&d=10&s=true&alea=tG1v&cd=1&v=latex
\documentclass[a4paper,11pt,fleqn]{article}
\usepackage{ProfMaquette}
\setKVdefault[Boulot]{CorrigeFin=true}
\usepackage{etoolbox}
\newbool{dys}
\setbool{dys}{false}          
\ifbool{dys}{
% POLICE DYS
\usepackage{multicol}
\usepackage{unicode-math}
\usepackage{fontspec}
\setmainfont{TeX Gyre Schola}
%\setmainfont{OpenDyslexic}[Scale=1.0]
\setmathfont{TeX Gyre Schola Math}
\usepackage[fontsize=14]{scrextend}
\usepackage{setspace}
\setstretch{1.7}
}{
% POLICE STANDARD
\usepackage{fontenc}
\usepackage[scaled=1]{helvet}
\usepackage[fontsize=12]{scrextend}
}
\usepackage[left=1.5cm,right=1.5cm,top=2cm,bottom=2cm]{geometry}
\usepackage[luatex]{hyperref}
\usepackage{tikz}
\usetikzlibrary{calc}
\usepackage{fancyhdr}
\pagestyle{fancy}
\renewcommand\headrulewidth{0pt}
\setlength{\headheight}{18pt}
\fancyhead[R]{\href{https://coopmaths.fr/alea}{Mathaléa}}
\fancyfoot[C]{\thepage}
\fancyfoot[R]{%
\begin{tikzpicture}[remember picture,overlay]
  \node[anchor=south east] at ($(current page.south east)+(-2,0.25cm)$) {\scriptsize {\bfseries \href{https://coopmaths.fr/}{Coopmaths.fr} -- \href{http://creativecommons.fr/licences/}{CC-BY-SA}}};
\end{tikzpicture}
}
\fancyhead[L]{
\begin{tikzpicture}[y=0.8, x=0.8, yscale=-0.04, xscale=0.04,remember picture, overlay,fill=orange!50,transform canvas={xshift=-1cm,yshift=1cm}]
%%%% Arc supérieur gauche%%%%
\path[fill](523,1424)..controls(474,1413)and(404,1372)..(362,1333)..controls(322,1295)and(313,1272)..(331,1254)..controls(348,1236)and(369,1245)..(410,1283)..controls(458,1328)and(517,1356)..(575,1362)..controls(635,1368)and(646,1375)..(643,1404)..controls(641,1428)and(641,1428)..(596,1430)..controls(571,1431)and(538,1428)..(523,1424)--cycle;
%%%% Dé face supérieur%%%%
\path[fill](512,1272)..controls(490,1260)and(195,878)..(195,861)..controls(195,854)and(198,846)..(202,843)..controls(210,838)and(677,772)..(707,772)..controls(720,772)and(737,781)..(753,796)..controls(792,833)and(1057,1179)..(1057,1193)..controls(1057,1200)and(1053,1209)..(1048,1212)..controls(1038,1220)and(590,1283)..(551,1282)..controls(539,1282)and(521,1278)..(512,1272)--cycle;
%%%% Dé faces gauche et droite%%%%
\path[fill](1061,1167)..controls(1050,1158)and(978,1068)..(900,967)..controls(792,829)and(756,777)..(753,756)--(748,729)--(724,745)..controls(704,759)and(660,767)..(456,794)..controls(322,813)and(207,825)..(200,822)..controls(193,820)and(187,812)..(187,804)..controls(188,797)and(229,688)..(279,563)..controls(349,390)and(376,331)..(391,320)..controls(406,309)and(462,299)..(649,273)..controls(780,254)and(897,240)..(907,241)..controls(918,243)and(927,249)..(928,256)..controls(930,264)and(912,315)..(889,372)..controls(866,429)and(848,476)..(849,477)..controls(851,479)and(872,432)..(897,373)..controls(936,276)and(942,266)..(960,266)..controls(975,266)and(999,292)..(1089,408)..controls(1281,654)and(1290,666)..(1290,691)..controls(1290,720)and(1104,1175)..(1090,1180)..controls(1085,1182)and (1071,1176)..(1061,1167)--cycle;
%%%% Arc inférieur bas%%%%
\path[fill](1329,861)..controls(1316,848)and(1317,844)..(1339,788)..controls(1364,726)and(1367,654)..(1347,591)..controls(1330,539)and(1338,522)..(1375,526)..controls(1395,528)and(1400,533)..(1412,566)..controls(1432,624)and(1426,760)..(1401,821)..controls(1386,861)and(1380,866)..(1361,868)..controls(1348,870)and(1334,866)..(1329,861)--cycle;
%%%% Arc inférieur gauche%%%%
\path[fill](196,373)..controls(181,358)and(186,335)..(213,294)..controls(252,237)and(304,190)..(363,161)..controls(435,124)and(472,127)..(472,170)..controls(472,183)and(462,192)..(414,213)..controls(350,243)and(303,283)..(264,343)..controls(239,383)and(216,393)..(196,373)--cycle;
\end{tikzpicture}
}
%%%%%% Style Fiche
\tcbset{%
  userfiche/.style={%
    %move upwards=-1cm,colback=red!75%
    top=5pt, left=5pt, right=5pt, colback=red!5!white%
  }%
}%
\tcbset{%
  userfichecor/.style={%
    %spread upwards=-1cm,colback=gray!5%
    top=5pt, left=5pt, right=5pt, colback=red!5!white%
  }%
}%

% Parametrages
\hypersetup{
    colorlinks=true,% On active la couleur pour les liens. Couleur par défaut rouge
    linkcolor=blue,% On définit la couleur pour les liens internes
    % filecolor=magenta,% On définit la couleur pour les liens vers les fichiers locaux      
    urlcolor=blue,% On définit la couleur pour les liens vers des sites web
    % pdftitle={Puissance Quatre},% On définit un titre pour le document pdf
    % pdfpagemode=FullScreen,% On fixe l'affichage par défaut à plein écran
}
\usepackage{qrcode}
\usepackage{mathrsfs}
\usepackage{enumitem}
\setlength{\parindent}{0cm}
% loadPackagesFromContent
\usepackage{wrapfig}
\usepackage{needspace}
\newcommand{\e}{\text{e}}
\usepackage{amsmath}
\begin{document}
\begin{Maquette}[Fiche, CorrigeApres=false, CorrigeFin=true]{Niveau=Tables de multiplication et critères de divisibilité,Classe=1112CT,Date=   ,Theme=Semaine 1}

% @see : https://coopmaths.fr/alea?uuid=0e6bd&id=6C10-1&n=10&d=10&s=2-3-4-5-6-7-8-9-10&s2=1&s3=true&alea=tPbO&cd=1&cols=1
\needspace{10\baselineskip}
\begin{exercice}
\begin{wrapfigure}{r}{2cm}
\centering
{\hypersetup{urlcolor=black}
\qrcode{https://coopmaths.fr/alea?uuid=0e6bd&id=6C10-1&n=10&d=10&s=2-3-4-5-6-7-8-9-10&s2=1&s3=true&alea=tPbO&cd=1&cols=1&v=eleve&es=0211}
}
Correction
\end{wrapfigure}\ 
Compléter.	
	\begin{enumerate}[itemsep=2em]
	\item $ 9\cdot 9 =\ldots\ldots$
	\item $ 9\cdot 10 =\ldots\ldots$
	\item $ 8\cdot 10 =\ldots\ldots$
	\item $ 7\cdot 3 =\ldots\ldots$
	\item $ 7\cdot 5 =\ldots\ldots$
	\item $ 5\cdot 6 =\ldots\ldots$
	\item $ 10\cdot 9 =\ldots\ldots$
	\item $ 2\cdot 5 =\ldots\ldots$
	\item $ 9\cdot 6 =\ldots\ldots$
	\item $ 5\cdot 5 =\ldots\ldots$
\end{enumerate}
\end{exercice}

\begin{Solution}
\begin{enumerate}[itemsep=1em]
	\item $ 9\cdot 9 = {\color[HTML]{f15929}\boldsymbol{81}}$
	\item $ 9\cdot 10 = {\color[HTML]{f15929}\boldsymbol{90}}$
	\item $ 8\cdot 10 = {\color[HTML]{f15929}\boldsymbol{80}}$
	\item $ 7\cdot 3 = {\color[HTML]{f15929}\boldsymbol{21}}$
	\item $ 7\cdot 5 = {\color[HTML]{f15929}\boldsymbol{35}}$
	\item $ 5\cdot 6 = {\color[HTML]{f15929}\boldsymbol{30}}$
	\item $ 10\cdot 9 = {\color[HTML]{f15929}\boldsymbol{90}}$
	\item $ 2\cdot 5 = {\color[HTML]{f15929}\boldsymbol{10}}$
	\item $ 9\cdot 6 = {\color[HTML]{f15929}\boldsymbol{54}}$
	\item $ 5\cdot 5 = {\color[HTML]{f15929}\boldsymbol{25}}$
\end{enumerate}
\end{Solution}

% @see : https://coopmaths.fr/alea?uuid=0e6bd&id=6C10-1&n=10&d=10&s=2-3-4-5-6-7-8-9-10&s2=2&s3=true&alea=xa3t&cd=1&cols=1
\needspace{10\baselineskip}
\begin{exercice}
\begin{wrapfigure}{r}{2cm}
\centering
{\hypersetup{urlcolor=black}
\qrcode{https://coopmaths.fr/alea?uuid=0e6bd&id=6C10-1&n=10&d=10&s=2-3-4-5-6-7-8-9-10&s2=2&s3=true&alea=xa3t&cd=1&cols=1&v=eleve&es=0211}
}
Correction
\end{wrapfigure}\ 
Compléter.
\begin{enumerate}[itemsep=2em]
	\item $7\cdot  \ldots\ldots =42$
	\item $6\cdot  \ldots\ldots =60$
	\item $2\cdot  \ldots\ldots =20$
	\item $3\cdot  \ldots\ldots =9$
	\item $7\cdot  \ldots\ldots =21$
	\item $3\cdot  \ldots\ldots =15$
	\item $4\cdot  \ldots\ldots =20$
	\item $4\cdot  \ldots\ldots =32$
	\item $4\cdot  \ldots\ldots =28$
	\item $7\cdot  \ldots\ldots =35$
\end{enumerate}
\end{exercice}

\begin{Solution}
\begin{enumerate}[itemsep=1em]
	\item $7 \cdot {\color[HTML]{f15929}\boldsymbol{6}} =42$
	\item $6 \cdot {\color[HTML]{f15929}\boldsymbol{10}} =60$
	\item $2 \cdot {\color[HTML]{f15929}\boldsymbol{10}} =20$
	\item $3 \cdot {\color[HTML]{f15929}\boldsymbol{3}} =9$
	\item $7 \cdot {\color[HTML]{f15929}\boldsymbol{3}} =21$
	\item $3 \cdot {\color[HTML]{f15929}\boldsymbol{5}} =15$
	\item $4 \cdot {\color[HTML]{f15929}\boldsymbol{5}} =20$
	\item $4 \cdot {\color[HTML]{f15929}\boldsymbol{8}} =32$
	\item $4 \cdot {\color[HTML]{f15929}\boldsymbol{7}} =28$
	\item $7 \cdot {\color[HTML]{f15929}\boldsymbol{5}} =35$
\end{enumerate}
\end{Solution}

% @see : https://coopmaths.fr/alea?uuid=ac64a&id=6C10-9&n=10&d=10&s=3-4-5-6-7-8-9&s2=9&alea=T3F0&cd=1&cols=1
\needspace{10\baselineskip}
\begin{exercice}
\begin{wrapfigure}{r}{2cm}
\centering
{\hypersetup{urlcolor=black}
\qrcode{https://coopmaths.fr/alea?uuid=ac64a&id=6C10-9&n=10&d=10&s=3-4-5-6-7-8-9&s2=9&alea=T3F0&cd=1&cols=1&v=eleve&es=0211}
}
Correction
\end{wrapfigure}\ 
Compléter avec deux nombres entiers différents de 1.
\begin{enumerate}[itemsep=1em]
	\item $12= \ldots\ldots \cdot \ldots\ldots$
	\item $48= \ldots\ldots \cdot \ldots\ldots$
	\item $10= \ldots\ldots \cdot \ldots\ldots$
	\item $30= \ldots\ldots \cdot \ldots\ldots$
	\item $24= \ldots\ldots \cdot \ldots\ldots$
	\item $14= \ldots\ldots \cdot \ldots\ldots$
	\item $36= \ldots\ldots \cdot \ldots\ldots$
	\item $28= \ldots\ldots \cdot \ldots\ldots$
	\item $15= \ldots\ldots \cdot \ldots\ldots$
	\item $49= \ldots\ldots \cdot \ldots\ldots$
\end{enumerate}
\end{exercice}

\begin{Solution}
\begin{enumerate}[itemsep=1em]
	\item $12={\color[HTML]{f15929}\boldsymbol{3}} \cdot {\color[HTML]{f15929}\boldsymbol{4}}$
	\item $48={\color[HTML]{f15929}\boldsymbol{8}} \cdot {\color[HTML]{f15929}\boldsymbol{6}}$
	\item $10={\color[HTML]{f15929}\boldsymbol{5}} \cdot {\color[HTML]{f15929}\boldsymbol{2}}$
	\item $30={\color[HTML]{f15929}\boldsymbol{6}} \cdot {\color[HTML]{f15929}\boldsymbol{5}}$
	\item $24={\color[HTML]{f15929}\boldsymbol{4}} \cdot {\color[HTML]{f15929}\boldsymbol{6}}$
	\item $14={\color[HTML]{f15929}\boldsymbol{7}} \cdot {\color[HTML]{f15929}\boldsymbol{2}}$
	\item $36={\color[HTML]{f15929}\boldsymbol{9}} \cdot {\color[HTML]{f15929}\boldsymbol{4}}$
	\item $28={\color[HTML]{f15929}\boldsymbol{4}} \cdot {\color[HTML]{f15929}\boldsymbol{7}}$
	\item $15={\color[HTML]{f15929}\boldsymbol{3}} \cdot {\color[HTML]{f15929}\boldsymbol{5}}$
	\item $49={\color[HTML]{f15929}\boldsymbol{7}} \cdot {\color[HTML]{f15929}\boldsymbol{7}}$
\end{enumerate}
\end{Solution}

% @see : https://coopmaths.fr/alea?uuid=fa2eb&id=6N43-2&n=6&d=10&s=true&alea=tG1v&cd=1&cols=1
\needspace{10\baselineskip}
\begin{exercice}
\begin{wrapfigure}{r}{2cm}
\centering
{\hypersetup{urlcolor=black}
\qrcode{https://coopmaths.fr/alea?uuid=fa2eb&id=6N43-2&n=6&d=10&s=true&alea=tG1v&cd=1&cols=1&v=eleve&es=0211}
}
\qrcode[height=1in]{https://coopmaths.fr/alea/?uuid=0e6bd&id=9NO3-8&n=10&d=10&s=2-3-4-5-6-7-8-9-10&s2=1&s3=true&cd=1&uuid=0e6bd&id=9NO3-8&n=10&d=10&s=&s2=2&s3=true&cd=1&uuid=ac64a&uuid=fa2eb&id=9NO4-5&n=6&d=10&s=true&cd=1&lang=fr-CH&v=eleve&es=0111000&title=Semaine+1}

Correction
\end{wrapfigure}\ 
Compléter le tableau en mettant oui ou non dans chaque case.
 

$\begin{array}{|l|c|c|c|c|}
\hline
\text{... est divisible} & \text{par }2 & \text{par }3 & \text{par }5 & \text{par }9\\
\hline
103\,622 & & & & \\
\hline
847\,494 & & & & \\
\hline
451\,806 & & & & \\
\hline
394\,299 & & & & \\
\hline
160\,383 & & & & \\
\hline
989\,085 & & & & \\
\hline
\end{array}
$
\end{exercice}

\begin{Solution}
 $\begin{array}{|l|c|c|c|c|}
\hline
\text{... est divisible} & \text{par }2 & \text{par }3 & \text{par }5 & \text{par }9\\
\hline
103\,622 & \color{blue}{\text{oui}} & \text{non} & \text{non} & \text{non} \\\hline
847\,494 & \color{blue}{\text{oui}} & \color{blue}{\text{oui}} & \text{non} & \color{blue}{\text{oui}} \\\hline
451\,806 & \color{blue}{\text{oui}} & \color{blue}{\text{oui}} & \text{non} & \text{non} \\\hline
394\,299 & \text{non} & \color{blue}{\text{oui}} & \text{non} & \color{blue}{\text{oui}} \\\hline
160\,383 & \text{non} & \color{blue}{\text{oui}} & \text{non} & \text{non} \\\hline
989\,085 & \text{non} & \color{blue}{\text{oui}} & \color{blue}{\text{oui}} & \text{non} \\\hline
\end{array}$

\end{Solution}

\end{Maquette}
\clearpage
\end{document}
