\documentclass[a4paper,12pt]{report}

\usepackage{../courslatex}
\usepackage{exos}
\usepackage{enumitem}

\setlist[enumerate]{align=left,leftmargin=1cm,itemsep=5pt,parsep=0pt,topsep=0pt,rightmargin=0.5cm}

\setlist[itemize]{align=left,labelsep=1em,leftmargin=*,itemsep=0pt,parsep=0pt,topsep=0pt,rightmargin=0cm}

\renewcommand{\titreChapitre}{Ch3 : Calcul littéral} 
\renewcommand{\numeroSerie}{7}

\begin{document}
\vspace*{-2\baselineskip}
\textLigne{Activités}
\begin{acti}
Factoriser complètement.
	\begin{tasks}(3)
\task $6 x^2-6$
\task $6 x^2-12 x+6$
\task $12 y^2-12 y+3$
\task $9 x^3-36 x$
\task $4 x^3-9 x$
\task $2 x y^2+4 x y+2 x$
\task $12 y^2+24 y z+12 z^2$
\task $27 x^4-12 x^2$
\task $12-8 x+x^2$
\task $9 x y^2+6 x y z+x z^2$
\task $20 x^2+60 x+45$
\task $a^2 x^4-a^4 x^2$
	\end{tasks}
\end{acti}
\begin{acti}
Factoriser complètement (utiliser notamment la méthode des groupements).
	\begin{tasks}(2)
\task $2 a x+a y-12 x-6 y$
\task $5 x^3-10 x^2-x+2$
\task $x^2-y^2+a\left(x^2-2 x y+y^2\right)$
\task $7 x^3+9-3 x^2-21 x$
\task $5 b x-a y+b y-5 a x$
\task $(x-y)(2 x-y+1)+(y-x)(x-y+1)$
\task $6 x^2-6 y+a y-a x^2$
\task $(x-8)(4 x-3)+x^2-8 x$
\task $y^2-1-x^2+x^2 y^2$
\task $3 x^4+6 x^3+2 x^2+4 x$
	\end{tasks}
\end{acti}
\begin{acti}
Factoriser complètement (utiliser notamment la méthode des groupements).
	\begin{tasks}(3)
\task $a x y^2+b x y^2-a x-b x$
\task $8 x^2+4 x y-2 a x-a y$
\task $u^3-u-u^2+1$
\task $a x^2-1-x^2+a$
\task $x^3-2 x^2+x-2$
\task (*) $x^3+2 x^2+2 x+1$
\task $\left(x^2-1\right)-3(1-x)$
\task $a^2-b^2-5 a+5 b$
\task $a^2 b^2+a^2-b^2-1$
\task $x^3+2 x^2-4 x-8$
\task $a^2 b^2+b^2-a^2-1$
\task $x^3-7 x^2-4 x+28$
	\end{tasks}
	Indice pour le j)\,: $2 x^2=x^2+x^2$ 
\end{acti}
\begin{acti}
Développer les produits, factoriser les sommes.
	\begin{tasks}(4)
\task $(2 x+3)^2$
\task $4 x+6 y^2$
\task $9 b^2+12 b+4$
\task $x^2+6 x-7$
\task $9 y^2-6 y+1$
\task $4 h^2(2 h+3)$
\task $(1-x)^2$
\task $16 a^2-25$
\task $(4 a-5)(4 a+5)$
\task $1-2 x+x^2$
\task $8 h^3+12 h^2$
\task $(3 y-1)^2$
\task $(x-1)(x+7)$
\task $(2+3 b)^2$
\task $\left(2 x+3 y^2\right) \cdot 2$
\task $4 x^2+12 x+9$
	\end{tasks}
\end{acti}
\begin{acti}
(*) Retrouver les identités remarquables pour le cube du binôme : $(a+b)^3$ et $(a-b)^3$.

En partant de ces identités, obtenir celles pour (ou la factorisation de) :
	\begin{tasks}(2)
\task $a^3+b^3$
\task $a^3-b^3$.
	\end{tasks}
\end{acti}
\begin{acti}
(*) Par la méthode de complétion du carré, factoriser (si possible) les polynômes suivants :
	\begin{tasks}(3)
\task $x^2-8 x+13$
\task $x^2-2 x-5$
\task $x^2+20 x+91$
\task $x^2+3 x+1$
\task $x^2+4 x+6$
\task $x^2-x-1$
\task $4 x^2+4 x-3$
\task $-3 x^2+3 x+1$
\task $2 x^2+7 x+3$
	\end{tasks}
\end{acti}
\begin{acti}
(*) Soit le polynôme $x^6-1$.
	\begin{tasks}
\task Le factoriser de deux manières différentes (Indication: $x^6=\left(x^3\right)^2=\left(x^2\right)^3$ ).
\task En déduire une factorisation pour le polynôme $x^4+x^2+1$.
	\end{tasks}
\end{acti}
\begin{acti}
Factoriser autant que possible.
	\begin{tasks}(3)
\task $2 x y^2+4 x y+2 x$
\task $45 a^2-30 a+5$
\task $5 x^4-20 x^2$
\task $3 x^2 y+30 x y+48 y$
\task $7 a^4 x-14 a^3 x^2+7 a^2 x^3$
\task $9 a^5+24 a^3 b^2+16 a b^4$
\task $4 x^3 y-16 x^2 y^2+16 x y^3$
\task $2 a^3 x^3-4 a^2 x^2+2 a x$
\task $3 x(x+1)^2-27 x$
\task $9 a b^2 c^4-4 a b^4$
\task $a^2 x^2-4 b^2 x^4$
\task $a^2(x+2 y)-4(x+2 y)$
	\end{tasks}
\end{acti}

\begin{acti}
Factoriser complètement.
	\begin{tasks}
\task $6 x^3-3 x^2+3 x$
\task $12 x^2-12 x+3$
\task $x^2-10 x-11$
\task $y^7-16 y^3$
\task $18 a^2-2 b^2$
\task $3 a^4 b+6 a^3 b^2-a^2 b^3$
\task $4 a^2+9+12 a$
\task $2 x y^2-20 x y+32 x$
\task $-a b^2+2 a^2 b-a^3$
\task $2 x^2 y+2 x y-24 y$
\task $2 y^3-2 y^2-2 y$
\task $2 a b^2-16 a b+32 a$
\task $7 x^3+28 x^2-35 x$
\task $1-36 m^2$
\task $6 a^3 b^2 c-21 a^2 b^3 c^2+9 a^3 b^2 c^2$
	\end{tasks}
\end{acti}
\textLigne{Exercices}
\textLigne{Automatismes}
Résolution d'équations du premier degré.

\end{document}

