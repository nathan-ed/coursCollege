\documentclass[a4paper,12pt]{report}

\usepackage{../courslatex}
\usepackage{exos}
\usepackage{enumitem}
\renewcommand{\typeDoc}{Corrigé série }

\setlist[enumerate]{align=left,leftmargin=1cm,itemsep=5pt,parsep=0pt,topsep=0pt,rightmargin=0.5cm}

\setlist[itemize]{align=left,labelsep=1em,leftmargin=*,itemsep=0pt,parsep=0pt,topsep=0pt,rightmargin=0cm}

\renewcommand{\titreChapitre}{Ch3 : Calcul littéral} 
\renewcommand{\numeroSerie}{4}

\begin{document}
\vspace*{-2\baselineskip}
\textLigne{Exercices}
\begin{core}
	\phantom{}
\begin{tasks}(2)
\task $7(8+9 x)=56+63x$
\task $6 a\left(5 a^2-12 a\right)=30a^3-72a^2$
\task $-5(-7 y+11)=35y-55$
\task $-12(-5 x-4)=60x+48$
\task $-8\left(6 x^2+4 x-3\right)=-48x^2-32x+24$
\task $-9 x^2\left(8 x^3+7 y\right)=-72x^5-63x^y$
\task $7 a^5\left(6 a-4 a^2\right)42a^6-28a^7$
\task $-5 x^4\left(7 x^4+9 x-1\right)=-35x^8-45x^5+5x^4$
\end{tasks}
\end{core}
\begin{core}
	\phantom{}
\begin{tasks}(3)
\task $2 x-2 x=0$
\task $(2 x)(-2 x)=-4x^2$
\task $2(x-2) x=-4x+2x^2$
\task $-5 y+9 y=4y$
\task $-(5 y+9 y)=-14y$
\task $(-5 y)(+9 y)=-45y^2$
\task $(-5 y+9) y=9y-5y^2$
\task $(-5 y)+9 y=4y$
\task $-5(y+9) y=-45y-5y^2$
\task $-5(y+9 y)=-50y$
\task $-x(-x)(-1)=-x^2$
\task $-x(-x-1)=x+x^2$
\task $-(x-x)-1=-1$
\task $x \cdot x \cdot x+x \cdot x=x^2+x^3$
\task $x \cdot x \cdot(x+x) \cdot x=2x^4$
\end{tasks}
\end{core}
\begin{core}
	\phantom{}
\begin{tasks}(2)
\task $5(5+3 x)=25+15x$
\task $2 x\left(2 x^2-2 x\right)=4x^3-4x^2$
\task $-5(-5 y+9)=25y-45$
\task $-1(-3 x-3)=3x+3$
\task $\left(x^2+x-1\right)(-1)=-x^2-x+1$
\task $-2(x+y)=-2x-2y$
\task $\left(1+x^2\right)\left(x^2-4\right)=-4-3x^2+x^4$
\task $-3 x^2\left(1-2 x^2+3 x\right)=-5+2x+3x^2$
\task $(5+3 x)(x-1)=$
\task $3 x y\left(x^2 y+x-1\right)$
\task $\left(4-x^2\right)\left(1-4 x^2\right)$
\task $\left(-4 x y^3-x^3 y\right)(-3 y)$
\task $-2(x+3)(x-1)$
\task $3(x-3)(x-3)$
\task $(-2 x+3)(x-1)$
\task $(-2 x+3)(3-2 x)$
\end{tasks}
\end{core}
\begin{core}
\phantom{}
On factorise l'expression pour obtenir (par la mise en évidence) 
\[4a^2+6a=2a\cdot (2a+3)\]
Ainsi, la longueur vaut $2a+3$.
\end{core}
\begin{core}
	\phantom{}

	$\begin{aligned}\left(n^2+n+1\right)\left(n^2-n+1\right)&=n^4-n^3+n^2+n^3-n^2+n+n^2-n+1\\
		&=n^4+n^2+1
	\end{aligned}$. 
\end{core}
\begin{core}
	On vérifie en développant\,:

	$\begin{aligned}\left(x^2+2 x+2\right)\left(x^2-2 x+2\right)=x^4+4&=x^4-2x^3\\
	\end{aligned}$
\end{core}


\begin{core}
Un élève a développé tous les produits de trois des binômes $(x+1),(x-1),(x+2)$ et $(x-2)$, de toutes les manières possibles, sans répétition d'un binôme. Il a noté les résultats suivants :
$$
x^3-x^2-4 x+4, x^3-2 x^2-x+2, x^3+2 x^2-x-2 \text { et } x^3+x^2-4 x-4 \text {. }
$$
Malheureusement, cet élève ne se souvient pas dans quel ordre il a effectué ses calculs.
Comment peut-on l'aider à s'y retrouver immédiatement, par une simple observation~?
\end{core}
\begin{core}
	\phantom{}
\begin{tasks}(3)
\task $(2 y-3)(5+3 x)$
\task $(5+2 x)(2 x-3)$
\task $(3-y)(-5 y+9)$
\task $\left(x^2+x-1\right)(x-1)$
\task $(y-x)(x+y)$
\task $(x+1)(x-1)(x+2)$
\task $(2 x-1)(x+3)(1-x)$
\task $\left(1+x^2\right)\left(x^2-4 x+2\right)$
\task $(x+2)^3$
\task $\left(z^3-5 x^3 z+2 z\right)\left(z^3-3 x\right)$
\task $(2-x)\left(x^2+4\right)(2+x)$
\task $(x-1)^4$
\end{tasks}
\end{core}

\begin{core}
	Développer les expressions de l'activité 4 aux lettres a), c), e), f), g) et h). 

(Expression réduite et ordonnée par puissances décroissantes.)
\end{core}


\end{document}

