\documentclass[a4paper,12pt]{report}

\usepackage{../courslatex}
\usepackage{exos}
\usepackage{enumitem}

\setlist[enumerate]{align=left,leftmargin=1cm,itemsep=5pt,parsep=0pt,topsep=0pt,rightmargin=0.5cm}

\setlist[itemize]{align=left,labelsep=1em,leftmargin=*,itemsep=0pt,parsep=0pt,topsep=0pt,rightmargin=0cm}

\renewcommand{\titreChapitre}{Ch2 : Ensembles et intervalles réels} 
\renewcommand{\numeroSerie}{3}

\begin{document}
\vspace*{-2\baselineskip}

\textLigne{Activités}
\begin{acti}
	\phantom{.}
	
	{\bfseries Partie 1}

Donner l'intervalle correspondant à l'ensemble entre accolades, ou vice versa.
\begin{tasks}(2)
\task $\quad \cdots \quad=\{x \in \mathbb{R} \mid-3 \leq x<4\}$
\task $\ldots \quad=\{x \in \mathbb{R} \mid x \geq-0,5\}$
\task $\interval[open left]{-\infty}{-2}=\{\ldots \quad\}$
\task $\interval[open]{-1}{-0,5}=\{... \}$
\end{tasks}

{\bfseries Partie 2}

Donner les sous-ensembles de $\R$ suivants à l'aide d'intervalles uniquement :
\begin{tasks}(5)
\task $\mathbb{R} \setminus\{2\}$
\task $\mathbb{R} \setminus\interval{2}{3}$
\task $\mathbb{R} \setminus\interval[open]{-1}{6}$
\task*(2) $\{x \in \mathbb{R} \mid x<-5$ ou $x \geq 2\}$
\end{tasks}
\end{acti}

\begin{acti}\phantom{}
	\begin{tasks}
\task Soient $A$ et $B$ les deux ensembles suivants : $A=\{-5 ; 3 ; 4 ; 6 ; 8 ; 9\}$ et $B=\{2 ; 3 ; 4 ; 8 ; 10\}$.

Déterminer $A \cup B, A \cap B, B \backslash A$ et $A \backslash B$.
\task Trouver les ensembles $C$ et $D$ puis $E$ et $F$ sachant que :

\[C \cup D=\{1 ; 2 ; 3 ; 4 ; 5\}, C \cap D=\{2 ; 3 ; 4\}, 1 \notin D \backslash C \text{ et } 5 \notin C \backslash D\]

\[E \cup F=\{2 ; 3 ; 4 ; 5\} \text{ et } E \cap F=\{2 ; 4\}\] 

 Donner toutes les possibilités.
	\end{tasks}
\end{acti}

\textLigne{Exercices}

\begin{exo}
	On donne trois sous-intervalles de $\mathbb{R}: I=\interval[open right]{-3}{4}, J=\interval[open right]{-2}{0}$ et $K=\interval[open left]{-5}{3}$.

Donner à l'aide d'intervalles : $I \cap J, I \cup K, I \cap K, I \setminus K$ et $K \setminus I$.
\end{exo}
\begin{exo}
Décrire les ensembles de réels suivants à l'aide d'intervalles:
\begin{tasks}(2)
\task $\{x \in \mathbb{R} \mid-3 \leq x \leq 2\}$
\task $\{x \in \mathbb{R} \mid x \geq 3\}$
\task $\{x \in \mathbb{R} \mid-1>x\}$
\task $\{x \in \mathbb{R} \mid x>-2$ et $x \leq 4\}$
\task $\left\{x \in \mathbb{R} \left\lvert\,-\frac{3}{2}<x \leq-\frac{1}{2}\right.\right\}$
\task $\{x \in \mathbb{R} \mid x \leq 1+\sqrt{2}\}$
\task $\mathbb{R}$
\task $\{x \in \mathbb{R} \mid x<-2$ ou $x \geq 4\}$
\end{tasks}
\end{exo}
\begin{exo}
Dans chaque cas, trouver $A$ et $B$, deux sous-ensembles de $\mathbb{Z}$ tels que:
\begin{tasks}(2)
\task $A \cup B=\{0 ; 1 ; 2 ; 3 ; 4\}$ et $A \cap B=\varnothing$
\task $A \cup B=\{0 ; 1 ; 2 ; 3 ; 4\}$ et $A \cap B=\{2 ; 3 ; 4\}$
\task $A \cup B=\{0 ; 1 ; 2 ; 3 ; 4\}$ et $A \setminus B=\{2 ; 3 ; 4\}$
\task $A \cup B=\{0 ; 1 ; 2 ; 3 ; 4\}$ et $B \setminus A=\{1 ; 4\}$
\end{tasks}
\end{exo}
\begin{exo}
Dans l'ensemble $T$ des triangles, on considère $I$, le sous-ensemble des triangles isocèles\,; $E$, le sous-ensemble des triangles équilatéraux\,; $R$, le sous-ensemble des triangles rectangles
\begin{tasks}
\task Représenter ces quatre ensembles à l'aide d'un diagramme.
\task Décrire par des mots les ensembles $I \cap E, R \cap E$ et $I \cap R$.
\end{tasks}
\end{exo}

\begin{exo}
Les propositions suivantes sont-elles vraies ou fausses ?
\begin{tasks}(3)
\task $0 \in \mathbb{R}_{+}$
\task $-2 \in\interval[open left]{-2}{5}$
\task $\N \subset \mathbb{R}$
\task $3 \in\{2 ; 4\}$
\task $3 \in\interval[open]{2}{4}$
\task $3 \notin \mathbb{R} \setminus\interval[open]{2}{3}$
\task $\interval{0}{2024} \cap \mathbb{R}_{-}=\varnothing$
\task $\pi \in \mathbb{R} \setminus \mathbb{Q}$
\task $\mathbb{N} \setminus \mathbb{Z}=\varnothing$
\end{tasks}
\end{exo}

\textLigne{Automatismes}
Jeûne genevois, pas d'automatismes cette semaine.

\begin{comment}
\begin{exo}
	Trouver dans chaque intervalle: $\interval[open]{-4}{-3}\,;\,\interval[open]{\dfrac{1}{4}}{\dfrac{1}{3}}\,;\,\interval[open]{10^{-4}}{10^{-3}}$\,:
\begin{tasks}
\task deux nombres rationnels, l'un à partie décimale finie et l'autre à partie décimale infinie périodique 

(les donner sous forme de fraction irréductible);
\task un nombre irrationnel.
\end{tasks}
\end{exo}

\begin{exo}
(*) Trouver dix fractions irréductibles distinctes et appartenant toutes à l'intervalle $] \frac{1}{3} ; \frac{2}{3}[$, sans l'aide d'une calculatrice. (Classez-les dans l'ordre croissant.)
\end{exo}
\begin{exo}
Déterminer les éléments des sous-ensembles $A$ et $B$ de $E$ sachant que:
\[E \setminus A=\{f ; g ; h ; i\},\quad A \cup B=\{a ; b ; c ; d ; e ; f\} \quad \text { et } \quad A \cap B=\{d ; e\}\]
\end{exo}
\begin{acti}
	Quel est le nombre réel situé à égale distance des bornes de l'intervalle $\interval{\sqrt{27}}{\sqrt{75}}$ ~?

(Réponse sous forme simplifiée; s'il s'agit d'une racine carrée: de quel entier ?)
\end{acti}

\end{comment}

\end{document}

