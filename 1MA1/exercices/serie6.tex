\documentclass[a4paper,12pt]{report}

\usepackage{../courslatex}
\usepackage{exos}
\usepackage{enumitem}

\setlist[enumerate]{align=left,leftmargin=1cm,itemsep=5pt,parsep=0pt,topsep=0pt,rightmargin=0.5cm}

\setlist[itemize]{align=left,labelsep=1em,leftmargin=*,itemsep=0pt,parsep=0pt,topsep=0pt,rightmargin=0cm}

\renewcommand{\titreChapitre}{Ch3\, : Calcul littéral} 
\renewcommand{\numeroSerie}{6}

\begin{document}
\vspace*{-2\baselineskip}
\textLigne{Activités}


\begin{acti}Factoriser au maximum les expressions suivantes

\begin{tasks}(2)
	\task $25 s^4-20 s^2 + 4 $
	\task $9 s^2 t^2 x^2 + 48 s t x + 64 $
 	\task $s^2 t^2 + 36 r^4-12 r^2 s t$
	\task $4  + 81 y^2 + 36 y$
	\task $s^2 t^2 z^2-1$
	\task $25 x^2 y^2-90 r x y + 81 r^2$
	\task $-147 t^4 y z - 420 t^2 y^2 z-300 y^3 z$
\end{tasks}
\end{acti}

\begin{acti}
Mettre en évidence le facteur commun.
\begin{tasks}(3)
\task $4x(x+y)+5 x(x+y)$
\task $3 a(3 a-b)-8(3 a-b)$
\task $5 a^2b(a-2 b)-15 a b^2(a-2 b)$
\task $9 x(x+2)^2-5 x(x+2)$
\task $4(x-y)+2 x(y-x)$
\task $x^2(2 x-1)+3 x^2(1-2 x)$
\end{tasks}
\end{acti}

\textLigne{Exercices}
Conseil\,: commencer par les automatismes avant de faire les exercices.
% @see : https://coopmaths.fr/alea?uuid=7f0cf&id=3L11-16&n=10&d=10&s=1-2-3-5-6-7&s2=3&s3=2&s4=3&alea=BNMz&cd=1&cols=1
\begin{exo}Compléter le terme manquant afin d'obtenir une identité remarquable. Écrire ensuite l'identité remarquable correspondante factorisée.

\begin{tasks}(2)
	\task $9 s^2 z^2-20 s z^2 + \ldots\ldots\,=\,\ldots\ldots$
	\task $-\dfrac{2}{3} r z + \dfrac{1}{9}  + \ldots\ldots\,=\,\ldots\ldots$
	\task $-\dfrac{49}{9}  + \ldots\ldots\,=\,\ldots\ldots$
	\task $\dfrac{9}{64} r^2 t^2 + \dfrac{3}{20} r t x + \ldots\ldots\,=\,\ldots\ldots$
	\task $14 r z + 49  + \ldots\ldots\,=\,\ldots\ldots$
	\task $36 x^2 y^2-\ldots\ldots\,=\,\ldots\ldots$
	\task $\dfrac{25}{9} x^2 + 64 r^4-\ldots\ldots\,=\,\ldots\ldots$
	\task $s^2 y^2 + 6 s y + \ldots\ldots\,=\,\ldots\ldots$
	\task $\dfrac{16}{25} -\dfrac{8}{5} x z + \ldots\ldots\,=\,\ldots\ldots$
	\task $16 r^4 + \dfrac{40}{3} r^2 z + \ldots\ldots\,=\,\ldots\ldots$
\end{tasks}
\end{exo}

\begin{exo}
	Factoriser le plus possible les expressions suivantes.
	\begin{tasks}(2)
\task $m(a-b)+n(a-b)$
\task $x(2 a-b)+y(b-2 a)$
\task $a(x-y)-(y-x)$
\task $(a+b)(x-3 y)-3 a(x-3 y)$
\task $(a+b)^3-(a+b)^2$
\task $(x-3)(x+1)-x+3+2(x-3)^2$
\task $(a-b)^3-(a-b)$
\task $(x-y)-(a+b)^2(x-y)$
	\end{tasks}
\end{exo}

\begin{exo}
(*) Pour quels entiers $x$ de 1 à 200 le nombre $x^4-x^3$ est-il le cube d'un entier?
\end{exo}
\textLigne{Automatismes}
Ces automatismes seront testés lors de la semaine 7, le 03.09.2024.

% @see : https://coopmaths.fr/alea?uuid=c4b73&id=3L12-2&n=10&d=10&s=1&s2=3&s3=3&s4=3&s5=false&alea=Iz6c&cd=1&cols=1
\begin{auto}Factoriser au maximum les expressions suivantes

\begin{tasks}(2)
	\task $30 s^2 t z + 10 s z-30 z$
	\task $54 r t^3 z^2-6 r t^3 z-18 t^2 z$
	\task $-45 r^2 t x-27 r x + 72 x$
	\task $40 s^2 y^3 + 25 s^2 x^2 y + 45 s x y$
	\task $-20 s x^4-20 s x^2 z-90 s^2 x^2$
	\task $10 y^3 z^2 + 20 y z^4-6 y z^2$
\end{tasks}
\end{auto}

% @see : https://coopmaths.fr/alea?uuid=c4b73&id=3L12-2&n=10&d=10&s=2&s2=2&s3=2&s4=2&s5=false&alea=W950&cd=1&cols=1
\begin{auto}Factoriser au maximum les expressions suivantes

\begin{tasks}(2)
	\task $36 t^2-12 t x + x^2$
	\task $49 r^2 s^2-25 $
	\task $16 x^2 z^2 + 40 x z + 25 $
	\task $16 r^2 s^2 + 56 r s + 49 $
	\task $4 z^4-12 z^2 + 9 $
	\task $49 s^4-28 s^2 x + 4 x^2$
	\task $-25 t^4 + 36 r^2$
	\task $r^2 y^2-14 r y + 49 $
	\task $36 r^2 t^2-49 $
	\task $64 x^2 + 112 x + 49 $
\end{tasks}
\end{auto}

% @see : https://coopmaths.fr/alea?uuid=c4b73&id=3L12-2&n=10&d=10&s=2&s2=2&s3=3&s4=3&s5=false&alea=SIk2&cd=1&cols=1
\begin{auto}Factoriser au maximum les expressions suivantes

\begin{tasks}(2)
	\task $49 y^2 z^4-56 t y z^2 + 16 t^2$
	\task $4 x^4-28 s x^2 + 49 s^2$
	\task $36 r^2 s^2 x^2 + 84 r s x + 49 $
	\task $100 r^2-140 r y + 49 y^2$
	\task $16 s^2 y^2 z^2-24 s y z + 9 $
	\task $4 x^2 y^2-4 x y z + z^2$
	\task $49 s^2-140 r s + 100 r^2$
	\task $4 x^2-12 x + 9 $
	\task $16 s^4 y^2 + 72 s^2 x y + 81 x^2$
	\task $25 r^2 x^2-30 r x + 9 $
\end{tasks}
\end{auto}

% @see : https://coopmaths.fr/alea?uuid=c4b73&id=3L12-2&n=10&d=10&s=3&s2=3&s3=4&s4=4&s5=false&alea=jd6a&cd=1&cols=1
\begin{auto}Factoriser au maximum les expressions suivantes

\begin{tasks}(2)
	\task $75 r^5 s^2 t^2 + 60 r^3 s t y + 12 r y^2$
	\task $-25 s^2 t^2 x^3 y^2-80 s t x^2 y-64 x$
	\task $-27 r y^2 z^3-180 r y z^2-300 r z$
	\task $-128 s^2 t^4 x^2 z + 32 s^4 t^2 x z^2-2 s^6 z^3$
	\task $-27 s^2 x^2 y^5 z^2-144 s x^2 y^3 z-192 x^2 y$
	\task $12 t^3 x^3 z^2-36 t^2 x^2 z + 27 t x$
	\task $-5 y^2 z + 80 y z-320 z$
	\task $-r^2 t^2 z-4 r t x^2 z-4 x^4 z$
	\task $98 r^2 s^4 t-72 t^3$
	\task $405 r x^2 y^4 z^3-360 r^2 x y^2 z^2 + 80 r^3 z$
\end{tasks}
\end{auto}

\begin{auto}Factoriser au maximum les expressions suivantes

\begin{tasks}(2)
	\task $192 s^3 z^4 + 336 s^2 z^3 + 147 s z^2$
	\task $-168 s x^3 y-98 x y^2-72 s^2 x^5$
	\task $-405 s z^2 + 360 s^2 z-80 s^3$
	\task $18 s^3 t^2 x-12 s^2 t x + 2 s x$
	\task $-8 s^3 x-98 s x^3 y^2 + 56 s^2 x^2 y$
	\task $180 s^2 t^3 + 125 t + 300 s t^2$
	\task $128 y^2 z^3-32 s y z^2 + 2 s^2 z$
	\task $-300 t^3 z^2 + 420 r^2 t^2 z^2-147 r^4 t z^2$
	\task $200 z + 360 z^2 + 162 z^3$
	\task $-36 s^2 x^2 z^2-108 s x^3 z-3 s^3 x z^3$
\end{tasks}
\end{auto}

\end{document}

% Local Variables:
% TeX-engine: luatex
% End:

\begin{comment}
	\begin{acti}
Factoriser par des mises en évidence.
\begin{tasks}(4)
\task $2 x+2 y$
\task $5 x-10 y$
\task $6 a^2-3 a$
\task $15 x^2-12 x y$
\task $x^2 y+x z^2$
\task $x^2-x^2 y$
\task $3 x^2 y-6 x z$
\task $0,8 x+0,8$
\task $7 y^2+14 y-7$
\task $x^3+x^2+2 x$
\task $13 a^5-13 a^2$
\task $x y+y$
\task $y z-y^2 z^2$
\task $y^4-y^2+y^3$
\task $24 x^4 y^3+56 x y^2$
\task $-3 a^2 b+6 a b^2$
\end{tasks}
\end{acti}
\begin{acti}
Factoriser à l'aide d'identités remarquables.
	\begin{tasks}(4)
\task $x^2+2 x y+y^2$
\task $x^2+0,4 x+0,04$
\task $x^2-y^2$
\task $9 x^2+6 x y+y^2$
\task $x^4+2 x^2 y^3+y^6$
\task $x^2-2 x+1$
\task $1-x^2$
\task $16 x^2-24 x+9$
\task $x^6-9 y^2$
\task $9 z^2-12 z+4$
\task $1-2 x+x^2$
\task $x^2 y^2+4 x y^2+4 y^2$
\task $x^4-2 x^2+1$
\task $4 x^2+8 x+4$
\task $4 a^2+12 a+9$
\task $x^2 y^2 z^2-25$
\task $9 x^6-30 x^3+25$
\task $a^2+6 a b+9 b^2$
\task $x^4-2 x^2+1$
\task $16 a^4 b^2-25$
\task $4 x^2 y^6-4 x y^3+1$
\task $x^8+2 x^4 y+y^2$
\task $1-a^2 x^8$.
\task $x^4-a^4$
	\end{tasks}
\end{acti}
\begin{acti}
	Factoriser, si possible, à l'aide de l'identité \enquote{somme-produit}\,: $x^2+(\underline{a+b}) x+(\underline{a b})=\ldots$.
\begin{tasks}(3)
\task $x^2+5 x+6$
\task $x^2-5 x+6$
\task $x^2+4 x+6$
\task $a^2-7 a+6$
\task $y^2+15 y+30$
\task $y^2-3 y+2$
\task $a^2-a-2$
\task $y^2+y-2$
\task $y^2+7 y+12$
\task $a^2+8 a+12$
\task $x^2+13 x+12$
\task $z^2+z-20$
\end{tasks}	
\end{acti}
\begin{acti}
On donne deux termes du carré d'un binôme, trouver ce binôme et le troisième terme manquant. 

Exemple : $1-4 a+\ldots=$ devient $: 1-4 a+4 a^2=(1-2 a)^2$.
\begin{tasks}(4)
\task $m^2+2 m p+\ldots=$
\task $4 c^2+a^2+\ldots=$
\task $4 a^2 x^2+4 a b x+\ldots=$
\task $m^4-2 m^2+\ldots:=$
\task $4 a^2+12 a b+\ldots=$
\task $x^2+4 y^2-\ldots=$
\task $1-2 x+\ldots=$
\task $4-x+\ldots=$
\task $25 x^2-40 x y+\ldots=$
\task $4 c^2+12 c d+\ldots=$
\task $1+4 x^2+\ldots=$
\task $x^2-2 a x+\ldots=$
\task $x^2+p x+\ldots=$
\task $c^2 d^2-2 b c d+\ldots=$
\task $9 x^2+6 x+\ldots=$
\task $4 x^2-12 x+\ldots=$
\task $9 m^2+12 m x+\ldots=$
\task $4+4 a^2 b^2+\ldots=$
\task $4 y^2+20 y+\ldots=$
\task $9 x^2+16-\ldots=$
\task $16 a^2-16 a b+\ldots=$
\task $9 x^2+4-\ldots=$
\task $\frac{x^2}{4}+x+\ldots=$
\task $16 x^2+1+\ldots=$
\end{tasks}
\end{acti}
\begin{acti}
Factoriser le plus possible.
	\begin{tasks}(4)
\task $2 x^2+14 x+24$
\task $3 x^2-30 x+63$
\task $5 x^2+15 x-50$
\task $2 a^2-2 a-24$
	\end{tasks}
\end{acti}
\end{comment}
\end{document}

