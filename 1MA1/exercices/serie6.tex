\documentclass[a4paper,12pt]{report}

\usepackage{../courslatex}
\usepackage{exos}
\usepackage{enumitem}

\setlist[enumerate]{align=left,leftmargin=1cm,itemsep=5pt,parsep=0pt,topsep=0pt,rightmargin=0.5cm}

\setlist[itemize]{align=left,labelsep=1em,leftmargin=*,itemsep=0pt,parsep=0pt,topsep=0pt,rightmargin=0cm}

\renewcommand{\titreChapitre}{Ch3\, : Calcul littéral} 
\renewcommand{\numeroSerie}{6}

\begin{document}
\vspace*{-2\baselineskip}
\textLigne{Activités}

\begin{acti}
Factoriser par des mises en évidence.
\begin{tasks}(4)
\task $2 x+2 y$
\task $5 x-10 y$
\task $6 a^2-3 a$
\task $15 x^2-12 x y$
\task $x^2 y+x z^2$
\task $x^2-x^2 y$
\task $3 x^2 y-6 x z$
\task $0,8 x+0,8$
\task $7 y^2+14 y-7$
\task $x^3+x^2+2 x$
\task $13 a^5-13 a^2$
\task $x y+y$
\task $y z-y^2 z^2$
\task $y^4-y^2+y^3$
\task $24 x^4 y^3+56 x y^2$
\task $-3 a^2 b+6 a b^2$
\end{tasks}
\end{acti}
\begin{acti}
Mettre en évidence le facteur commun.
\begin{tasks}(3)
\task $4(x+y)+5 x(x+y)$
\task $3 a(3 a-b)-8(3 a-b)$
\task $5 a^2(a-2 b)-15 a b^2(a-2 b)$
\task $9 x(x+2)^2-5 x(x+2)$
\task $4(x-y)+2 x(y-x)$
\task $x(2 x-1)+3 x^2(1-2 x)$
\end{tasks}
\end{acti}
\begin{acti}
Factoriser à l'aide d'identités remarquables.
	\begin{tasks}(4)
\task $x^2+2 x y+y^2$
\task $x^2+0,4 x+0,04$
\task $x^2-y^2$
\task $9 x^2+6 x y+y^2$
\task $x^4+2 x^2 y^3+y^6$
\task $x^2-2 x+1$
\task $1-x^2$
\task $16 x^2-24 x+9$
\task $x^6-9 y^2$
\task $9 z^2-12 z+4$
\task $1-2 x+x^2$
\task $x^2 y^2+4 x y^2+4 y^2$
\task $x^4-2 x^2+1$
\task $4 x^2+8 x+4$
\task $4 a^2+12 a+9$
\task $x^2 y^2 z^2-25$
\task $9 x^6-30 x^3+25$
\task $a^2+6 a b+9 b^2$
\task $x^4-2 x^2+1$
\task $16 a^4 b^2-25$
\task $4 x^2 y^6-4 x y^3+1$
\task $x^8+2 x^4 y+y^2$
\task $1-a^2 x^8$.
\task $x^4-a^4$
	\end{tasks}
\end{acti}
\begin{acti}
	Factoriser, si possible, à l'aide de l'identité \enquote{somme-produit}\,: $x^2+(\underline{a+b}) x+(\underline{a b})=\ldots$.
\begin{tasks}(3)
\task $x^2+5 x+6$
\task $x^2-5 x+6$
\task $x^2+4 x+6$
\task $a^2-7 a+6$
\task $y^2+15 y+30$
\task $y^2-3 y+2$
\task $a^2-a-2$
\task $y^2+y-2$
\task $y^2+7 y+12$
\task $a^2+8 a+12$
\task $x^2+13 x+12$
\task $z^2+z-20$
\end{tasks}	
\end{acti}
\begin{acti}
On donne deux termes du carré d'un binôme, trouver ce binôme et le troisième terme manquant. 

Exemple : $1-4 a+\ldots=$ devient $: 1-4 a+4 a^2=(1-2 a)^2$.
\begin{tasks}(4)
\task $m^2+2 m p+\ldots=$
\task $4 c^2+a^2+\ldots=$
\task $4 a^2 x^2+4 a b x+\ldots=$
\task $m^4-2 m^2+\ldots:=$
\task $4 a^2+12 a b+\ldots=$
\task $x^2+4 y^2-\ldots=$
\task $1-2 x+\ldots=$
\task $4-x+\ldots=$
\task $25 x^2-40 x y+\ldots=$
\task $4 c^2+12 c d+\ldots=$
\task $1+4 x^2+\ldots=$
\task $x^2-2 a x+\ldots=$
\task $x^2+p x+\ldots=$
\task $c^2 d^2-2 b c d+\ldots=$
\task $9 x^2+6 x+\ldots=$
\task $4 x^2-12 x+\ldots=$
\task $9 m^2+12 m x+\ldots=$
\task $4+4 a^2 b^2+\ldots=$
\task $4 y^2+20 y+\ldots=$
\task $9 x^2+16-\ldots=$
\task $16 a^2-16 a b+\ldots=$
\task $9 x^2+4-\ldots=$
\task $\frac{x^2}{4}+x+\ldots=$
\task $16 x^2+1+\ldots=$
\end{tasks}
\end{acti}
\begin{acti}
Factoriser le plus possible.
	\begin{tasks}(4)
\task $2 x^2+14 x+24$
\task $3 x^2-30 x+63$
\task $5 x^2+15 x-50$
\task $2 a^2-2 a-24$
	\end{tasks}
\end{acti}
\begin{acti}
(*) Pour quels entiers $x$ de 1 à 200 le nombre $x^4-x^3$ est-il le cube d'un entier?
\end{acti}
\textLigne{Exercices}
\textLigne{Automatismes}
Pas d'automatismes cette semaine, car il y eu une évaluation.

\end{document}

