\documentclass[a4paper,12pt]{report}

\usepackage{../courslatex}
\usepackage{exos}

\renewcommand{\titreChapitre}{Ch1 : Des Nombres} 
\renewcommand{\numeroSerie}{2}

\begin{document}
\vspace*{-2\baselineskip}
\textLigne{Activités}
\begin{acti}
Effectuer la division $1: 7$, par écrit, assez longtemps pour...
\begin{enumerate}
\item trouver l'écriture décimale complète de $\frac{1}{7}$\,;
\item expliquer pourquoi la longueur de la période de $\frac{1}{7}$ ne peut pas dépasser 6 chiffres\,;
\item (*) expliquer pourquoi l'écriture décimale de $\frac{1}{n}$ est finie ou infinie périodique et que sa partie périodique ne peut jamais dépasser $(n-1)$ chiffres.
\end{enumerate}
\end{acti}
\begin{acti}
Calculer et donner la réponse sous forme d'une fraction irréductible\,:
\begin{tasks}(2)
\task $0, \overline{72} \cdot 0, \overline{810}$
\task $(0,297297297 \ldots) \cdot(3,3636363 \ldots)$
\end{tasks}
\end{acti}
\begin{acti}
	Donner tous les ensembles parmis $\N, \Z, \Q$ et $\R$ au(x)quel(s) appartient chacun des nombres suivants\,: 
	\[\dfrac{2}{7}\,;\,\sqrt{100}\,;\,\sqrt{200}\,;\,\pi+1\,;\,-\sqrt{1,21}\,;\,3,14\cdot 10^5\,;\, -\dfrac{17}{2}.\] 
\end{acti}
\begin{acti}
Enumérer les éléments des ensembles suivants (donnés par une condition)\,:
\[
\{2 n-3 \mid n \in \mathbb{N} \text { et } n \leq 5\} \quad
\left\{\left.\dfrac{1}{n} \right\rvert\, n \in \mathbb{N}^*\right\}\quad 
\left\{\left.\dfrac{n-1}{n^2+n} \right\rvert\, n \in \mathbb{N}^* \text { et } n<6\right\}
\]
\end{acti}
\textLigne{Exercices}
\begin{exo}
Quelle est la millième décimale de chacun des nombres rationnels suivants~?
\begin{inlineumerate}
\item $\dfrac{1}{7}$
\item $\dfrac{17}{41}$
\end{inlineumerate}
\end{exo}

\begin{exo}
Transformer chaque nombre rationnel en fraction irréductible.
\begin{tasks}(5)
\task $0,35$ 
\task $0, \overline{35}$
\task $0, \overline{349}$
\task $0,3 \overline{49}$
\task $0,3 \overline{5}$
\task $0,34 \overline{9}$
\task $1, \overline{2}$ 
\task $3,25$
\task $15 \%$
\task $1,004$ 
\task $0, \overline{80}$
\task $0,16$
\task $2, \overline{9}$
\task $3, \overline{141}$
\end{tasks}	
\end{exo}
\begin{exo}
Entre $1$ et $2$, trouver trois nombres...
\begin{center}
	\begin{tasks}(3)
\task rationnels à développement décimal fini\,;
\task rationnels à développement décimal infini périodique\,;
\task irrationnels.
	\end{tasks}
\end{center}
Donner si possible l'écriture fractionnaire irréductible.
\end{exo}

\begin{exo}
On considère les fractions 
\[\dfrac{1}{7}, \quad \dfrac{2}{7}, \quad \dfrac{3}{7}, \quad \dfrac{4}{7}, \quad \dfrac{5}{7}, \quad \dfrac{6}{7}.\]
\begin{enumerate}
\item Trouver l'écriture décimale exacte de ces nombres à l'aide d'une calculatrice.
\item Remarquer qu'en plus d'avoir les même chiffres $1,4,2,8,5,7$, ceux-ci sont toujours dans cet ordre de gauche à droite. Par exemple, pour 2, on commence par lire 2, 8, 5, 7, puis on revient au début avec 1, 4. (On dit que les chiffres de la période sont cycliques.)
\item Les fractions dont le dénominateur est 23 ont les mêmes propriétés. Au lieu d'avoir une période cyclique de 6 chiffres, elles en ont 22 .
A l'aide d'une calculatrice uniquement (sans poser la division), trouver les 22 décimales de la période de $\dfrac{22}{23}$.
\end{enumerate}
\end{exo}





\begin{exo}
Décrire les ensembles suivants par une condition d'appartenance (comme dans l'énoncé de l'activité 4).
\begin{center}
\begin{inlineumerate}
\item $\{\ldots\,;\,-3\,;\,-1\,;\,1\,;\,3\,;\,5\,;\,7\,;\,9\,;\,11\,;\,13\,;\,\ldots\}$
\item $\{0,;\,2\,;\,4\,;\,6\,;\,8\,;\,\ldots\}$
\item $\{1\,;\,4\,;\,9\,;\,25\,;\,\ldots\}$
%\item $\left(^*\right) \left\{0\,;\,\dfrac{1}{3}\,;\,\dfrac{1}{2}\,;\,\dfrac{3}{5}\,;\,\dfrac{2}{3}\,;\,\dfrac{5}{7}\,;\,\ldots\right\}$
\end{inlineumerate}
\end{center}
\end{exo}

\begin{exo}
Dessiner un diagramme de Venn représentant simultanément les ensembles $\mathbb{N}, \mathbb{Z}, \mathbb{Q}$ et $\mathbb{R}$. Placer dans la bonne plage chacun des nombres $a_1, a_2, a_3, \ldots, a_{14}$ et $a_{15}$~:
\[
\begin{aligned}
& a_1=3,14\,;\,a_2=\frac{5}{7}\,;\,a_3=-8\,;\,a_4=5\,;\,a_5=-1,4\,;\,a_6=-\frac{7}{5}\,;\,a_7=\sqrt{8}\,;\,a_8=-\sqrt{9}\,;\,a_9=\sqrt{1000}\,;\,\\
& a_{10}=1, \overline{7}\,;\,a_{11}=1,020220222022220 \ldots\,;\,a_{12}=\frac{169}{13}\,;\,a_{13}=1,231 \cdot 10^8\,;\,a_{14}=\frac{\pi}{2}\,;\,a_{15}=2+\sqrt{3} .
\end{aligned}
\]
\end{exo}
\begin{exo}
Compléter le tableau suivant en indiquant par une croix chacun des ensembles auquel le nombre donné appartient.
\begin{center}
\begin{tabular}{c|c|c|c|c|c|} 
& $\mathbb{N}$ & $\mathbb{Z}$ & $\Q$ & $\R$ & aucun \\
\hline\rule{0pt}{4ex}
	$\dfrac{3}{2}$ & & & & & \\[2ex]
\hline\rule{0pt}{4ex}
	$\dfrac{3,14}{0,01}$ & & & & & \\[2ex]
\hline\rule{0pt}{3ex}
$\sqrt{7}$ & & & & & \\
\hline\rule{0pt}{4ex}
$\dfrac{2-\sqrt{8}}{\sqrt{2}-1}$ & & & & & \\[2ex]
\hline\rule{0pt}{3ex}
$\sqrt{9}$ & & & & & \\
\hline$\pi$ & & & & & \\
\hline\rule{0pt}{3ex}
$-\sqrt{100}$ & & & & & \\
\hline
\end{tabular}
\end{center}
\end{exo}

\begin{exo}
Pour chaque nombre, simplifier et donner les ensembles de nombres auxquels il appartiennent.
\begin{tasks}(4)
\task $\dfrac{3-7}{2}$
\task $\dfrac{4}{4-1}$
\task $2,5: 3+1$
\task $\dfrac{2^0}{1^2}$
\task $(\sqrt{2}-1): 2$
\task $\dfrac{3-\sqrt{9}}{\pi}$
\task $\sqrt{3 \cdot 27}$
\task $\dfrac{\sqrt{3}-\sqrt{12}}{\sqrt{27}}$
\task $\sqrt{\sqrt{25}-\dfrac{3}{\sqrt{9}}}$
\task $\dfrac{14}{\sqrt{25}-\sqrt{144}}$
\task $\dfrac{\sqrt{2}}{\sqrt{81}-\dfrac{16}{2}}$
\task $\dfrac{5-\sqrt{3}}{\sqrt{3}-5}$
\end{tasks}
\end{exo} 


\textLigne{Automatismes}
%https://coopmaths.fr/alea/?uuid=99b29&id=11NO1-5&n=10&d=10&s=1&alea=z1xik&cd=1&uuid=cc300&n=5&d=10&alea=z1xik&cd=1&uuid=4771d&n=7&d=10&s=1-2&alea=z1xik&cd=1&lang=fr-CH&v=latex

\begin{auto}Donner, si possible, une écriture simplifiée des calculs suivants.

\begin{tasks}(4)
	\task $\left(-9 \sqrt{7}\right)^{2}$
	\task $8 \sqrt{7}\cdot 8\sqrt{7}$
	\task $ \sqrt{\dfrac{484}{4}}$
	\task $\sqrt{6}+\sqrt{5}$
	\task $ -4 \sqrt{6}\left( -9  -9\sqrt{6}\right)$
	\task $  \sqrt{64}+\sqrt{25}$
	\task $ \sqrt{12}\cdot \sqrt{4}$
	\task $\sqrt{3}+\sqrt{11}$
	\task $ -9 \sqrt{7}\left( 5  -6\sqrt{7}\right)$
	\task $3 \sqrt{2}\cdot 4\sqrt{2}$
\end{tasks}

\end{auto}

% @see : https://coopmaths.fr/alea?uuid=cc300&id=can2C06.js&n=5&d=10&alea=z1xik&cd=1&cols=1
\begin{auto}
Écrire sous la forme $a\sqrt{b}$ où $a$ et $b$ sont des entiers avec $b$ le plus petit possible.

\begin{tasks}(5)
	\task  $\sqrt{500}$
	\task  $\sqrt{54}$
	\task  $\sqrt{27}$
	\task  $\sqrt{98}$
	\task  $\sqrt{44}$
\end{tasks}
\end{auto}

% @see : https://coopmaths.fr/alea?uuid=4771d&id=2N32-7.js&n=7&d=10&s=1-2&alea=z1xik&cd=1&cols=2
\begin{auto} Trouver une fraction égale à celle proposée en supprimant la racine carrée de son dénominateur.

\begin{tasks}(4)
	\task $\dfrac{ 9 }{\sqrt{2}} $
	\task $\dfrac{ 4 }{3+4\sqrt{10}} $ 
	\task $\dfrac{ 11 }{8+7\sqrt{11}} $ 
	\task $\dfrac{ 10 }{\sqrt{7}} $
	\task $\dfrac{ 3 }{\sqrt{6}} $
	\task $\dfrac{ 11 }{3+3\sqrt{6}} $ 
	\task $\dfrac{ 9 }{\sqrt{6}} $
\end{tasks}

\end{auto}
\begin{comment}
\begin{auto}
Calculer et donner la réponse sous la forme d'un entier ou d'une fraction irréductible ne contenant pas de racines.
	\begin{tasks}(4)
\task $\sqrt{16+9}$
\task $\sqrt{0,49}$
\task $\sqrt{\dfrac{16}{9}}$
\task $\sqrt[3]{27 \cdot 125}$
\task $\sqrt{16 \cdot 9}$
\task $\sqrt{20}+\sqrt{5}$
\task $\dfrac{\sqrt{6}}{\sqrt{216}}$
\task $\sqrt{20} \cdot \sqrt{45}$
\task $\sqrt{1600}$
\task $\sqrt{100-36}$
\task $\dfrac{3 \sqrt{20}}{\sqrt{5}}$
\task $\dfrac{\sqrt{30}}{9 \sqrt{15}}$
	\end{tasks}
\end{auto}
\begin{auto}
Calculer et donner la réponse sous la forme d'un entier ou d'une fraction irréductible ne contenant pas de racines.
	\begin{tasks}(3)
\task $\sqrt{25-16}$
\task $\sqrt[3]{2} \cdot \sqrt[3]{-4}$
\task $\sqrt{2} \cdot \sqrt{8}$
\task $\sqrt[3]{10^6}+0,1^2$
\task $\sqrt{8} \cdot\left(\dfrac{\sqrt{2}}{3}+\dfrac{1}{\sqrt{32}}\right)$
\task $\dfrac{\sqrt{2}}{\sqrt{8}}$
\task $\sqrt[3]{3^4} \cdot \sqrt[3]{3^8}$
\task $\sqrt{\dfrac{16}{81}}+\dfrac{5}{6} \div\left(\dfrac{5}{27} \cdot \sqrt{\dfrac{27}{12}}\right)$
\task $\sqrt{10^2-8^2}$
\task $\sqrt{2} \cdot(\sqrt{2}+\sqrt{8})$
\task*(2) $\left(\sqrt{\dfrac{1}{4}-\dfrac{1}{9}} \cdot \sqrt{\dfrac{4}{5}}\right) \div\left(\sqrt{\dfrac{1}{4}}+\sqrt{\dfrac{1}{9}}\right)$
\end{tasks}
\end{auto}

\begin{auto}{
Extraire les carrés des racines suivantes, c'est-à-dire écrire les racines sous la forme $a\sqrt{b}$ avec $b$ minimal.	
\begin{tasks}(6)
	\task $\sqrt{128}$
	\task $\sqrt{90}$
	\task $\sqrt{300}$
	\task $\sqrt{720}$
	\task $\sqrt{405}$
	\task $\sqrt{810}$
\end{tasks}
}
\end{auto}


%\begin{auto}{
%Écrire sous la forme $a\sqrt{b}$ avec $a,b\in \N$ et $b$ minimal.
%\begin{tasks}(4)
%	\task $\sqrt{32}+\sqrt{8}$
%	\task $\sqrt{192}+\sqrt{432}$
%	\task $\sqrt{72}-\sqrt{128}$
%	\task $\sqrt{160}+\sqrt{490}$
%\end{tasks}
%}
%\end{auto}

\begin{auto}{
Calculer et extraire les entiers du résultat.		
\begin{tasks}[before-skip  -\parskip](4)
	\task $\sqrt{18}\cdot \sqrt{2}$
	\task $\sqrt{32}\cdot \sqrt{8}$
	\task $\sqrt{12}\cdot \sqrt{6}$
	\task $\sqrt{11}\cdot \sqrt{44}$
	\task $\sqrt{8}\cdot \sqrt{48}$
	\task $2\sqrt{3}\cdot \sqrt{12}$
	\task $\sqrt{30}\cdot \sqrt{20}$
	\task $\sqrt{14}\cdot \sqrt{6}$
\end{tasks}
}
\end{auto}
\begin{auto}{
Calculer et extraire les entiers du résultat.		
\begin{tasks}(6)
	\task $\dfrac{\sqrt{6}}{\sqrt{24}}$
	\task $\dfrac{\sqrt{32}}{\sqrt{8}}$
	\task $\sqrt{\dfrac{25}{16}}$
	\task $\dfrac{2\sqrt{2}}{\sqrt{16}}$
	\task $\sqrt{\dfrac{72}{81}}$
	\task $\dfrac{\sqrt{6}}{\sqrt{42}}$
\end{tasks}	
}
\end{auto}
\begin{auto}
Transformer afin de supprimer les racines au dénominateur. Réduire au maximum le résultat. 
\begin{tasks}(6)
	\task $\dfrac{3}{\sqrt{5}}$ 
	\task $\dfrac{1+\sqrt{2}}{\sqrt{3}}$
	\task $\dfrac{4}{\sqrt{7}-1}$
	\task $\dfrac{9}{5+\sqrt{2}}$
	\task $\dfrac{1+\sqrt{3}}{\sqrt{3}}$
	\task $\dfrac{5}{\sqrt{2}+\sqrt{7}}$
\end{tasks}
\end{auto}

	\begin{exo}
	Déterminer la valeur de 
	\[\dfrac{101+103+105+\ldots+199}{1+3+5+\ldots+99}.\]
\end{exo}
\end{comment}
\end{document}

