\documentclass[a4paper,12pt]{report}

\usepackage{../courslatex}
\usepackage{exos}
\usepackage{enumitem}
\renewcommand{\typeDoc}{Corrigé série }

\setlist[enumerate]{align=left,leftmargin=1cm,itemsep=5pt,parsep=0pt,topsep=0pt,rightmargin=0.5cm}

\setlist[itemize]{align=left,labelsep=1em,leftmargin=*,itemsep=0pt,parsep=0pt,topsep=0pt,rightmargin=0cm}

\renewcommand{\titreChapitre}{Ch1 : Des Nombres} 
\renewcommand{\numeroSerie}{1}

\begin{document}
\vspace*{-2\baselineskip}
\textLigne{Exercices}

\begin{core}
	On note un nombre à cinq chiffres  \[a+b\cdot 10+c\cdot 10^2+d\cdot 10^3+e\cdot 10^4 \quad\text{ où } a,b,c,d,e\in \N, e\neq 0\]
	Si le nombre a quatre chiffres, alors on prend $e=0$ et $d\neq 0$.

	\begin{tasks}
	\task On a $a=4$ et $b=2$. Par ailleurs la somme $a+b+c+d+e$ doit être divisible par $3$ pour que le nombre soit un multiple de $3$. On a $2+4=6$ qui est déjà un multiple de $3$. Le nombre recherché est donc $99924$. 
	\task Le nombre recherché est $1224$.
	\task Le nombre recherché est $2046$.
	\task Le nombre recherché est $9753$.
	\end{tasks}
\end{core}
\begin{core}
	\phantom{.}
	\begin{tasks}(3)
		\task[] Pour $n=0$ on obtient $41$.	
		\task[] Pour $n=1$ on obtient $43$.	
		\task[] Pour $n=2$ on obtient $47$.	
		\task[] Pour $n=3$ on obtient $53$.	
		\task[] Pour $n=4$ on obtient $61$.	
		\task[] Pour $n=5$ on obtient $71$.	
		\task[] Pour $n=6$ on obtient $83$.	
	\end{tasks}
\end{core}
\begin{core}
	On utilise surtout la décomposition de $10=2\cdot 5$ et donc que $10^n=2^n\cdot 5^n$.
%corriger cet exercice
	\begin{tasks}(3)
\task $10=2\cdot 5$
\task $10^2=2^2\cdot 5^2$
\task $100000=2^5\cdot 5^5$
\task $24 \cdot 1000=2^6\cdot 3\cdot 5^3$
\task $38 \cdot 10^5=2^6\cdot 5^5\cdot 19$
\task $25000=5^5\cdot 2^3$
\task $28000= 2^5\cdot 5^3\cdot 7$
\task $66000=2^4\cdot 3\cdot 5^3\cdot 11$
\task $16000=2^7\cdot 5^3$
\task $3600000=2^7\cdot 3^2\cdot 5^5$
	\end{tasks}
\end{core}
\begin{core}
	\phantom{.}
\begin{tasks}(3)
\task $2^3 \cdot 3 \cdot 5^2=600$
\task $2^3 \cdot 7 \cdot 5^3=7000$
\task $2^4 \cdot 5^2=400$
\task $2^3 \cdot 5^4=5000$
\task $2^5 \cdot 5^5 \cdot 7=700000$
\task $2 \cdot 3 \cdot 5 \cdot 7=210$
\task $2^4 \cdot 5^4 \cdot 11=110000$
\task $2^6 \cdot 5^3=8000$
\task $2^4 \cdot 5^6=250000$
\task $2^3 \cdot 5^3 \cdot 7^2=49000$
\task $2^4 \cdot 3 \cdot 5^2=1200$
\task $2^6 \cdot 3 \cdot 5^8=75000000$
\end{tasks}
\end{core}
\textLigne{Automatismes}
\begin{cora}\phantom{ }

\begin{tasks}(6)
\task $63$
\task $-100$
\task $-18$
\task $-14$
\task $8$
\task $-23$
\end{tasks}

\end{cora}

\begin{cora}\phantom{ }

\begin{tasks}(5)
\task ${{25}}$
\task ${{-14}}$
\task ${{4}}$
\task ${{58}}$
\task ${{-36}}$
\end{tasks}

\end{cora}

\begin{cora}\phantom{ }

\begin{tasks}(5)
\task $\dfrac{59}{8}$
\task $\dfrac{5}{6}$ 
\task $\dfrac{11}{24}$
\task $\dfrac{11}{28}$ 
\task $\dfrac{13}{50}$ 
\end{tasks}

\end{cora}

\begin{cora}\phantom{ }

\begin{tasks}(5)
\task $\dfrac{35}{32}$
\task $\dfrac{9}{8}$
\task $\dfrac{8}{5}$
\task $\dfrac{3}{14}$
\task $\dfrac{5}{4}$
\end{tasks}

\end{cora}

\begin{cora}\phantom{ }
\begin{tasks}(4)
\task $-\dfrac{17}{32}$

\task $\dfrac{5}{2}$

\task $\dfrac{17}{5}$

\task $\dfrac{20}{3}$

\task $\dfrac{14}{9}$

\task $-\dfrac{13}{4}$ 

\task $\dfrac{27}{7}$

\task $\dfrac{10}{3}$ 

\end{tasks}


\end{cora}

\begin{cora}\phantom{ }

\begin{tasks}(5)
\task ${{\dfrac{1}{12}}}$
\task  ${{\dfrac{16}{25}}}$
\task ${{\dfrac{7}{24}}}$
\task ${{\dfrac{1}{15}}}$
\task ${{\dfrac{25}{63}}}$
\end{tasks}

\end{cora}
\end{document}

