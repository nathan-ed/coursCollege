\documentclass[a4paper,12pt]{report}

\usepackage{../courslatex}
\usepackage{exos}
\usepackage{enumitem}
\renewcommand{\typeDoc}{Corrigé série }

\setlist[enumerate]{align=left,leftmargin=1cm,itemsep=5pt,parsep=0pt,topsep=0pt,rightmargin=0.5cm}

\setlist[itemize]{align=left,labelsep=1em,leftmargin=*,itemsep=0pt,parsep=0pt,topsep=0pt,rightmargin=0cm}

\renewcommand{\titreChapitre}{Ch1 : Des Nombres} 
\renewcommand{\numeroSerie}{2}

\begin{document}
\vspace*{-2\baselineskip}
\textLigne{Exercices}
\begin{core}
	\phantom{.}
	\begin{tasks}
		\task On a $1\div 7=0,\overline{142857}$, donc une période de six chiffres. On divise $1000$ par $6$ et on obtient $166$ reste $4$. Le millième chiffre après la virgule est le quatrième chiffre de la période soit $8$. 
		\task On a $17\div 41=0,\overline{41463}$, donc une période de cinq chiffres. On divise $1000$ par $5$ et on obtient $200$ reste $0$. Le millième chiffre après la virgule est le cinquième chiffre de la période soit $3$. 
	\end{tasks}
\end{core}
\begin{core}
	\phantom{.}
\begin{tasks}(5)
	\task $\dfrac{35}{10}=\dfrac{5}{2}$
	\task $\dfrac{35}{99}$
	\task $\dfrac{349}{999}$
	\task $\dfrac{3}{10}+\dfrac{49}{990}=\dfrac{173}{495}$
	\task $\dfrac{34}{100}+\dfrac{9}{900}=\dfrac{7}{20}$. 

	Noter que $0,\overline{9}=1$ et que $0,0\overline{9}=0,01$.
	\task $1+\dfrac{2}{9}=\dfrac{11}{9}$
	\task $\dfrac{325}{100}=\dfrac{13}{4}$
	\task $\dfrac{15}{100}=\dfrac{3}{20}$
	\task $1+\dfrac{4}{10000}=\dfrac{251}{250}$
	\task $\dfrac{80}{99}$
	\task $\dfrac{16}{100}=\dfrac{4}{25}$
	\task $3$
	\task $3+\dfrac{141}{999}=\dfrac{1046}{333}$
\end{tasks}
\end{core}
\begin{core}
	\phantom{.}
	\begin{tasks}(3)
		\task $\dfrac{12}{10};\, \dfrac{13}{10};\,\dfrac{14}{10};\,$
		\task $1,\overline{1}=\dfrac{10}{9};\, \dfrac{11}{9};\, \dfrac{12}{9};\,$
		\task $\sqrt{2};\, \sqrt{3};\, \dfrac{\sqrt{5}}{2}.$	
	\end{tasks}
\end{core}
\begin{core}
	\phantom{.}
	\begin{tasks}
		\task $\dfrac{1}{7}=0,\overline{142857};\,\dfrac{2}{7}=0,\overline{285714};\,\dfrac{3}{7}=0,\overline{428571};\,\dfrac{4}{7}=0,\overline{571428};\,\dfrac{5}{7}=0,\overline{714285};\,\dfrac{6}{7}=0,\overline{857142}$.
		\task À remarquer.
		\task $\dfrac{22}{23}=0,\overline{9565217391304347826086}$
	\end{tasks}
\end{core}
\begin{core}
	\phantom{.}
	\begin{tasks}(3)
		\task $E=\{2n+1 \mid n\in \N\}$
		\task $S=\{2n \mid n\in \N\}$
		\task $I=\{n^2 \mid n\in \N^*\}$
	\end{tasks}
\end{core}
\begin{core}
	\phantom{.}
\begin{center}
	\def\natEl{(0,0) ellipse (2cm and 1cm)}
	\def\intEl{(0.5,0) ellipse (3cm and 2cm)}
	\def\ratEl{(1,0) ellipse (4cm and 3cm)}
	\def\realEl{(1.5,0) ellipse (5cm and 4cm)}
\begin{tikzpicture}
        \node (A) at (1.5,0) {$\N$};
        \node (B) at (3,0) {$\Z$};
        \node (C) at (4.5,0) {$\Q$};
        \node (D) at (6,0) {$\R$};
        \node (E) at (0.3,0.2) {$a_4$};
	\node (F) at (-0.5,-0.2) {$a_{13}$};
	\node (G) at (2,2.5) {$a_1$};
	\node (H) at (1,-0.5) {$a_{12}$};
        \node (I) at (1.2,-1.3) {$a_3$};
	\node (J) at (1.1,1.5) {$a_8$};
        \node (K) at (3,1.9) {$a_2$};
        \node (L) at (2.3,-2.3) {$a_5$};
	\node (M) at (3.9,1.1) {$a_6$};
	\node (N) at (3.4,-1.5) {$a_{10}$};
	\node (O) at (5.2,1.5) {$a_7$};
        \node (P) at (5.5,-1) {$a_9$};	
	\node (Q) at (5,-1.5) {$a_{14}$};	
	\node (R) at (4.9,2.2) {$a_{15}$};	
	\node (S) at (3.9,0.4) {$a_{11}$};	
	\draw \natEl;
	\draw \intEl;
	\draw \ratEl;
	\draw \realEl;
\end{tikzpicture}
\end{center}

\end{core}
\newpage
\begin{core}
	\phantom{.}
\begin{center}
\begin{tabular}{c|c|c|c|c|c|} 
& $\mathbb{N}$ & $\mathbb{Z}$ & $\Q$ & $\R$ & aucun \\
\hline\rule{0pt}{4ex}
	$\dfrac{3}{2}$ & & &X &X& \\[2ex]
\hline\rule{0pt}{4ex}
	$\dfrac{3,14}{0,01}$ &X&X&X&X& \\[2ex]
\hline\rule{0pt}{3ex}
$\sqrt{7}$ & & &X& & \\
\hline\rule{0pt}{4ex}
$\dfrac{2-\sqrt{8}}{\sqrt{2}-1}$ &&X&X&X& \\[2ex]
\hline\rule{0pt}{3ex}
$\sqrt{9}$ &X&X&X&X& \\
\hline$\pi$ & & & &X& \\
\hline\rule{0pt}{3ex}
$-\sqrt{100}$ & &X&X&X& \\
\hline
\end{tabular}
\end{center}
\end{core}
\begin{core}
	\phantom{}
\begin{tasks}(2)
	\task $\dfrac{3-7}{2}=\dfrac{-4}{2}=-2\in \Z$
	\task $\dfrac{4}{4-1}=\dfrac{4}{3}\in \Q$
	\task $2,5: 3+1=\dfrac{25}{30}+1=\dfrac{5}{6}+1=\dfrac{11}{6}\in\Q$
	\task $\dfrac{2^0}{1^2}=\dfrac{1}{1}=1\in \N$
	\task $(\sqrt{2}-1): 2=\dfrac{\sqrt{2}}{2}-\dfrac{1}{2}\in\R$
	\task $\dfrac{3-\sqrt{9}}{\pi}=\dfrac{3-3}{\pi}=0\in \N$
	\task $\sqrt{3 \cdot 27}=\sqrt{81}=9\in \N$
	\task $\dfrac{\sqrt{3}-\sqrt{12}}{\sqrt{27}}=\dfrac{\sqrt{3}-2\sqrt{3}}{3\sqrt{3}}=\dfrac{1-2}{3}=-\dfrac{1}{3}\in\Q$
	\task $\sqrt{\sqrt{25}-\dfrac{3}{\sqrt{9}}}=\sqrt{5-\dfrac{3}{3}}=\sqrt{4}=2\in \N$
	\task $\dfrac{14}{\sqrt{25}-\sqrt{144}}=\dfrac{14}{5-12}=\dfrac{14}{-7}=-2\in\Z$
	\task $\dfrac{\sqrt{2}}{\sqrt{81}-\dfrac{16}{2}}\dfrac{\sqrt{2}}{9-8}=\dfrac{\sqrt{2}}{1}=\sqrt{2}\in \R$
	\task $\dfrac{5-\sqrt{3}}{\sqrt{3}-5}=\dfrac{5-\sqrt{3}}{-(5-\sqrt{3})}=-1\in \Z$
\end{tasks} 
\end{core}
\textLigne{Automatismes}
\begin{cora}\phantom{ }

\begin{tasks}(3)
\task $\left(-9 \sqrt{7}\right)^{2}=(-9)^{2}\cdot \left(\sqrt{7}\right)^{2}$
                        $\phantom{\left(-9 \sqrt{7}\right)^{2}}$\\
                        $\phantom{\left(-9 \sqrt{7}\right)^{2}}=81\cdot 7$\\
                        $\phantom{\left(-9 \sqrt{7}\right)^{2}}=567$
\task $ 8 \sqrt{7}\cdot 8 \sqrt{7}=8\cdot 8 \sqrt{7} \cdot \sqrt{7}$\\
                        $\phantom{8 \sqrt{7}\cdot 8 \sqrt{7}}=64\cdot 7$\\
                        $\phantom{8 \sqrt{7}\cdot 8 \sqrt{7}}=448$
\task $ \sqrt{\dfrac{484}{4}}= \sqrt{\dfrac{11^{2}\cdot4}{4}}$\\
                        $\phantom{\sqrt{\dfrac{484}{4}}}=\sqrt{11^{2}}$\\
                        $\phantom{\sqrt{\dfrac{484}{4}}}=11$ 
\task $  \sqrt{6}+\sqrt{5}$ n'est pas simplifiable
\task $-4 \sqrt{6}\left( -9  -9\sqrt{6}\right)=
                        -4 \sqrt{6}\cdot (-9)-4 \sqrt{6}\cdot (-9)\sqrt{6}$\\
                        $\phantom{-4 \sqrt{6}\left( -9  -9\sqrt{6}\right)}=36\sqrt{6}-4\cdot (-9)\cdot 6$\\
                        $\phantom{-4 \sqrt{6}\left( -9  -9\sqrt{6}\right)}=36\sqrt{6}+216$
\task $  \sqrt{64}+\sqrt{25}=8+5=13$ 
\task $ \sqrt{12}\cdot \sqrt{4}=\sqrt{12\cdot4}$\\
                        $\phantom{\sqrt{12}\cdot \sqrt{4}}=\sqrt{3\cdot4\cdot4}$\\
                        $\phantom{\sqrt{12}\cdot \sqrt{4}}=4\sqrt{3}$ 
\task $  \sqrt{3}+\sqrt{11}$ n'est pas simplifiable
\task $-9 \sqrt{7}\left( 5  -6\sqrt{7}\right)=
                        -9 \sqrt{7}\cdot 5-9 \sqrt{7}\cdot (-6)\sqrt{7}$\\
                        $\phantom{-9 \sqrt{7}\left( 5  -6\sqrt{7}\right)}=-45\sqrt{7}-9\cdot (-6)\cdot 7$\\
                        $\phantom{-9 \sqrt{7}\left( 5  -6\sqrt{7}\right)}=-45\sqrt{7}+378$
\task $ 3 \sqrt{2}\cdot 4 \sqrt{2}=3\cdot 4 \sqrt{2} \cdot \sqrt{2}$\\
                        $\phantom{3 \sqrt{2}\cdot 4 \sqrt{2}}=12\cdot 2$\\
                        $\phantom{3 \sqrt{2}\cdot 4 \sqrt{2}}=24$
\end{tasks}

\end{cora}

\begin{cora}\phantom{ }

\begin{tasks}
\task On simpifie $\sqrt{500}$ en $10\sqrt{5}$ , car
    $\sqrt{500}=\sqrt{10^2\cdot 5} =
    \sqrt{10^2}\cdot \sqrt{5} =
    {{10\sqrt{5}}}$.
\task On simpifie $\sqrt{54}$ en $3\sqrt{6}$ , car
    $\sqrt{54}=\sqrt{3^2\cdot 6} =
    \sqrt{3^2}\cdot \sqrt{6} =
    {{3\sqrt{6}}}$.
\task On simpifie $\sqrt{27}$ en $3\sqrt{3}$ , car
    $\sqrt{27}=\sqrt{3^2\cdot 3} =
    \sqrt{3^2}\cdot \sqrt{3} =
    {{3\sqrt{3}}}$.
\task On simpifie $\sqrt{98}$ en $7\sqrt{2}$ , car
    $\sqrt{98}=\sqrt{7^2\cdot 2} =
    \sqrt{7^2}\cdot \sqrt{2} =
    {{7\sqrt{2}}}$.
\task On simpifie $\sqrt{44}$ en $2\sqrt{11}$ , car
    $\sqrt{44}=\sqrt{2^2\cdot 11} =
    \sqrt{2^2}\cdot \sqrt{11} =
    {{2\sqrt{11}}}$.
\end{tasks}

\end{cora}

\begin{cora}\phantom{ }


\begin{tasks}(3)
\task Il suffit de multiplier le numérateur et le dénominateur de la fraction par $\sqrt{2}$.\\$A=\dfrac{ 9 }{\sqrt{2}}=\dfrac{ 9 \cdot \sqrt{2}} {\sqrt{2} \cdot \sqrt{2}} $

\medskip
$A={{\dfrac{9\sqrt{2}}{2}}}$
\task Ici, il faut multiplier le numérateur et le dénominateur de la fraction par  $3-4\sqrt{10}$.

\medskip
$B=\dfrac{ 4 }{3+4\sqrt{10}}$

\medskip
$B=\dfrac{ 4\cdot (3-4\sqrt{10}) }{(3+4\sqrt{10})(3-4\sqrt{10})}$

\medskip
$B=\dfrac{ 12 -16\sqrt{10}}{(3)^2-\left(4\sqrt{10}\right)^2}$ 

\medskip
$B=\dfrac{ 12 -16\sqrt{10}}{9-(16\cdot10)}$

\medskip
$B=\dfrac{ 12 -16\sqrt{10}}{9-160}$

\medskip
$B=\dfrac{12 -16\sqrt{10}}{-151}$

\medskip
$B={{\dfrac{-12 +16\sqrt{10}}{151}}}$
\task Ici, il faut multiplier le numérateur et le dénominateur de la fraction par  $8-7\sqrt{11}$.

\medskip
$C=\dfrac{ 11 }{8+7\sqrt{11}}$

\medskip
$C=\dfrac{ 11\cdot (8-7\sqrt{11}) }{(8+7\sqrt{11})(8-7\sqrt{11})}$

\medskip
$C=\dfrac{ 88 -77\sqrt{11}}{(8)^2-\left(7\sqrt{11}\right)^2}$ 

\medskip
$C=\dfrac{ 88 -77\sqrt{11}}{64-(49\cdot11)}$

\medskip
$C=\dfrac{ 88 -77\sqrt{11}}{64-539}$

\medskip
$C=\dfrac{88 -77\sqrt{11}}{-475}$

\medskip
$C={{\dfrac{-88 +77\sqrt{11}}{475}}}$
\task Il suffit de multiplier le numérateur et le dénominateur de la fraction par $\sqrt{7}$.\\$D=\dfrac{ 10 }{\sqrt{7}}=\dfrac{ 10 \cdot \sqrt{7}} {\sqrt{7} \cdot \sqrt{7}} $

\medskip
$D={{\dfrac{10\sqrt{7}}{7}}}$
\task  Il suffit de multiplier le numérateur et le dénominateur de la fraction par $\sqrt{6}$.\\$E=\dfrac{ 3 }{\sqrt{6}}=\dfrac{ 3 \cdot \sqrt{6}} {\sqrt{6} \cdot \sqrt{6}} $

\medskip
$E=\dfrac{3\sqrt{6}}{6}$

\medskip
 $E={{\dfrac{\sqrt{6}}{2}}}$
\task Ici, il faut multiplier le numérateur et le dénominateur de la fraction par  $3-3\sqrt{6}$.

\medskip
$F=\dfrac{ 11 }{3+3\sqrt{6}}$

\medskip
$F=\dfrac{ 11\cdot (3-3\sqrt{6}) }{(3+3\sqrt{6})(3-3\sqrt{6})}$

\medskip
$F=\dfrac{ 33 -33\sqrt{6}}{(3)^2-\left(3\sqrt{6}\right)^2}$ 

\medskip
$F=\dfrac{ 33 -33\sqrt{6}}{9-(9\cdot6)}$

\medskip
$F=\dfrac{ 33 -33\sqrt{6}}{9-54}$

\medskip
$F=\dfrac{ 33 -33\sqrt{6}}{-45}$

\medskip
$F=\dfrac{11 -11\sqrt{6}}{-15}$

\medskip
$F={{\dfrac{-11 +11\sqrt{6}}{15}}}$
\task Il suffit de multiplier le numérateur et le dénominateur de la fraction par $\sqrt{6}$.\\$G=\dfrac{ 9 }{\sqrt{6}}=\dfrac{ 9 \cdot \sqrt{6}} {\sqrt{6} \cdot \sqrt{6}} $

\medskip
$G=\dfrac{9\sqrt{6}}{6}$

\medskip
 $G={{\dfrac{3 \sqrt{6}}{2}}}$
\end{tasks}


\end{cora}
\end{document}

