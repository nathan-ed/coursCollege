\documentclass[a4paper,12pt]{report}

\usepackage{../courslatex}
\usepackage{exos}
\usepackage{enumitem}
\renewcommand{\typeDoc}{Corrigé série }

\setlist[enumerate]{align=left,leftmargin=1cm,itemsep=5pt,parsep=0pt,topsep=0pt,rightmargin=0.5cm}

\setlist[itemize]{align=left,labelsep=1em,leftmargin=*,itemsep=0pt,parsep=0pt,topsep=0pt,rightmargin=0cm}

\renewcommand{\titreChapitre}{Ch2 : Ensembles et intervalles réels} 
\renewcommand{\numeroSerie}{3}

\begin{document}
\vspace*{-2\baselineskip}
\textLigne{Exercices}
\begin{core}
	\phantom{}
	\begin{tasks}(2)
		\task $I\cap J=\interval[open right]{-3}{4}\cap \interval[open right]{-2}{0}=\interval[open right]{-2}{0}$ 
		\task $I\cup K=\interval[open right]{-3}{4}\cup \interval[open left]{-5}{3}=\interval[open]{-5}{4}$
		\task $I\cap K=\interval[open right]{-3}{4}\cap \interval[open left]{-5}{3} = \interval[]{-3}{3}$
		\task $I\setminus K=\interval[open right]{-3}{4}\setminus \interval[open left]{-5}{3}=\interval[open]{3}{4}$
		\task $K\setminus I=\interval[open left]{-5}{3}\setminus \interval[open right]{-3}{4}=\interval[open]{-5}{-3}$
	\end{tasks}
\end{core}
\begin{core}
	\phantom{}
	\begin{tasks}(4)
		\task $\interval[]{-3}{2}$
		\task $\interval[open right]{3}{+\infty}$
		\task $\interval[open]{-\infty}{-1}$
		\task $\interval[open left]{-2}{4}$
		\task $\interval[open left]{-\dfrac{3}{2}}{-\dfrac{1}{2}}$
		\task $\interval[open left]{-\infty}{1+\sqrt{2}}$
		\task $\interval[open]{-\infty}{+\infty}$
		\task $\interval[open]{-\infty}{-2}\cup \interval[open right]{4}{+\infty}$
	\end{tasks}
\end{core}
\def\firstcircle{(0,0) circle (1cm)}
\def\thirdcircle{(0:1cm) circle (1cm)}
\begin{core}
	Il y a plusieurs réponses correctes. 
	\begin{tasks}(2)
		\task $A=\{1;2\}$ et $B=\{0;3;4\}$
\begin{center}
\begin{tikzpicture}
        \draw \firstcircle node[above, yshift=1cm] {$A$} node[left] {$1;2$};
        \draw \thirdcircle node [above, yshift=1cm] {$B$} node[right] {$0;3;$} node[right, yshift=-0.4cm] {$4$};	
\end{tikzpicture}
\end{center}
		\task $A=\{0;1;2;3;4\}$ et $B=\{2;3;4\}$
\begin{center}
\begin{tikzpicture}
        \draw \firstcircle node[above, yshift=1cm] {$A$} node[left] {$0;1$};
        \draw \thirdcircle node [above, yshift=1cm] {$B$} node[left] {$2;3;$} node[left, yshift=-0.4cm, xshift=-0.2cm] {$4$};	
\end{tikzpicture}
\end{center}
		\task $A=\{0;2;3;4\}$ et $B=\{0;1\}$
\begin{center}
\begin{tikzpicture}
	\draw \firstcircle node[above, yshift=1cm] {$A$} node[left] {$2;3;$} node[left,xshift=-0.2cm, yshift=-0.4cm] {$4$};
        \draw \thirdcircle node [above, yshift=1cm] {$B$} node[left, xshift=-0.2cm] {$0$} node[right] {$1$};	
\end{tikzpicture}
\end{center}
		\task $A=\{0;2;3\}$ et $B=\{1;4\}$
\begin{center}
\begin{tikzpicture}
	\draw \firstcircle node[above, yshift=1cm] {$A$} node[left] {$0;2;$} node[left,xshift=-0.2cm, yshift=-0.4cm] {$3$};
        \draw \thirdcircle node [above, yshift=1cm] {$B$} node[right] {$1;4$};
\end{tikzpicture}
\end{center}

	\end{tasks}
\end{core}
\def\trianglecircle{(0,0) ellipse (3cm and 2cm)}
\def\rectcircle{(-1,0) ellipse (1.5cm and 1.25cm)}
\def\isocircle{(0:1.2cm) circle (1cm)}
\def\equicircle{(0:1.5cm) circle (0.5cm)}

\begin{core}
	\phantom{}
\begin{tasks}(2)
	\task La taille des diagrammes n'est pas représentative de la \enquote{taille} des ensembles.	

\begin{center}
\begin{tikzpicture}
        % Draw the shapes
        \draw \trianglecircle node[above, yshift=2cm] {$T$};
        \draw \rectcircle node[above, yshift=1.3cm,xshift=0.3cm] {$R$};
        \draw \isocircle node[above, yshift=1cm] {$I$};
        \draw \equicircle node[] {$E$};

        % Find and mark the intersection between isocircle and rectcircle
        %\path [name path=isocircle] \isocircle;
        %\path [name path=rectcircle] \rectcircle;
        %\path [name intersections={of=isocircle and rectcircle, by=intersection}];
        %\fill[red] (intersection) circle (2pt);
    \end{tikzpicture}
\end{center}
\task 
\begin{itemize}
	\item $I\cap E=E$, car l'ensemble des triangles \\ équilatéraux est contenu dans l'ensemble de triangles isocèles. 
	\item $R\cap E=\emptyset$, car il n'existe aucun triangle qui est équilatéral et rectangle (par le théorème de Pythagore, si $a\in \R^*_+$ est la longueur du côté du triangle, alors $a^2+a^2\neq a^2$).
	\item $I\cap R$ est l'ensemble des triangles dont les deux cathètes mesure $a\in \R^*_+$ et l'hypoténuse mesure $a\sqrt{2}$ (par Pythagore).
\end{itemize}
\end{tasks}

\end{core}
\begin{core}
	\phantom{}
	\begin{tasks}(3)
		\task Vrai
		\task Faux, semi-ouvert à gauche
		\task Vrai
		\task Faux, ce n'est pas l'intervalle
		\task Vrai
		\task Faux, il y appartient
		\task Faux, $0$ est dans l'intersection
		\task Vrai
		\task Vrai
	\end{tasks}
\end{core}
\textLigne{Automatismes}
\end{document}
