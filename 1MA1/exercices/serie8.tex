\documentclass[a4paper,12pt]{report}

\usepackage{../courslatex}
\usepackage{exos}
\usepackage{enumitem}

\setlist[enumerate]{align=left,leftmargin=1cm,itemsep=5pt,parsep=0pt,topsep=0pt,rightmargin=0.5cm}

\setlist[itemize]{align=left,labelsep=1em,leftmargin=*,itemsep=0pt,parsep=0pt,topsep=0pt,rightmargin=0cm}

\renewcommand{\titreChapitre}{Ch4 : Équations du premier degré à une inconnue} 
\renewcommand{\numeroSerie}{8}

\begin{document}
\vspace*{-2\baselineskip}
\textLigne{Activités}
\begin{acti}
Parmi les égalités suivantes, lesquelles sont toujours vraies? lesquelles toujours fausses ? lesquelles parfois vraies parfois fausses ?
	\begin{tasks}(4)
\task $5+5=5^2$
\task $x+x=x^2$
\task $x+x=2 x$
\task $(x+1)^3=x^3+1^3$
\task $0 \cdot x=1$
\task $x^2 \cdot x^2 \cdot x^2=3 x^2$
\task $(x+1)^2=x^2+2 x+1$
\task $0 \cdot x=0$
	\end{tasks}
\end{acti}
\begin{acti}
	Répondre par vrai ou faux en justifiant.
\begin{tasks}
	\task  Le nombre $-8$ est-il solution de l'équation : $x^2=32-4 x$ ?
\task Le nombre $0$ est-il solution de l'équation : $x^2+12 x+12=3 x^3-3 x^2-x+12$ ?
\task Le nombre $-\frac{1}{2}$ est-il solution de l'équation : $x(x-2)=x^2-1$ ?
\task Le nombre $\frac{1}{2}$ est-il solution de l'équation : $x(x-2)=x^2-1$ ?
\end{tasks}
\end{acti}
\begin{acti}
On considère l'équation : $x^3-4=15 x$.
	\begin{tasks}
\task Un entier naturel est solution de cette équation; trouver lequel et justifier à l'aide de la définition du mot solution.
\task Montrer que le nombre irrationnel $\sqrt{3}-2$ est aussi solution de cette équation.
	\end{tasks}
\end{acti}
\begin{acti}
Observer les écritures suivantes pour trouver comment les réduire sans développer les carrés.
	\begin{tasks}(2)
\task $(2 x-y+1)^2-(2 x+y+1)^2$
\task $(2 x+y)^2+2(2 x+y)(2 x-y)+(2 x-y)^2$
\task $\left(\dfrac{1}{2} x-\dfrac{1}{2} y\right)^2-\left(\dfrac{1}{2} x+\dfrac{1}{2} y\right)^2$
\task $\left(x^2-2\right)^2-2\left(x^2-2\right)\left(x^2+x+1\right)+\left(x^2+x+1\right)^2$
	\end{tasks}
\end{acti}
\begin{acti}
A chaque étape, écrire explicitement la propriété ou le principe d'équivalence qui a été utilisé :
\end{acti}
\begin{acti}
Résoudre quatre fois de suite l'équation $\dfrac{x}{2}-3 x=\dfrac{5}{4}+x$, en utilisant la méthode proposée :
	\begin{tasks}
\task Votre manière de faire. 

(Dans les méthodes b), c) et d), simplifier au fur et à mesure l'expression obtenue.)
\task $\left[P E_2\right]$, en multipliant par 4 ; puis $\left[P E_1\right]$, en ajoutant $-4 x$; puis $\left[P E_2\right]$, en multipliant par $-\dfrac{1}{14}$.
\task $\left[P E_1\right]$, en ajoutant $-x$; puis $\left[P E_2\right]$, en multipliant par 2 ; puis $\left[P E_2\right]$, en multipliant par $-\dfrac{1}{7}$.
\task $\left[P E_1\right]$, en ajoutant $\dfrac{5}{2} x$; puis $\left[P E_1\right]$, en ajoutant $-\dfrac{5}{4} ;$ puis $\left[P E_2\right]$, en multipliant par $\dfrac{2}{7}$.
	\end{tasks}
\end{acti}
\begin{acti}
Déterminer le nombre $a$ pour que l'équation ait la solution demandée.
	\begin{tasks}(2)
\task $a x+1=2 x+5 \quad ;$ solution: $S=\{-2\}$;
\task $1-a x=4 x+2 \quad$; solution : $S=\left\{\dfrac{1}{3}\right\}$
\task $3=a \cdot\left(-\dfrac{1}{2} x+3\right)$; solution : $S=\{-1\}$
\task $7-2 x=x+a x$; solution: $S=\{3\}$.
	\end{tasks}
\end{acti}
\textLigne{Exercices}
\textLigne{Automatismes}

\begin{comment}
89. Résoudre les équations de l'exercice 80 .
\end{comment}

\end{document}

