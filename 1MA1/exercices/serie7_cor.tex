\documentclass[a4paper,12pt]{report}

\usepackage{../courslatex}
\usepackage{exos}
\usepackage{enumitem}

\setlist[enumerate]{align=left,leftmargin=1cm,itemsep=5pt,parsep=0pt,topsep=0pt,rightmargin=0.5cm}

\setlist[itemize]{align=left,labelsep=1em,leftmargin=*,itemsep=0pt,parsep=0pt,topsep=0pt,rightmargin=0cm}

\renewcommand{\titreChapitre}{Ch1 : Des Nombres} 
\renewcommand{\numeroSerie}{1}

\begin{document}
\vspace*{-2\baselineskip}
\textLigne{Exercices}
\textLigne{Automatismes}

\begin{core}\phantom{ }

\begin{tasks}(2)
\task ${{\left(6 r y-1\right)\left(6 r y+5 \right)}}$
\task ${{\left(2 x z-5 \right)\left(2 x z-1\right)}}$
\task ${{\left(5 t z-7 \right)\left(5 t z+2 \right)}}$
\task ${{\left(4 z-5 \right)\left(4 z-7 \right)}}$
\task ${{\left(8 y+1\right)\left(8 y+7 \right)}}$
\task ${{\left(6 t+7 \right)\left(6 t-5 \right)}}$
\end{tasks}

\end{core}

\begin{core}\phantom{ }

\begin{tasks}
\task $\begin{aligned}A&={{3 x}}\mathopen{}\left(16 s^2-16 s x + 3 x^2\right)\mathclose{}\\&={{3 x\mathopen{}\left(4 s-x\right)\mathclose{}\mathopen{}\left(4 s-3 x\right)\mathclose{}}}\end{aligned}$
\task $\begin{aligned}B&={{5 y}}\mathopen{}\left(49 s^2-112 s y + 64 y^2\right)\mathclose{}\\&={{5 y\mathopen{}\left(7 s-8 y\right)\mathclose{}^2}}\end{aligned}$
\task ${{-7 y(10 s + y + 7 )}}$
\task $\begin{aligned}D&={{-2 z}}\mathopen{}\left(81 y^2 + 81 y + 8 \right)\mathclose{}\\&={{-2 z\mathopen{}\left(9 y+8 \right)\mathclose{}\mathopen{}\left(9 y+1\right)\mathclose{}}}\end{aligned}$
\task ${{\left(9 t x-8 \right)\left(9 t x+2 \right)}}$
\task ${{\left(3 y+2 \right)^2}}$
\end{tasks}

\end{core}

\begin{core}\phantom{ }

\begin{tasks}
\task On procède de la manière suivante 

\medskip
 $\begin{aligned}A&=-4 t(-3 t + 8 )-t(-3 t + 8 )&&\text{réarranger les termes}\\&=(-4 t-t)(-3 t + 8 )&& \text{mettre en évidence le facteur } (-3 t + 8 )\\&={{-5 t(-3 t + 8 )}}&& \text{réduire et ordonner les facteurs}\\\end{aligned}$
\task On procède de la manière suivante 

\medskip
 $\begin{aligned}B&=7 y^2(y-8 )-y(y-8 )&&\text{réarranger les termes}\\&=(7 y^2-y)(y-8 )&& \text{mettre en évidence le facteur } (y-8 )\\&={{y(7 y-1)(y-8 )}}&& \text{factoriser le premier terme}\\\end{aligned}$
\task On procède de la manière suivante 

\medskip
 $\begin{aligned}C&=-4 x^2(-3 x + 7 )-2 (-3 x + 7 )&&\text{réarranger les termes}\\&=(-4 x^2-2 )(-3 x + 7 )&& \text{mettre en évidence le facteur } (-3 x + 7 )\\&={{2 (-2 x^2-1)(-3 x + 7 )}}&& \text{factoriser le premier terme}\\\end{aligned}$
\task On procède de la manière suivante 

\medskip
 $\begin{aligned}D&=(-8  + 3 z^2)(8 z-3 )&& \text{mettre en évidence le facteur } (8 z-3 )\\&={{(3 z^2-8 )(8 z-3 )}}&& \text{réduire et ordonner les facteurs}\\\end{aligned}$
\end{tasks}

\end{core}

\begin{core}\phantom{ }

\begin{tasks}
\task On procède de la manière suivante 

\medskip
 $\begin{aligned}A&=7 t(-9 t-8 )-6 t^2(9 t + 8 )+9 t^2(9 t + 8 )&&\text{réarranger les termes}\\&=-7 t(9 t + 8 )-6 t^2(9 t + 8 )+9 t^2(9 t + 8 )&&\text{multplier par } -1 \text{ certains groupements}\\&=(-7 t-6 t^2 + 9 t^2)(9 t + 8 )&& \text{mettre en évidence le facteur } (9 t + 8 )\\&=(3 t^2-7 t)(9 t + 8 ) && \text{réduire et ordonner les facteurs}\\&={{t(3 t-7 )(9 t + 8 )}}&& \text{factoriser le premier terme}\\\end{aligned}$
\task On procède de la manière suivante 

\medskip
 $\begin{aligned}B&=4 s(5 s + 8 )-4 (5 s + 8 )-3 s(5 s + 8 )&&\text{multplier par } -1 \text{ certains groupements}\\&=(4 s-4 -3 s)(5 s + 8 )&& \text{mettre en évidence le facteur } (5 s + 8 )\\&={{(s-4 )(5 s + 8 )}}&& \text{réduire et ordonner les facteurs}\\\end{aligned}$
\task On procède de la manière suivante 

\medskip
 $\begin{aligned}C&=7 (-6 y + 7 )+3 y^2(6 y-7 )-7 y^2(-6 y + 7 )&&\text{réarranger les termes}\\&=-7 (6 y-7 )+3 y^2(6 y-7 )+7 y^2(6 y-7 )&&\text{multplier par } -1 \text{ certains groupements}\\&=(-7  + 3 y^2 + 7 y^2)(6 y-7 )&& \text{mettre en évidence le facteur } (6 y-7 )\\&={{(10 y^2-7 )(6 y-7 )}}&& \text{réduire et ordonner les facteurs}\\\end{aligned}$
\task On procède de la manière suivante 

\medskip
 $\begin{aligned}D&=-2 z(-2 z-7 )+3 z(2 z + 7 )+7 z^2(2 z + 7 )&&\text{réarranger les termes}\\&=2 z(2 z + 7 )+3 z(2 z + 7 )+7 z^2(2 z + 7 )&&\text{multplier par } -1 \text{ certains groupements}\\&=(2 z + 3 z + 7 z^2)(2 z + 7 )&& \text{mettre en évidence le facteur } (2 z + 7 )\\&=(7 z^2 + 5 z)(2 z + 7 ) && \text{réduire et ordonner les facteurs}\\&={{z(7 z + 5 )(2 z + 7 )}}&& \text{factoriser le premier terme}\\\end{aligned}$
\end{tasks}

\end{core}

\begin{core}\phantom{ }

\begin{tasks}
\task On procède de la manière suivante 

\medskip
 $\begin{aligned}A&=-4 (r + 10 )+8 r^2(r + 10 )&&\text{mettre en évidence pour faire apparaître}\\
        & &&\text{le groupement }(r + 10 )\\&=(-4  + 8 r^2)(r + 10 )&& \text{mettre en évidence le facteur } (r + 10 )\\&=(8 r^2-4 )(r + 10 ) && \text{réduire et ordonner les facteurs}\\&={{4 (2 r^2-1)(r + 10 )}}&& \text{factoriser le premier terme}\\\end{aligned}$
\task On procède de la manière suivante 

\medskip
 $\begin{aligned}B&=s^2(-7 s + 5 )-9 s^2(-7 s + 5 )&&\text{mettre en évidence pour faire apparaître}\\
        & &&\text{le groupement }(-7 s + 5 )\\&=(s^2-9 s^2)(-7 s + 5 )&& \text{mettre en évidence le facteur } (-7 s + 5 )\\&={{-8 s^2(-7 s + 5 )}}&& \text{réduire et ordonner les facteurs}\\\end{aligned}$
\task On procède de la manière suivante 

\medskip
 $\begin{aligned}C&=10 t^2(7 t + 10 )+5 t(7 t + 10 )&&\text{mettre en évidence pour faire apparaître}\\
        & &&\text{le groupement }(7 t + 10 )\\&=(10 t^2 + 5 t)(7 t + 10 )&& \text{mettre en évidence le facteur } (7 t + 10 )\\&={{5 t(2 t + 1)(7 t + 10 )}}&& \text{factoriser le premier terme}\\\end{aligned}$
\task On procède de la manière suivante 

\medskip
 $\begin{aligned}D&=-10 t(6 t + 7 )+10 t^2(6 t + 7 )&&\text{mettre en évidence pour faire apparaître}\\
        & &&\text{le groupement }(6 t + 7 )\\&=(-10 t + 10 t^2)(6 t + 7 )&& \text{mettre en évidence le facteur } (6 t + 7 )\\&=(10 t^2-10 t)(6 t + 7 ) && \text{réduire et ordonner les facteurs}\\&={{10 t(t-1)(6 t + 7 )}}&& \text{factoriser le premier terme}\\\end{aligned}$
\end{tasks}

\end{core}

\end{document}

