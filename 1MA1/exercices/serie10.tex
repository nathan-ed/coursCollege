\documentclass[a4paper,12pt]{report}

\usepackage{../courslatex}
\usepackage{exos}
\usepackage{enumitem}

\setlist[enumerate]{align=left,leftmargin=1cm,itemsep=5pt,parsep=0pt,topsep=0pt,rightmargin=0.5cm}

\setlist[itemize]{align=left,labelsep=1em,leftmargin=*,itemsep=0pt,parsep=0pt,topsep=0pt,rightmargin=0cm}

\renewcommand{\titreChapitre}{Ch5~: Équations du second degré} 
\renewcommand{\numeroSerie}{10}

\begin{document}
\vspace*{-2\baselineskip}
\textLigne{Activités}
\begin{acti}
Factoriser le plus possible.
	\begin{tasks}(3)
\task $4 x^4-4$
\task $x^3-x^2-4(x-1)$
\task $16 x^4-9 y^2$
\task $3 x^2+6 x-24$
\task $8 x^3-8 x^2+2 x$
\task $(x+y)^2-4 u^2$
\task $x^3-5 x$
\task $x^4-64$
\task $4 y^2-12 y+9$
\task $a^2-a b-a+b$
\task $(4 x-1)^2-9(3-x)^2$
\task $4 a x^2 y^3-(a x y)^2+5 b x^3 y^2$
	\end{tasks}
\end{acti}
\begin{acti}
(Entraînement individuel.) Résoudre les équations dans $\mathbb{R}$.
	\begin{tasks}(4)
\task $2\left(\dfrac{x}{3}+3\right)=0$
\task $\dfrac{1-6 x}{4}=2\left(1-\dfrac{3}{4} x\right)$
\task $3 x=\dfrac{x-55}{6}$
\task $x+\dfrac{1}{4}=-\dfrac{3}{7}$
	\end{tasks}
\end{acti}
\begin{acti}
	Il s'agit de partager 2100 francs entre trois personnes de manière que la première ait le quart de la part de la troisième et 120 francs de plus que la deuxième.
	\begin{tasks}
		\task Voici trois façons de commencer. Compléter chacune de ces possibilités en fonction de $x$.

\begin{tabular}{|c|c|}
\hline part de la 1re personne~: & $x$ \\
\hline part de la 2e personne~: & \\ 
\hline part de la 3e personne~: &\\ 
\hline
\end{tabular}
\begin{tabular}{|c|c|}
\hline part de la lre personne~: & \\
\hline part de la 2e personne~: & $x$ \\ 
\hline part de la 3e personne~: &\\ 
\hline
\end{tabular}
\begin{tabular}{|c|c|}
\hline part de la lre personne~: &  \\
\hline part de la 2e personne~: & \\ 
\hline part de la 3e personne~: &$x$\\ 
\hline
\end{tabular}
\task Résoudre ce problème.
	\end{tasks}

\end{acti}
\begin{acti}
Trouver deux nombres entiers consécutifs tels que le quart du premier ajouté au cinquième du plus grand donne 29.
\end{acti}
\begin{acti}
 Résoudre les équations dans $\mathbb{R}$ (indication pour $i$ ) : utiliser la complétion du carré).
	\begin{tasks}(3)
\task $(2 x-3)^2=(7 x+3)^2$
\task $12 x-9 x^2=4$
\task $4 x(x+1)=-1$
\task $9 x^2-27=0$
\task $\dfrac{1}{\sqrt{2}}(5 x-7)=\sqrt{2} x+\sqrt{18}$
\task $x^2+4 x=32$
\task $4(x-7)=x^2(x-7)$
\task $x^3-2=x(2 x-1)$
\task $x^2+6 x+3=0$
	\end{tasks}
\end{acti}
\begin{acti}
Ayant reçu un héritage, je dépense 2000 francs pour acheter une moto et je place les deux tiers du reste à la banque. Il me reste alors $30 \%$ du montant total de l'héritage. Quel était ce montant?
\end{acti}
\begin{acti}
 Le rectangle représenté cí-dessous a été découpé en 5 carrés. Le périmètre du rectangle est de 1 m . Déterminer son aire.
\begin{center}
	\begin{tikzpicture}[scale=0.5]
    \tkzDefPoint[label=below:{}](0,0){A}
    \tkzDefPoint[label=right:{}](0,2.25){B}
    \tkzDefPoint[label=above:{}](0,4.5){C}
    \tkzDefPoint[label=above:{}](-2.25,4.5){D}
    \tkzDefPoint[label=above:{}](-5.25,4.5){E}
    \tkzDefPoint[label=above:{}](-5.25,1.5){F}
    \tkzDefPoint[label=above:{}](-5.25,0){G}
    \tkzDefPoint[label=above:{}](-3.75,0){H}
    \tkzDefPoint[label=above:{}](-2.25,0){I}
    \tkzDefPoint[label=above:{}](-2.25,2.25){J}
    \tkzDefPoint[label=above:{}](-2.25,1.5){K}
    \tkzDefPoint[label=above:{}](-3.75,1.5){L}
\tkzDrawPolygon(A,B,J,I)
\tkzDrawPolygon(B,C,D,J)
\tkzDrawPolygon(D,E,F,K)
\tkzDrawPolygon(F,G,H,L)
\tkzDrawPolygon(H,I,K,L)
 \end{tikzpicture}
\end{center}
\end{acti}
\begin{acti}
 Résoudre l'équation $x^2+2 x-1=0$ en appliquant pas à pas les instructions données :
	\begin{tasks}(3)
\task $\left[P E_1\right]:$ ajouter 2
\task factoriser le membre de gauche
\task $\left[P E_1\right]:$ ajouter $(-2)$
\task factoriser le membre de gauche
\task $\left[P E_3\right]$
	\end{tasks}
\end{acti}
\begin{acti}
Quand on demande à Mme Marguerite avec combien de chats elle vit, elle répond malicieusement : \enquote{Avec les 5 sixièmes de mes chats et 5 sixièmes de chat.} Combien Mme Marguerite a-t-elle de chats ?
\end{acti}
\textLigne{Exercices}
\textLigne{Automatismes}
\begin{comment}
	\begin{acti}
Résoudre les équations de l'exercice 81 . [( $\left.{ }^*\right)$ Pour la seconde, utiliser la complétion du carré.]
	\end{acti}
	\begin{acti}
(*) En utilisant des identités remarquables, écrire les nombres suivants sous la forme de deux nombres entiers supérieurs à 1 . (Indiquer les étapes de manière détaillée.)
		\begin{tasks}
\task $1.014 \cdot 049$
\task $899^{\circ} 999^{\circ} 879$
\task $256^{\circ} 027$
\task 159'991
\task $99^{\circ} 940^{\circ} 009$ (pensez à soustraire)
\task $81^{\circ} 053^{\circ} 993$
\task 10'807
\task 491'401
		\end{tasks}
	\end{acti}
\end{comment}
\end{document}

