\documentclass[a4paper,12pt]{report}

\usepackage{../courslatex}
\usepackage{exos}
\usepackage{enumitem}

\setlist[enumerate]{align=left,leftmargin=1cm,itemsep=5pt,parsep=0pt,topsep=0pt,rightmargin=0.5cm}

\setlist[itemize]{align=left,labelsep=1em,leftmargin=*,itemsep=0pt,parsep=0pt,topsep=0pt,rightmargin=0cm}

\renewcommand{\titreChapitre}{Ch5~: Équations du second degré} 
\renewcommand{\numeroSerie}{10}

\begin{document}
\vspace*{-2\baselineskip}
\textLigne{Activités}



\begin{acti}
Quand on demande à Mme Marguerite avec combien de chats elle vit, elle répond malicieusement : \enquote{Avec les 5 sixièmes de mes chats et 5 sixièmes de chat.} Combien Mme Marguerite a-t-elle de chats ?
\end{acti}
\textLigne{Exercices}
\textLigne{Automatismes}
\begin{comment}
	\begin{acti}
Résoudre les équations de l'exercice 81 . [( $\left.{ }^*\right)$ Pour la seconde, utiliser la complétion du carré.]
	\end{acti}
	\begin{acti}
(*) En utilisant des identités remarquables, écrire les nombres suivants sous la forme de deux nombres entiers supérieurs à 1 . (Indiquer les étapes de manière détaillée.)
		\begin{tasks}
\task $1.014 \cdot 049$
\task $899^{\circ} 999^{\circ} 879$
\task $256^{\circ} 027$
\task 159'991
\task $99^{\circ} 940^{\circ} 009$ (pensez à soustraire)
\task $81^{\circ} 053^{\circ} 993$
\task 10'807
\task 491'401
		\end{tasks}
	\end{acti}
\end{comment}
\end{document}

