\documentclass[a4paper,12pt]{report}

\usepackage{../courslatex}
\usepackage{exos}
\usepackage{enumitem}

\setlist[enumerate]{align=left,leftmargin=1cm,itemsep=5pt,parsep=0pt,topsep=0pt,rightmargin=0.5cm}

\setlist[itemize]{align=left,labelsep=1em,leftmargin=*,itemsep=0pt,parsep=0pt,topsep=0pt,rightmargin=0cm}

\renewcommand{\titreChapitre}{Ch3 : Calcul littéral} 
\renewcommand{\numeroSerie}{}
\renewcommand{\typeDoc}{Activité}

\begin{document}
\vspace*{-2\baselineskip}

\textLigne{Activité \enquote{somme-produit}}
\textLigne{Partie 1/4}
Voici l'énoncé du problème \enquote{somme-produit}:
\begin{quote}
\enquote{Déterminer deux nombres entiers $a$ et $b$ connaissant leur somme $s$ et leur produit $p$.} 
\end{quote}
Ce problème remonte aux babyloniens qui souhaitaient déterminer la longueur $\ell$ et la largeur $L$ d'un terrain rectangulaire (jardin, champ, etc.) connaissant son périmètre $2(\ell+L)$ et son aire $\ell\cdot L$. 

\begin{exo}
Déterminer deux nombres $a$ et $b$ dont~:
	\begin{tasks}(2)
\task le produit vaut $6$ et la somme $5$

\vspace{15pt}
$a=\ligne{2}$ $b=\ligne{2}$
\task le produit vaut $12$ et la somme $7$

\vspace{15pt}
$a=\ligne{2}$ $b=\ligne{2}$
\task le produit vaut $12$ et la somme $-7$

\vspace{15pt}
$a=\ligne{2}$ $b=\ligne{2}$
\task le produit vaut $-5$ et la somme $4$

\vspace{15pt}
$a=\ligne{2}$ $b=\ligne{2}$
\task le produit vaut $10$ et la somme $-7$

\vspace{15pt}
$a=\ligne{2}$ $b=\ligne{2}$
\task le produit vaut $-9$ et la somme $8$

\vspace{15pt}
$a=\ligne{2}$ $b=\ligne{2}$
\task le produit vaut $-8$ et la somme $-2$

\vspace{15pt}
$a=\ligne{2}$ $b=\ligne{2}$
\task le produit vaut $15$ et la somme $-8$

\vspace{15pt}
$a=\ligne{2}$ $b=\ligne{2}$
\end{tasks}
\end{exo}

\begin{exo}
	Écrire clairement une procédure pour obtenir une solution.  

	\vspace{15pt}
	\myrulefill
	\vspace{15pt}

	\myrulefill
	\vspace{15pt}

	\myrulefill
	\vspace{15pt}

	\myrulefill
	\vspace{15pt}
\end{exo}
Pour rappel, la quatrième identité remarquable est de la forme
\[(x+a)(x+b)=x^2+(a+b)x+ab.\]
On retrouve le terme somme ($a+b$) et le terme produit ($ab$).

\begin{exo}
	À l'aide du premier exercice, factoriser les expressions suivante en utilisant la quatrième identité remarquable.
\begin{tasks}(2)
\task $x^2+5x+6=\ligne{4}$ 
\task $x^2+7x+12=\ligne{4}$
\task $x^2-7x+12=\ligne{4}$
\task $x^2+4x+-5=\ligne{4}$
\task $x^2-7x+10=\ligne{4}$
\task $x^2+8x-9=\ligne{4}$
\task $x^2-2x-8=\ligne{4}$
\task $x^2-8x+15=\ligne{4}$
\end{tasks}	
\end{exo}

\afterpage{\bpage}
\newpage

\textLigne{Partie 2/4}
Un exemple de procédure 
\begin{exo}
	Utiliser votre méthode ou la méthode ci-dessus pour déterminer deux nombres $a$ et $b$ dont~:
	\begin{tasks}(2)
\task le produit vaut $-20$ et la somme $-8$

\vspace{15pt}
$a=\ligne{2}$ $b=\ligne{2}$
\task le produit vaut $-20$ et la somme $1$

\vspace{15pt}
$a=\ligne{2}$ $b=\ligne{2}$
\task le produit vaut $12$ et la somme $8$

\vspace{15pt}
$a=\ligne{2}$ $b=\ligne{2}$
\task le produit vaut $12$ et la somme $13$

\vspace{15pt}
$a=\ligne{2}$ $b=\ligne{2}$
\task le produit vaut $-40$ et la somme $3$

\vspace{15pt}
$a=\ligne{2}$ $b=\ligne{2}$
\task le produit vaut $28$ et la somme $-11$

\vspace{15pt}
$a=\ligne{2}$ $b=\ligne{2}$
	
\end{tasks}
\end{exo}
\begin{exo}
	Factoriser à l'aide de la quatrième identité remarquable.
	\begin{tasks}(2)
\task $x^2-8x-20=\ligne{4}$
\task $x^2+x-20=\ligne{4}$
\task $x^2-8x+12=\ligne{4}$
\task $x^2+13x+12=\ligne{4}$
\task $x^2+3x-40=\ligne{4}$
\task $x^2-11x+28=\ligne{4}$	
\end{tasks}
\end{exo}
\begin{exo}
Essayer de déterminer deux nombres $a$ et $b$ dont \[\text{le produit vaut } 233543149332 \text{ et la somme vaut } 1423373.\]
\end{exo}
Si vous n'y arrivez pas, quel est l'obstacle rencontré par rapport à votre méthode ou à la méthode proposée~?

	\vspace{15pt}
	\myrulefill
	\vspace{15pt}

	\myrulefill
	\vspace{15pt}

\afterpage{\bpage}
\newpage

\textLigne{Partie 3/4}

Les babyloniens ont trouvé une méthode pour résoudre ce problème, la voici.

Si on note $P$ le produit et $S$ la somme.
\begin{description}
	\item[Étape 1] On pose $r=\dfrac{S}{2}$.
	\item[Étape 2] Il existe $m$ tel que \[(r+m)(r-m)=P.\] On va déterminer la valeur de $m$ pour obtenir les nombres recherchés $a=r+m$ et $b=r-m$. 
	\item[Étape 3] On développe l'égalité ci-dessus
		\[(r+m)(r-m)=P \iff r^2-m^2=P \iff -m^2=P-r^2\iff m^2=r^2-P \iff m=\sqrt{r^2-P}\]
	\item[Étape 4] On connaît $r=\dfrac{S}{2}$ et $P$, ainsi
		\[m=\sqrt{\dfrac{S^2}{4}-P}\]
	\item[Étape 5] On obtient 
		\[a=\dfrac{S}{2}-\sqrt{\dfrac{S^2}{4}-P} \text{ et } b=\dfrac{S}{2}+\sqrt{\dfrac{S^2}{4}-P}\]
\end{description}
Déterminer $a$ et $b$ sachant que leur produit vaut $233543149332$ et leur somme vaut $1423373$ puis factoriser \[x^2+1423373x+233543149332.\]
\framebox[\textwidth]{\rule{0pt}{170pt}}

\begin{exo}
	Factoriser les expressions suivantes
	\begin{tasks}
		\task $x^2+1423373x+233543149332$

\framebox[0.9\textwidth]{\rule{0pt}{170pt}}
		\task $x^2+1423373x+233543149332$

\framebox[0.9\textwidth]{\rule{0pt}{170pt}}
	\end{tasks}
\end{exo}
Cette methode vous fait-elle penser à quelque chose que vous connaissez déjà~? Une autre procédure pour factoriser ce type d'expressions vous vient-elle à l'esprit~? 


	\vspace{15pt}
	\myrulefill
	\vspace{15pt}

	\myrulefill

	\vspace{15pt}
	\myrulefill
	\vspace{15pt}

	\myrulefill
	\vspace{15pt}

	\newpage

\textLigne{Partie 4/4}

\end{document}

