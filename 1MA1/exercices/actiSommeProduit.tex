\documentclass[a4paper,12pt]{report}

\usepackage{../courslatex}
\usepackage{exos}
\usepackage{enumitem}

\setlist[enumerate]{align=left,leftmargin=1cm,itemsep=5pt,parsep=0pt,topsep=0pt,rightmargin=0.5cm}

\setlist[itemize]{align=left,labelsep=1em,leftmargin=*,itemsep=0pt,parsep=0pt,topsep=0pt,rightmargin=0cm}

\renewcommand{\titreChapitre}{Ch3 : Calcul littéral} 
\renewcommand{\numeroSerie}{}
\renewcommand{\typeDoc}{Activité série 7}

\begin{document}
\vspace*{-2\baselineskip}

\textLigne{Activité \enquote{somme-produit}}
\textLigne{Partie 1/4}
Voici l'énoncé du problème \enquote{somme-produit}:
\begin{quote}
\enquote{Déterminer deux nombres entiers $a$ et $b$ connaissant leur somme $S$ et leur produit $P$.} 
\end{quote}
Ce problème remonte aux babyloniens qui souhaitaient déterminer la longueur $\ell$ et la largeur $L$ d'un terrain rectangulaire (jardin, champ, etc.) connaissant son périmètre $2(\ell+L)$ et son aire $\ell\cdot L$. 

\begin{exo}
Déterminer deux nombres $u$ et $v$ dont~:
	\begin{tasks}(2)
\task le produit vaut $6$ et la somme $5$

\vspace{15pt}
$u=\ligne{2}$ $v=\ligne{2}$
\task le produit vaut $12$ et la somme $7$

\vspace{15pt}
$u=\ligne{2}$ $v=\ligne{2}$
\task le produit vaut $12$ et la somme $-7$

\vspace{15pt}
$u=\ligne{2}$ $v=\ligne{2}$
\task le produit vaut $-5$ et la somme $4$

\vspace{15pt}
$u=\ligne{2}$ $v=\ligne{2}$
\task le produit vaut $10$ et la somme $-7$

\vspace{15pt}
$u=\ligne{2}$ $v=\ligne{2}$
\task le produit vaut $-9$ et la somme $8$

\vspace{15pt}
$u=\ligne{2}$ $v=\ligne{2}$
\task le produit vaut $-8$ et la somme $-2$

\vspace{15pt}
$u=\ligne{2}$ $v=\ligne{2}$
\task le produit vaut $15$ et la somme $-8$

\vspace{15pt}
$u=\ligne{2}$ $v=\ligne{2}$
\end{tasks}
\end{exo}

\begin{exo}
	Écrire clairement une procédure pour obtenir une solution.  

	\vspace{15pt}
	\myrulefill
	\vspace{15pt}

	\myrulefill
	\vspace{15pt}

	\myrulefill
	\vspace{15pt}

	\myrulefill
	\vspace{15pt}
\end{exo}
Pour rappel, la quatrième identité remarquable est de la forme
\[(x+u)(x+v)=x^2+(u+v)x+uv.\]
On retrouve le terme somme ($u+v$) et le terme produit $uv$.

\begin{exo}
	À l'aide du premier exercice, factoriser les expressions suivantes en utilisant la quatrième identité remarquable.
\begin{tasks}(2)
\task $x^2+5x+6=\ligne{4}$ 
\task $x^2+7x+12=\ligne{4}$
\task $x^2-7x+12=\ligne{4}$
\task $x^2+4x+-5=\ligne{4}$
\task $x^2-7x+10=\ligne{4}$
\task $x^2+8x-9=\ligne{4}$
\task $x^2-2x-8=\ligne{4}$
\task $x^2-8x+15=\ligne{4}$
\end{tasks}	
\end{exo}

\afterpage{\bpage}
\newpage

\textLigne{Partie 2/4}
Un exemple de procédure. On note $P$ le produit et $S$ la somme. 
\tikzstyle{startstop} = [rectangle, rounded corners, minimum width=3cm, minimum height=1cm,text centered, draw=black,
execute at begin node={\begin{varwidth}{15em}},
execute at end node={\end{varwidth}}]
\tikzstyle{io} = [trapezium, trapezium left angle=70, trapezium right angle=110, minimum width=3cm, minimum height=1cm, text centered, draw=black, fill=blue!30]
\tikzstyle{process} = [rectangle, minimum width=3cm, minimum height=1cm, text centered, draw=black, fill=orange!30]
\tikzstyle{decision} = [diamond, minimum width=3cm, minimum height=1cm, text centered, draw=black, fill=green!30]
\tikzstyle{arrow} = [thick,->,>=stealth]
\begin{center}
\begin{tikzpicture}[scale=0.8,node distance=1.5cm]
	\node (e1) [startstop] {\centerline{Déterminer les diviseurs de $P$}};
	\node (e21) [startstop, below of=e1,  xshift=-6cm, yshift=-1cm] {\centerline{$P$ est positif~?}};
\node (e23) [startstop, below of=e1,  xshift=6cm, yshift=-1cm] {\centerline{$P$ négatif~?}};
\node (e34) [startstop, below of=e23, yshift=-1cm, xshift=0.5cm] {\centerline{$S$ positif}\centerline{$|u|\geq |v|$}\centerline{$u\geq0$ et $v\leq0$}};
\node (e31) [startstop, below of=e21, yshift=-1cm,xshift=-0.5cm] {\centerline{$S$ négatif}\centerline{$u\leq0$ et $v\leq0$}};
\node (e32) [startstop, below of=e21, xshift=3.5cm, yshift=-1cm] {\centerline{$S$ positif}\centerline{$u\geq0$ et $v\geq0$}};
\node (e33) [startstop, below of=e23, xshift=-3.5cm, yshift=-1cm] {\centerline{$S$ négatif}\centerline{$|u|\geq |v|$}\centerline{$u\leq0$ et $v\geq0$}};
\draw[arrow] (e1)--(e21);
\draw[arrow] (e1)--(e23);
\draw[arrow] (e21)--(e31);
\draw[arrow] (e21)--(e32);
\draw[arrow] (e23)--(e34);
\draw[arrow] (e23)--(e33);
\end{tikzpicture}
\end{center}
\begin{exo}
	Utiliser votre méthode ou la méthode ci-dessus pour déterminer deux nombres $u$ et $v$ dont~:
	\begin{tasks}(2)
\task le produit vaut $-20$ et la somme $-8$

\vspace{15pt}
$u=\ligne{2}$ $v=\ligne{2}$
\task le produit vaut $-20$ et la somme $1$

\vspace{15pt}
$u=\ligne{2}$ $v=\ligne{2}$
\task le produit vaut $12$ et la somme $8$

\vspace{15pt}
$u=\ligne{2}$ $v=\ligne{2}$
\task le produit vaut $12$ et la somme $13$

\vspace{15pt}
$u=\ligne{2}$ $v=\ligne{2}$
\task le produit vaut $-40$ et la somme $3$

\vspace{15pt}
$u=\ligne{2}$ $v=\ligne{2}$
\task le produit vaut $28$ et la somme $-11$

\vspace{15pt}
$u=\ligne{2}$ $v=\ligne{2}$
	
\end{tasks}
\end{exo}
\begin{exo}
	Factoriser à l'aide de la quatrième identité remarquable.
	\begin{tasks}(2)
\task $x^2-8x-20=\ligne{4}$
\task $x^2+x-20=\ligne{4}$
\task $x^2-8x+12=\ligne{4}$
\task $x^2+13x+12=\ligne{4}$
\task $x^2+3x-40=\ligne{4}$
\task $x^2-11x+28=\ligne{4}$	
\end{tasks}
\end{exo}
\begin{exo}
Essayer de déterminer deux nombres $a$ et $b$ dont \[\text{le produit vaut } 233543149332 \text{ et la somme vaut } 1423373.\]
\end{exo}
Si vous n'y arrivez pas, quel est l'obstacle rencontré par rapport à votre méthode ou à la méthode proposée~?

	\vspace{15pt}
	\myrulefill
	\vspace{15pt}

	\myrulefill
	\vspace{15pt}

	\myrulefill
	\vspace{15pt}


	\myrulefill
	\vspace{15pt}


\afterpage{\bpage}
\newpage

\textLigne{Partie 3/4}
\begin{center}
Dès cette partie, l'usage de la calculatrice est recommandé.
\end{center}

Les babyloniens ont trouvé une méthode pour résoudre ce problème, la voici.

Si on note $P$ le produit et $S$ la somme.
\begin{description}
	\item[Étape 1] On pose $r=\dfrac{S}{2}$.
	\item[Étape 2] Il existe $m$ tel que \[(r+m)(r-m)=P \text{ et par définition } (r+m)+(r-m)=2r=S.\] On va déterminer la valeur de $m$ pour obtenir les nombres recherchés $u=r+m$ et $v=r-m$. 
	\item[Étape 3] On isole $m$ dans l'égalité ci-dessus.
		\[(r+m)(r-m)=P \iff r^2-m^2=P \iff -m^2=P-r^2\iff m^2=r^2-P \iff m=\sqrt{r^2-P}\]
	\item[Étape 4] On connaît $r=\dfrac{S}{2}$ et $P$, ainsi
		\[m=\sqrt{\dfrac{S^2}{4}-P}\]
	\item[Étape 5] On obtient 
		\[u=\dfrac{S}{2}-\sqrt{\dfrac{S^2}{4}-P} \text{ et } v=\dfrac{S}{2}+\sqrt{\dfrac{S^2}{4}-P}\]
\end{description}
Déterminer $u$ et $v$ sachant que leur produit vaut $233543149332$ et leur somme vaut $1423373$ puis factoriser \[x^2+1423373x+233543149332.\]
\begin{alignat*}{3}
	S&=\ligne{3}\quad  P&&=\ligne{3} &&\\ 
	\dfrac{S^2}{4}&=\ligne{3}\quad \dfrac{S}{2}&&=\ligne{3}\quad  \sqrt{\dfrac{S^2}{4}-P}&&=\ligne{3}\quad 
\end{alignat*}

	\[v=\ligne{3} \quad \quad u=\ligne{3}\]


	\[x^2+1423373x+233543149332 = (x+\ligne{3})(x+\ligne{3})\]
\begin{exo}
	Factoriser les expressions suivantes
	\begin{tasks}
		\task $x^2+4533498x+4622763439976$

\framebox[0.9\textwidth]{\rule{0pt}{170pt}}
		\task $x^2+4405091x+124184968158$

\framebox[0.9\textwidth]{\rule{0pt}{170pt}}
	\end{tasks}
\end{exo}
Cette methode vous fait-elle penser à quelque chose que vous connaissez déjà~? Une autre procédure pour factoriser ce type d'expressions vous vient-elle à l'esprit~? 


	\vspace{15pt}
	\myrulefill
	\vspace{15pt}

	\myrulefill

	\vspace{15pt}
	\myrulefill
	\vspace{15pt}

	\myrulefill
	\vspace{15pt}

	\newpage

\textLigne{Partie 4/4}

On peut résoudre l'équation 
\begin{equation}
	x^2+Sx+P=0 \label{eq:1}
\end{equation}
en utilisant la formule quadratique. 

Pour rappel~: si $ax^2+bx+c=0$ admet une ou deux solutions réelles, alors
$\Delta=b^2-4ac\geq 0$. De plus, 
\[ax^2+bx+c=(x-s_1)(x-s_2),\]
avec $s_1=\dfrac{-b+\sqrt{\Delta}}{2a}$ et $s_2=\dfrac{-b-\sqrt{\Delta}}{2a}$. 

Dans le cas de l'équation \eqref{eq:1}, $a=\ligne{2}, b=\ligne{2}$ et $c=\ligne{3}$. 

\begin{exo}
	On reprend les expressions de l'exercice 7.
	\begin{tasks}	
	\task  Les factoriser en résolvant l'équation du second degré.

$x^2+4533498x+4622763439976=0$

\framebox[0.9\textwidth]{\rule{0pt}{160pt}}

$x^2+4405091x+124184968158=0$

\framebox[0.9\textwidth]{\rule{0pt}{160pt}}
	\task Que remarques-tu~? Pourquoi~?

	\myrulefill
	\vspace{15pt}

	\myrulefill
	\vspace{15pt}

	\myrulefill


	\end{tasks}
\end{exo}

À l'issue de cette activité, je suis capable de~:
\begin{itemize}
	\item Déterminer deux nombres connaissant leur somme et leur produit;
	\item Expliciter le lien entre le problème somme-produit avec la factorisation de la quatrième identité;
	\item  Appliquer une méthode générale pour résoudre ce problème datant des babyloniens;
	\item  Utiliser le lien entre le problème somme-produit et les équations du second degré pour ramener le problème somme-produit à la résolution d'une équation du second degré.
\end{itemize}
\end{document}

