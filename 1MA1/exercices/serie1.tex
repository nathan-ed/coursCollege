\documentclass[a4paper,12pt]{report}
\usepackage{../courslatex}
\usepackage{exos}
\renewcommand{\numeroSerie}{1}



\renewcommand{\titreChapitre}{Ch1 : Des Nombres} 
\begin{document}
\vspace*{-2\baselineskip}
\textLigne{Activités}

\begin{acti}
Quel est le chiffre des unités du nombre $8^{2024}$~? %(*) ...Et celui des dizaines~?!
\end{acti}

\begin{acti}
Donner l'ensemble des diviseurs pour chacun des entiers allant de 1 à 10, sous la forme habituelle~: 

\[\operatorname{Div}_1=\{1\} ;\quad \operatorname{Div}_2=\{1 ; 2\} ;\quad \operatorname{Div}_3=\{1 ; 3\} ;\quad \ldots ;\quad \operatorname{Div}_{10}=\ldots.\]
\begin{enumerate}
\item Relever la liste des entiers de 1 à 10 qui ont un nombre impair de diviseurs :
\begin{enumerate}
\item Pouvez-vous trouver un point commun à ces entiers, ou leur nom~?
\item Donner la liste des quinze premiers nombres entiers qui ont cette caractéristique.
\end{enumerate}
\item Relever la liste des entiers de 1 à 10 qui ont exactement deux diviseurs :
\begin{enumerate}
\item Pouvez-vous trouver un point commun à ces entiers, ou leur nom~?
\item Donner la liste des nombres entiers inférieurs à 50 qui ont cette caractéristique.
\end{enumerate}
\end{enumerate}
\end{acti}

%\begin{acti}[*]
%En plaçant dans les cases des nombres premiers distincts et inférieurs à 20, quel est le plus grand résultat entier possible pour le nombre $\frac{\square+\square+\square+\square+\square+\square+\square}{\square}$ ?
%\end{acti}


\textLigne{Exercices}
\begin{exo}
Sur les multiples de $3$~:
\begin{enumerate}
\item Trouver le plus grand multiple de $3$, formé de cinq chiffres et terminant par $24$.
\item Trouver le plus petit multiple de $3$, formé de quatre chiffres et terminant par $24$.
\item Trouver le plus petit multiple de $3$, formé de quatre chiffres pairs distincts.
\item Trouver le plus grand multiple de $3$, formé de quatre chiffres impairs distincts.
\end{enumerate}
\end{exo}


\begin{exo}
Leonhard EULER énonça en 1772 :
\enquote{Le nombre $n^2+n+41$ est premier pour $n \leq 39.$} $(n \in \mathbb{N})$

Vérifier son affirmation pour $0 \leq n \leq 6$ en contrôlant dans la liste de la page 2 du cours.
%\begin{enumerate}
%\item 
%\item $\left({ }^*\right)$ Montrer que $n^2+n+41$ n'est premier ni pour 41 ni pour 40, sans calculer la valeur du nombre pour 40 ni 41 et sans la liste, mais uniquement par factorisation.
%\end{enumerate}
\end{exo}

\begin{exo}
Décomposer les nombres suivants en produit de facteurs premiers (sans calculatrice)~:
\begin{tasks}(5)
\task 10
\task $10^2$
\task 100000
\task $24 \cdot 1000$
\task $38 \cdot 10^5$
\task 25000
\task 28000
\task 66000
\task 16000
\task 3600000
\end{tasks}
\end{exo}
\begin{exo}
Calculer de tête, le plus simplement et rapidement possible (environ 3 minutes pour l'ensemble de l'exercice)~:
\begin{tasks}(4)
\task $2^3 \cdot 3 \cdot 5^2$
\task $2^3 \cdot 7 \cdot 5^3$
\task $2^4 \cdot 5^2$
\task $2^3 \cdot 5^4$
\task $2^5 \cdot 5^5 \cdot 7$
\task $2 \cdot 3 \cdot 5 \cdot 7$
\task $2^4 \cdot 5^4 \cdot 11$
\task $2^6 \cdot 5^3$
\task $2^4 \cdot 5^6$
\task $2^3 \cdot 5^3 \cdot 7^2$
\task $2^4 \cdot 3 \cdot 5^2$
\task $2^6 \cdot 3 \cdot 5^8$
\end{tasks}
\end{exo}


%\begin{acti}[*] Déterminer l'écriture scientifique exacte du nombre $64^{21} \cdot 25^{64}$.	
%\end{acti}

\textLigne{Automatismes}
%https://coopmaths.fr/alea/?uuid=62f66&id=10NO6-2&alea=dg3hm&uuid=6fda8&alea=dg3hm&uuid=b51ec&alea=dg3hm&uuid=cb572&alea=dg3hm&uuid=6575c&n=8&d=10&s=5&alea=dg3hm&cd=0&uuid=29919&n=5&d=10&s=2&s2=true&s3=true&s4=3&alea=dg3hm&cd=1&lang=fr-CH&v=latex

\begin{auto}Calculer.

\begin{tasks}(3)
	\task $-4\cdot(-5)\cdot3+3$
	\task $-5\cdot(-29+33)\cdot5$
	\task $-3\cdot(10-4)$
	\task $-1-(4+9)$
	\task $8\div(3-2)$
	\task $-11-(+8)-4$
\end{tasks}

\end{auto}

% @see : https://coopmaths.fr/alea?uuid=6fda8&id=2N31-4.js&n=5&d=10&s=false&alea=dg3hm&cd=1&cols=1
\begin{auto}Calculer.

\begin{tasks}(3)
	\task $(-2 -2 + (-3)^2 ) \cdot 5$
	\task $4 + 3^2 \cdot (-2)$
	\task $-4 \cdot ( (-1)^2+2\cdot(-1))$
	\task $(-4)^2 +  7 \cdot 6$
	\task $(-2)^2 \cdot ( -2-7)$
\end{tasks}
\end{auto}

% @see : https://coopmaths.fr/alea?uuid=b51ec&id=2N30-2.js&n=5&d=10&s=2&s2=false&s3=true&s4=false&alea=dg3hm&cd=1&cols=1
\begin{auto}Calculer et donner le résultat sous la forme d'une fraction simplifiée au maximum.

\begin{tasks}(5)
	\task $7+\dfrac{3}{8}$
	\task $\dfrac{8}{24}+\dfrac{3}{6}$
	\task $\dfrac{5}{8}-\dfrac{2}{12}$
	\task $\dfrac{8}{7}-\dfrac{6}{8}$
	\task $\dfrac{7}{50}+\dfrac{9}{75}$
\end{tasks}
\end{auto}

% @see : https://coopmaths.fr/alea?uuid=cb572&id=2N30-4.js&n=5&d=10&s=1&alea=dg3hm&cd=1&cols=1
\begin{auto}Calculer et donner le résultat sous forme irréductible.

\begin{tasks}(5)
	\task $\dfrac{5}{8}\div\dfrac{4}{7}$
	\task $\dfrac{3}{4}\div\dfrac{2}{3}$
	\task $\dfrac{2}{5}\div\dfrac{1}{4}$
	\task $\dfrac{1}{7}\div\dfrac{2}{3}$
	\task $\dfrac{1}{8}\div\dfrac{1}{10}$
\end{tasks}
\end{auto}

% @see : https://coopmaths.fr/alea?uuid=6575c&id=2N30-5.js&n=8&d=10&s=5&alea=dg3hm&cd=0&cols=4
\begin{auto}Effectuer les calculs suivants.

\begin{tasks}(4)
	\task $\dfrac{7}{32}-\dfrac{6}{8}$
	\task $\dfrac{6}{3} \cdot \dfrac{5}{4}$
	\task $2 + \dfrac{7}{5} $ 
	\task $\dfrac{9}{3} \cdot \dfrac{5}{3} + \dfrac{5}{3}$
	\task $\dfrac{4}{6} \cdot \dfrac{7}{3}$
	\task $\dfrac{3}{4}-4$
	\task $\dfrac{6}{2} + \dfrac{4}{2} \cdot \dfrac{3}{7}$ 
	\task $\dfrac{6}{3}+\dfrac{4}{3}$ 
\end{tasks}

\end{auto}

% @see : https://coopmaths.fr/alea?uuid=29919&id=2N30-3.js&n=5&d=10&s=2&s2=true&s3=true&s4=3&alea=dg3hm&cd=1&cols=1
\begin{auto}Calculer et donner le résultat sous forme irréductible.

\begin{tasks}(5)
	\task $\dfrac{\phantom{\dfrac{(_(^(}{(_(^(}}-\dfrac{22}{21}\phantom{\dfrac{(_(^(}{(_(^(}}}{\phantom{\dfrac{(_(^(}{(_(^(}}\dfrac{88}{-7}\phantom{\dfrac{(_(^(}{(_(^(}}}$
	\task $\dfrac{28}{10}\cdot\dfrac{8}{35}$
	\task $\dfrac{-33}{24}\cdot\dfrac{21}{-99}$
	\task $\dfrac{\phantom{\dfrac{(_(^(}{(_(^(}}\dfrac{5}{15}\phantom{\dfrac{(_(^(}{(_(^(}}}{\phantom{\dfrac{(_(^(}{(_(^(}}\dfrac{15}{3}\phantom{\dfrac{(_(^(}{(_(^(}}}$
	\task $\dfrac{\phantom{\dfrac{(_(^(}{(_(^(}}\dfrac{15}{63}\phantom{\dfrac{(_(^(}{(_(^(}}}{\phantom{\dfrac{(_(^(}{(_(^(}}\dfrac{-21}{-35}\phantom{\dfrac{(_(^(}{(_(^(}}}$
\end{tasks}
\end{auto}
\begin{comment}
\begin{auto}
	Calculer et donner la réponse sous la forme d'un entier ou d'une fraction irréductible.
	\begin{tasks}(2)
\task $\dfrac{5}{6}:\left(\dfrac{4}{3}+\dfrac{3}{4}\right)$
\task $\dfrac{7}{9}-\left(-\dfrac{2}{5}\right) \cdot \dfrac{5}{9}$
\task $\quad\left(\dfrac{2}{3} \cdot\left(-\dfrac{1}{2}\right)\right):\left(\dfrac{2}{3}+\left(-\dfrac{1}{2}\right)\right)$
\task $\left(\dfrac{7}{9}-\left(-\dfrac{2}{5}\right)\right):\left(-\dfrac{5}{9}\right)$
\task $\left(\dfrac{2}{5}: 3\right):\left(\dfrac{2}{5}+3\right)$
\task $\dfrac{75}{42}: \dfrac{55}{154}$
	\end{tasks}
\end{auto}

\newpage
\begin{auto}
	Calculer et donner la réponse sous la forme d'un entier ou d'une fraction irréductible.
	\begin{tasks}(2)
%\task $\dfrac{121}{77} \cdot \dfrac{69}{92}$
\task $0,5 \cdot \dfrac{4}{5} \cdot(-3)$
\task $\dfrac{1}{2} \cdot \dfrac{1}{3}-\dfrac{5}{6}$
\task $-\left(-\dfrac{2}{3}\right)+\left(-\dfrac{7}{6}\right)-$ $\left(-\dfrac{1}{12}\right)-(+2)$
\task $\dfrac{3}{4} \cdot \dfrac{8}{9}-3 \cdot \dfrac{7}{18}$
\task $-\dfrac{77}{11}-\left(-\dfrac{32}{8}\right)+\left(-\dfrac{49}{7}\right)$
\task $\dfrac{\left(-\dfrac{4}{3}\right)-\left(-\dfrac{6}{5}\right)+\left(-\dfrac{3}{2}\right)}{\left(-\dfrac{3}{4}\right)-\left(-\dfrac{5}{6}\right)-\left(-\dfrac{2}{3}\right)}$
%\task $\dfrac{\dfrac{2}{9} \cdot\left(3-\dfrac{7}{2}\right)}{\left(\dfrac{1}{3}\right)^2-\left(\dfrac{2}{3}\right)^3}$
\task $\dfrac{\dfrac{5}{12}-\dfrac{4}{13}}{\dfrac{3}{13}+\dfrac{1}{12}}$
%\task $\dfrac{\dfrac{2}{5}-\left(-\dfrac{4}{3}\right)}{\left(-\dfrac{12}{5}\right) \cdot\left(-\dfrac{1}{3}\right)^3}$
%\task $\dfrac{\left(-\dfrac{3}{2}\right):\left(\left(\dfrac{1}{3}\right)-\left(\dfrac{1}{2}\right)\right)}{\left(\dfrac{1}{3}+\dfrac{1}{2}\right):\left(-\dfrac{2}{3}\right)}$
\task $\dfrac{\left(-\dfrac{1}{7}\right)^2 \cdot\left(\dfrac{7}{2}\right)^2 \cdot(-1)^3}{6-\left(\dfrac{5}{2}\right)^2}$
	\end{tasks}
\end{auto}
\begin{auto}
	Calculer et réduire le plus possible.
\begin{tasks}(2)
	\task $\dfrac{4 \cdot 10^9}{10^{-3} \cdot 16}$
	\task $\dfrac{6 \cdot 10^2 \cdot 15 \cdot 10^9}{30 \cdot 10^4}$
	\task $\dfrac{5\cdot 10^4\cdot 6\cdot 10^{-3}\cdot 1040}{4\cdot 300\cdot 10^{-22}\cdot 10^{15}}$
	\task $\dfrac{32\cdot 10^4\cdot 10^{-5}\cdot 12 \cdot 10^{3}}{10^{-8}\cdot 8\cdot 16 \cdot 10^{-6}}$
%\task $\dfrac{3 \cdot 10^{14} \cdot 12}{3 \cdot 10^{11}}$
\task $\dfrac{3,2 \cdot 10^{-3} \cdot 5 \cdot\left(10^2\right)^3}{4 \cdot 10^2}$ 
\task $\dfrac{1,5 \cdot 10^7 \cdot 4 \cdot 10^{-5}}{25 \cdot 10^8}$
%\task $2,5 \cdot 10^{-2}+7,5 \cdot 10^{-3}$
\task $\dfrac{3 \cdot 10^3 \cdot 5 \cdot 10^2}{12 \cdot\left(10^3\right)^4}$
%\task $3 \cdot 10^5-3 \cdot 10^4+3 \cdot 10^2$
%\task $152 \cdot 10^{-4}+32 \cdot 10^{-3}-16 \cdot 10^{-5}$
\task $\dfrac{24 \cdot 10^{-2} \cdot 3,5 \cdot 10^5}{6 \cdot 10^{-3} \cdot 56 \cdot 10^4}$
%\task $\dfrac{5132 \cdot 10^{-5}+0,000055}{10^6}$
%\task $-4^2+10^3 \cdot 10^{-1}+(-3)^2$
\end{tasks}
\end{auto}

\begin{exo} On donne une suite régulière de nombres~:
\[1^2-0^2 ;\quad 2^2-1^2 ;\quad 3^2-2^2 ;\quad 4^2-3^2 ;\quad \ldots\]
\begin{enumerate}
\item Calculer les quatre premiers termes donnés ci-dessus.
%\item (*) Calculer le $2017^{\text {ème}}$ à la main, et donner la valeur réduite du $n$-ième terme de la suite.
\end{enumerate}
\end{exo}
\begin{exo}[*]
Tout nombre s'écrivant sous la forme "$a b c a b c$", c'est-à-dire un nombre de six chiffres dont les trois premiers sont identiques aux trois derniers (par exemple : $817817$), est un multiple de $7$, de $11$ et de $13$. Pourquoi~?
\end{exo}
\begin{exo}[*]
On donne l'identité : $(1099511627776) \cdot(9094947017729282379150390625)=10^{40}$.
\begin{enumerate}
\item Montrer, à l'aide de la notation scientifique, que l'ordre du grandeur du produit est correct.
\item Déduire de cette identité la décomposition en produits de facteurs premiers des entiers 1099511627776 et 9094947017729282379150390625.
\end{enumerate}
\end{exo}
\end{comment}
\end{document}
