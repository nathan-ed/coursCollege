\documentclass[a4paper,12pt]{report}

\usepackage{../courslatex}
\usepackage{exos}
\usepackage{enumitem}

\setlist[enumerate]{align=left,leftmargin=1cm,itemsep=5pt,parsep=0pt,topsep=0pt,rightmargin=0.5cm}

\setlist[itemize]{align=left,labelsep=1em,leftmargin=*,itemsep=0pt,parsep=0pt,topsep=0pt,rightmargin=0cm}

\renewcommand{\titreChapitre}{Ch3\,: Calcul littéral} 
\renewcommand{\numeroSerie}{5}

\begin{document}
\vspace*{-2\baselineskip}
\textLigne{Activités}
\begin{acti}
Connaître par coeur et savoir démontrer les identités suivantes :
\begin{tasks}(2)
\task $(a+b)^2=a^2+2 a b+b^2$
\task $(a-b)^2-a^2-2 a b+b^2$
\task $(a+b)(a-b)=a^2-b^2$
\task $(x+a)(x+b)=x^2+(a+b) x+a b$
\end{tasks}
\end{acti}
\begin{acti}	
On considère l'identité suivante, appelée égalité de Lagrange (mathématicien du XVI ${ }^e$ siècle):
\[
\left(a^2+b^2\right)\left(c^2+d^2\right)=(a c+b d)^2+(a d-b c)^2
\]
\begin{tasks}
\task Démontrer cette identité.
\task Appliquer cette identité à quatre entiers (par exemple $2,3,4,5$ ) en utilisant la calculatrice.
%\task Décrire cette identité par une phrase: \enquote{Le produit d...}
\end{tasks}
\end{acti}

\begin{acti}
On considère le nombre $123456789^2-123456786 \cdot 123456792$.
\begin{tasks}(2)
\task Calculer ce nombre à l'aide d'une calculatrice.
\task Poser $x=123456789$ et exprimer le nombre considéré en fonction de $x$.
\task Développer et réduire l'expression trouvée en b).
\task Que conclure des calculs précédents ?
\end{tasks}	
\end{acti}
\begin{acti}
Développer le carré: $(3+2 \sqrt{2})^2$. En déduire une autre écriture pour $\sqrt{17+12 \sqrt{2}}$.
\end{acti}
\textLigne{Exercices}
\begin{exo}
\enquote{Donner une preuve géométrique} pour les identités remarquables 2), 3) et 4) en suivant l'exemple de celle donnée dans la théorie pour la première identité.
\end{exo}
\begin{exo}
Développer à l'aide d'une identité remarquable, directement et rapidement 

(sans copier l'énoncé, ne pas s'accorder plus de 5').
	\begin{tasks}(4)
\task $(x+y)^2$
\task $(2x^2+2)(2x^2-6)$
\task $(x-y)(x+y)$
\task $(3 x+y)^2$
\task $\left(x^2+y^3\right)^2$
\task $(x-1)^2$
\task $(1-x)(1+x)$
\task $(4 x-3)^2$
\task $\left(x^3+3 y\right)\left(x^3-3 y\right)$
\task $(3 z-2)^2$
\task $(1-x)^2$
\task $(x y+2 y)^2$
\task $\left(x^2-1\right)^2$
\task $(2 x+2)^2$
\task $(2 a+3)(2 a+3)$
\task $(x y z+5)(x y z-5)$
\task $\left(3 x^3-5\right)^2$
\task $(a+3 b)(a+3 b)$
\task $\left(x^2-1\right)\left(x^2-1\right)$
\task $\left(4 a^2 b-5\right)\left(4 a^2 b+5\right)$
\task $\left(2 x y^3-1\right)\left(2 x y^3-1\right)$
\task $\left(x^4+y\right)\left(x^4+y\right)$
\task $\left(1-a x^4\right)\left(1+a x^4\right)$
\task $\left(x^2+a^2\right)\left(x^2-a^2\right)$
	\end{tasks}
\end{exo}
\begin{exo}
Développer directement à l'aide des identités remarquables sans écrire l'étape intermédiaire.

Exemple~: $(x-3)(x+2)=x^2-x-6$.
	\begin{tasks}(4)
\task $(x-1)(x-2)$
\task $(x+3)(x+1)$
\task $(x-4)(x+4)$
\task $(y+6)(y-8)$
\task $(a+1)(a-12)$
\task $(y+9)(y-4)$
\task $(a+7)(a+3)$
\task $(x-3)(x-3)$
	\end{tasks}
\end{exo}
\begin{exo}
	Utiliser les identités remarquables pour calculer (sans calculatrice) les carrés suivants\,:
	\begin{tasks}(2)
\task Avec $(a+b)^2\,: \quad  23^2\,;\quad  92^2\,;\quad 101^2\,;\quad 42^2$
\task Avec $(a-b)^2\,:\quad 39^2\,;\quad 68^2\,;\quad 99^2\,;\quad 298^2$
	\end{tasks}
\end{exo}
\begin{exo}
Déterminer les identités remarquables pour : $(a+b)^1 ; \quad(a+b)^2 ; \quad(a+b)^3 ; \quad(a+b)^4 ; \quad(a+b)^5$ 

(Expressions réduites et ordonnées selon les puissances décroissantes de $a$)
\end{exo}

\begin{exo}
	Développer et réduire en utilisant les identités remarquables.
\begin{tasks}(2)
	\task $\left(10 r x+8 \right)\left(10 r x+5 \right)$
	\task $\left(s t+ \dfrac{3}{2} s\right)\left(s t-\dfrac{3}{2} s\right)$
	\task $\left(5 s^2 y-\dfrac{5}{8} s^2\right)^2$
	\task $\left(\dfrac{1}{8} x+\dfrac{2}{3} s x\right)^2$
	\task $\left(\dfrac{2}{5} z^2+ \dfrac{3}{5} r^2 z\right)\left(\dfrac{2}{5} z^2-\dfrac{3}{5} r^2 z\right)$
	\task $\left(\dfrac{7}{2} r-4 r t^2\right)^2$
	\task $\left(6 r y+3 \right)\left(6 r y+2 \right)$
	\task $\left(\dfrac{4}{3} z^2+\dfrac{5}{9} r z\right)^2$
	\task $\left(\dfrac{4}{7} t x-8 \right)\left(\dfrac{4}{7} t x+7 \right)$
	\task $\left(10 s+\dfrac{8}{7} t^3\right)^2$
\end{tasks}
\end{exo}
\textLigne{Automatismes}
Automatismes testés en semaine 7, car il y a l'évaluation en semaine 6.

\begin{auto}Développer explicitement à l'aide d'une identité remarquable
(réduire le développement obtenu lorsque c'est possible).

\begin{tasks}(3)
	\task  $\left(-3 \sqrt{3} +5\sqrt{5}\right)^{2}$ 
	\task  $\left(6 \sqrt{3} +6\right)\left(6 \sqrt{3}-6\right)$ 
	\task  $\left(-3 \sqrt{2} +5\sqrt{2}\right)\left(-3 \sqrt{2}-5\sqrt{2}\right)$ 
	\task  $\left(-5 \sqrt{2} +5\right)^{2}$ 
	\task  $\left(4 \sqrt{3} -6\right)^{2}$ 
	\task  $\left(-2 \sqrt{11} -5\right)^{2}$ 
\end{tasks}

\end{auto}

\begin{auto}
Développer explicitement à l'aide d'une identité remarquable
(réduire le développement obtenu lorsque c'est possible).
	\begin{tasks}(2)
\task $(1+\sqrt{5})^2$
\task $(\sqrt{3}+\sqrt{7})^2$
\task $(3 \sqrt{2}-2 \sqrt{3})^2$
\task $(\sqrt{3}+\sqrt{15})(\sqrt{3}+\sqrt{12})$
\task $(2-\sqrt{5})(2+\sqrt{5})$
\task $(\sqrt{6}-\sqrt{2})^2$
	\end{tasks}
\end{auto}


\begin{auto}Factoriser au maximum les expressions suivantes.

\begin{tasks}(2)
	\task $40 t y-10 t^2 + 35 y$
	\task $12 s^2 t + 32 s^2 + 24 s t$
	\task $-30 z + 45 t + 10 $
	\task $70 s z^2 + 50 s z + 60 z$
	\task $50 r s^2 + 40 r^2 + 20 r$
	\task $21 t + 56 $
	\task $72 r x^2 + 32 x^2-56 x$
	\task $-63 y^2-56 x y-42 y$
	\task $-56 s y-40 s + 24 $
	\task $-6 s^2 t + 9 t^3$
\end{tasks}
\end{auto}

% @see : https://coopmaths.fr/alea?uuid=c4b73&id=3L12-2&n=10&d=10&s=1&s2=3&s3=3&s4=4&s5=false&alea=T0jE&cd=1&cols=1
\begin{auto}Factoriser au maximum les expressions suivantes.

\begin{tasks}(2)
	\task $-60 r s y z-48 s z-18 z$
	\task $-30 r^2 t^2 x^2-60 r^2 t^2 x + 10 r t x^2$
	\task $-21 s t x z + 42 s x z + 56 s$
	\task $-2 s y z^2-16 s t + 14 s$
	\task $10 s t^2 x^2 y-14 s^2 t x$
	\task $-18 r x z^2 + 21 t z-15 z$
	\task $r s t + 7 r t$
	\task $56 r t x y^2 + 48 r t x^2$
	\task $35 r x^3 + 42 x y z^2$
	\task $-3 r y + 15 y + 15 $
\end{tasks}
\end{auto}

\begin{comment}
\begin{auto}
	Factoriser (au maximum) à l'aide de la mise en évidence.
	\begin{tasks}(3)
\task $2 x^2-4 x y$
\task $4 a^2-16 a b$
\task $3 a^3-9 a b$
\task $a^3-2 a^2$
\task $5 x^3 y-15 x y^3$
\task $14 a b-7 a b^2$
\task $3 v^4-6 v w$
\task $7 x^2 y^3-14 x y^4$
\task $2 a^4-8 a^3$
\task $4 a^3 b-8 a b^3$
\task $15 a^4-5 a$
\task $44 x^2-22 x y^4$
\task $8 x^3 y z^2-16 x^2 y^2 z$
\task $3 a^3-7 a^4$
\task $3 x^3 z^3-2 x^3 y^3$
\task $12 a^4-24 a^4 b$
\task $2 x^4-26 x y^2$
\task $2 a^3-14 b^2$
\task $2 a^3 b-4 a b^2+8 a b$
\task $2 a b^3-16 a^3 b+4 a^3 b^3$
\task $3 a^4 b^3-12 a^3 b+9 a b^4$
\task $5 t^2 u-10 t u^3+15 t^2 u^2$
\task $7 x^4 y-14 x^2 y^4+21 x y^5$
\task $13 x^4 y^5-26 x^2 y^3+169 x^4 y^4$
	\end{tasks}
\end{auto}

\begin{auto}
	Trouver deux nombres $a$ et $b$ dont ...
	\begin{tasks}(2)
\task le produit vaut $6$ et la somme $5$
\task le produit vaut $12$ et la somme $7$
\task le produit vaut $12$ et la somme $8$
\task le produit vaut $12$ et la somme $13$
\task le produit vaut $12$ et la somme $-7$
\task le produit vaut $-5$ et la somme $4$
\task le produit vaut $10$ et la somme $-7$
\task le produit vaut $-9$ et la somme $8$
\task le produit vaut $-8$ et la somme $-2$
\task le produit vaut $15$ et la somme $-8$
\task le produit vaut $48$ et la somme $14$
\task le produit vaut $24$ et la somme $11$
\task le produit vaut $7$ et la somme $8$
\task le produit vaut $-20$ et la somme $-8$
\task le produit vaut $-20$ et la somme $1$
\task le produit vaut $36$ et la somme $12$
\task le produit vaut $-40$ et la somme $3$
\task le produit vaut $28$ et la somme $-11$
\end{tasks}
\end{auto}

57. $\left({ }^*\right)$ On considère le nombre $12345678913 \cdot 12345678907-12345678910^2$.
a) Calculer algébriquement sa valeur, en posant $12345678910=a$.
b) Calculer algébriquement sa valeur, en posant $12345678913=b$.
c) Calculer algébriquement sa valeur, en posant $12345678907=c$.
58. $\left(^*\right)$ Trouver les identités remarquables pour : $(a+b)^3$ et $(a-b)^3$.
Utiliser ces identités pour développer directement : $(2 x+3)^3$ et $(\sqrt{2}-1)^3$.
61. $\left(^*\right)$ Voici un grand rectangle qui n'est construit qu'avec des carrés. Les lettres $x$ et $y$ représentent .la longueur du côté de deux de ces carrés (celui ombré et celui du centre en bas).
a) Exprimer le périmètre du grand rectangle en fonction de $x$ et de $y$.
b) En s'aidant des longueurs des côtés du rectangle, exprimer le périmètre et l'aire du rectangle en fonction de $x$.
c) Exprimer le périmètre et l'aire du rectangle en fonction de $y$.
\end{comment}

\end{document}

