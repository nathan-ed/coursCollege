\documentclass[a4paper,12pt]{report}
\usepackage{../courslatex}
\usepackage{theorie}

\setcounter{chapter}{2}

\renewcommand{\titreChapitre}{Ch. \thechapter\,: Calcul littéral}
\begin{document}
\chapter{Calcul littéral}
\thispagestyle{fancy}
\section{Opération sur les polynômes}
\subsection{Rappels sur l'addition et la multiplication}
Voici quelques rappels qui formalisent certaines de vos connaissances sur l'addition et la multiplication. 

Ces deux opérations ont beaucoup de points communs, car elles forment sur l'ensemble des nombres réels pour l'addition et l'ensemble des nombres réels différents de zéro une structure mathématique plus générale appelée \enquote{un groupe abélien}.

\begin{description}[leftmargin= 2cm]
	\item[] 
		\emph{L'élément neutre} est l'élément qui \enquote{ne fait rien} pour l'opération en question.

		\begin{minipage}[t]{0.4\textwidth}{
		\vspace{0pt}
		{\bfseries Addition}

		Pour l'addition, l'élément neutre est le $0$, car
		\[0+a=a+0=a \quad \forall a\in \R\]
		}
		\end{minipage}
		\hfill
		\begin{minipage}[t]{0.4\textwidth}{
		\vspace{0pt}
		{\bfseries Multiplication}

		Pour la multiplication, l'élément neutre est le $1$, car
		\[1\cdot a=a\cdot 1=a \quad \forall a\in \R^*\]
		}
		\end{minipage}
	\item[]
		On appelle \emph{l'opposé} de $a\in \R$ dans le cas de l'addition l'élément qui permet \enquote{d'atteindre} $0$. Ce nombre existe et on le note $(-a)$. On a  
		\[a+(-a)=(-a)+a=0\]
		Remarquons que l'opposé de l'opposé de $a$ est égal à $a$. 

		De plus, l'opposé d'un nombre n'est pas toujour négatif. En effet, l'opposé de $-3$ est $3$.

		Sur la calcultrice, la petite touche $(-)$ permet justement de noter \enquote{l'opposé} d'un nombre en opposition à la plus grande touche $-$ utilisée pour dénoter l'opération \enquote{moins}.
	\item[]
		On appelle \emph{l'inverse} de $a\in \R^*$ dans le cas de la multiplication l'élément qui permet \enquote{d'atteindre} $1$. Ce nombre existe et on le note $\dfrac{1}{a}$. On a 
		\[a\cdot \dfrac{1}{a}=\dfrac{1}{a}\cdot a=1\]
		Comme pour l'addition, l'inverse de l'inverse de $a$ est égal à $a$. On ne considère que $\R^*$, car $0$ n'a pas d'inverse.
	\item[Associativité]

		\begin{minipage}[t]{0.4\textwidth}{
		\vspace{0pt}
		{\bfseries Addition}

		L'associativité de l'addition nous dit que pour tout $a,b,c\in \R$, on a 
		\[(a+b)+c=a+(b+c)\]
				}
		\end{minipage}
		\hfill
		\begin{minipage}[t]{0.4\textwidth}{
		\vspace{0pt}
		{\bfseries Multiplication}

	L'associativité de la multiplication nous dit que pour tout $a,b,c\in \R^*$, on a 
	\[(a\cdot b)\cdot c=a\cdot(b\cdot c)\]
		}
		\end{minipage}
	\item[Commutativité]

		\begin{minipage}[t]{0.4\textwidth}{
		\vspace{0pt}
		{\bfseries Addition}

		La commutativité de l'addition nous dit que pour tout $a,b\in \R$, on a
		\[a+b=b+a\]
		}
		\end{minipage}
		\hfill
		\begin{minipage}[t]{0.4\textwidth}{
		\vspace{0pt}
		{\bfseries Multiplication}
	
		La commutativité de la multiplication nous dit que pour tout $a,b\in \R^*$, on a
		\[a\cdot b=b\cdot a\]
	}
		\end{minipage}
\end{description}
\begin{rem}
Calculons deux inverses.

\begin{minipage}[t]{0.4\textwidth}{
\vspace{0pt}
L'inverse de $\dfrac{4}{5}$ vaut $\dfrac{5}{4}$, car $\dfrac{5}{4}\cdot \dfrac{4}{5}=1$}
\end{minipage}
\hfill
\begin{minipage}[t]{0.5\textwidth}{
\vspace{0pt}
L'inverse de $-\dfrac{\sqrt{5}}{3}$ vaut $-\dfrac{3}{\sqrt{5}}$, car
\[-\dfrac{\sqrt{5}}{3}\cdot (-\dfrac{3}{\sqrt{5}})=1.\]
Toutefois, on n'accepte pas les racines au dénominateur, donc $-\dfrac{3}{\sqrt{5}}=-\dfrac{3\sqrt{5}}{5}$.
}
\end{minipage}
\end{rem}
Une dernière propriété à garder en tête est la distributivité de la multiplication par rapport à l'addition. 
Elle est valable pour les nombres, mais également dans un contexte plus général du calcul littéral.

\begin{minipage}[t]{0.4\textwidth}{
\vspace{0pt}
\begin{center}
{\bfseries Distributivité simple}

\end{center}
\[a\cdot (b+c)=ab+ac\]
}\end{minipage}
\hfill
\begin{minipage}[t]{0.5\textwidth}{
\vspace{0pt}
\begin{center}
{\bfseries Distributivité double}

\end{center}
\[(a+b)\cdot (c+d)= ac+ad+bc+bd\]
}\end{minipage}

\vspace{1em}
Nous utiliserons très souvent la distributivité simple et double. 
\subsection{Conventions et définitions}
On commence à présent le calcul littéral avec des rappels sur les conventions et des définitions vues au cycle.
\begin{center}
{\bfseries Conventions}
\begin{enumerate}
	\item[] On ne note pas le signe de multiplication entre 
		\begin{itemize}
			\item un nombre et une variable, p. ex. $3\cdot a =3a$\,;
			\item un nombre et une parenthèse, p. ex. $3\cdot (x+y)=3(x+y)$\,;
			\item deux variables, p. ex. $x\cdot y=xy$\,;
			\item une variable et une parenthèse, p. ex. $x\cdot (1+y)=x(1+y)$\,;
			\item deux parenthèses, p. ex. $(x+y)\cdot (y+1)=(x+y)(y+1)$.
		\end{itemize}
	\item[] On note $x$ pour $1x$.
	\item[] On note $0$ pour $0x$.
\end{enumerate}
	\end{center}
\begin{defi}[monôme]
	Un \emph{monôme} est le produit d'un nombre réel, appelé  \emph{coefficient}, et d'une ou plusieurs variables élevées à certaines puissances entières positives, appelé \emph{partie littérale}. Le \emph{degré} d'un monôme est la somme des exposants de la partie littérale.
\end{defi}
Par exemple
\begin{tasks}
	\task $4xy^2$ est un monôme, son coefficient est $4$, sa partie littérale $xy^2$ et son degré est $1+2=3$\,; 
	\task $3$ est un monôme, son coefficient est $3$, il n'a pas de partie littérale et son degré est $0$\,;
	\task $x$ est un monôme, son coefficient est $1$, sa partie littérale est $x$ et son degré est $1$. 
\end{tasks}
\begin{defi}[monômes semblables]
	Deux monômes sont \emph{semblables} s'ils ont la même partie littérale.
\end{defi}
Par exemple
\begin{tasks}(2)
	\task $1$ et $4$ sont semblables\,;
	\task $x$ et $3x^2$ ne sont pas semblables\,;
	\task $2y$ et $3xy$ ne sont pas semblables\,;
	\task $4x^2yz$ et $\pi x^2yz$ sont semblables.
\end{tasks}
\begin{defi}[polynômes]
	Un \emph{polynôme} est une somme de monômes. 
\end{defi}
Remarquez qu'une soustraction peut être transformée en une somme, ainsi une soustraction de monômes est également un polynôme. Par exemple
\begin{tasks}
	\task $2xy^3$ est un monôme, mais aussi un polynôme\,;
	\task $x+y-xy^2$ est un polynôme.
\end{tasks}
\begin{defi}[opposé]
Soit $P$ un polynôme, \emph{l'opposé} de $P$ est le polynôme $R$ tel que $P+R=0$.	
\end{defi}
Par exemple
\begin{tasks}
	\task l'opposé de $-2xyz$ est $2xyz$\,;
	\task l'opposé de $3x^2-2xy+1$ est $-(3x^2-2xy+1)=-3x^2+2xy-1$.
\end{tasks}
On remarque que l'opposé d'un polynôme $P$ est le polynôme $-P$. 

\begin{rem}
	Afin de déterminer l'expression du polynôme $-P$, on réduit l'expression $-(P)$ et on se rappelle que $-(P)=-1\cdot (P)$.
	{\bfseries Il faut appliquer la distributivité~!} 
\end{rem}
\subsection{Addition et soustraction}
On définit d'abord l'addition et la soustraction de monômes, puis on passe aux opérations sur les polynômes.

{\bfseries On peut additionner (soustraire) deux monômes seulement s'ils sont semblables.} Dans ce cas, on additionne (soustrait) les coefficients entre-eux {\bfseries sans changer} la partie littérale.

Par exemple
\begin{tasks}
	\task $2x^3+4x$, les deux monômes ne sont pas semblables, on ne peut rien faire.
	\task $4x^2y-5x^2y$ les deux monômes sont semblables, on a $4xy^2-5x^2y=(4-5)x^2y=-x^2y$. 
	\task $8x-3x^2+2x-x^3+x^2$, on détermine les monômes semblables et on réduit. On a l'habitude d'utiliser des couleurs ou des codes (souligner une ou plusieurs fois, souligner en vaguelette, etc.) pour différencier les \enquote{différents} monômes semblables (voici un exemple avec {\bfseries tous} les détails)~: 
\end{tasks}
\begin{align*}
	\uwave{8x}-\uline{3x^2}+\uwave{2x}-\uuline{x^3}+\uline{x^2}&\stackrel{\text{regrouper}}{=}8x+2x-3x^2+x^2-x^3=(8+2)x+(-3+1)x^2-x^3\\
&\stackrel{\text{réduire}}{=}10x-2x^2-x^3
\end{align*}
Pour additioner deux polynômes, on additionne les monômes semblables qui les constituent, par exemple,
\begin{align*}
	(3xy-7x-3xy^2)+(9x+3xy+x^3)&=3xy-7x-3xy^2+9x+3xy+x^3\\
				   &=2x+6xy+x^3-3xy^2
\end{align*}
\begin{rem}
	On ordonne le résultat selon l'ordre croissant du degré des monômes.
\end{rem}
Afin de soustraire un polynôme à un autre polynôme, on utilise la définition de l'opposé. {\bfseries Soustraire c'est additionner l'opposé}, dès lors
\[A-B=A+\text{opposé}(B).\]
Par exemple avec $A=3x-2xy+9xy^2z$ et $B=-4x+3y-7xy^2z$, 
\begin{align*}
	(3x-2xy+9xy^2z)-(-4x+3y-7xy^2z)&=3x-2xy+9xy^2z\uline{+}4x\uline{-}3y\uline{+}7xy^2z\\
&=7x-3y-2xy+16xy^2z
\end{align*}
\subsection{Multiplication}
On définit comme pour l'addition la multiplication sur les monômes. Afin de multiplier deux monômes, on multiplie les coefficients entre-eux et les parties littérales entre-elles. 
Par exemple, 
\begin{tasks}(2)
	\task $3x^2\cdot 2y=3\cdot 2\cdot x^2\cdot y=6x^2y$
	\task $5x(-4xy)=5\cdot (-4)\cdot x\cdot xy=-20x^2y$
\end{tasks}

Pour multipier deux (ou plus) polynômes, on applique la distributivité. Par exemple, 
\begin{align*}
(3x^2-4yz-z^2)(x-y)&\stackrel{\text{dist.}}{=}3x^2\cdot x+3x^2\cdot (-y)+(-4yz)\cdot x+(-4yz)\cdot (-y)+(-z^2)\cdot x+(-z^2)\cdot (-y)\\
&\stackrel{\text{réduire}}{=}3x^3-3x^2y-4xyz+4y^2z-xz^2+yz^2
\end{align*}
{\bfseries Terminologie} 
\begin{description}
	\item[] Une expression est dite \emph{réduite} si on ne peut plus effectuer d'addition, de soustraction ou de multiplication de monômes. Par exemple, $3x^2\cdot 2y+4$ et $3y^2+x-y^2$ ne sont pas réduites.
	
	\item[Expression développée] Une expression est dite \emph{développée} si tous les produites de monômes ont été effectués. Par exemple $x\cdot (1+y)$ et $1+3y²\cdot 4x$ ne sont pas des expressions développées.
\end{description}

\section{Identités remarquables}
Les identités remarquables sont des identités à connaître par coeur. {\bfseries Elles permettent de faciliter le développement et la factorisation d'expressions.}
Les voici:

\begin{center}
{\bfseries Les quatre identités}
\begin{align}
	(a+b)^2&=a^2+2ab+b^2\\[1em]
	(a-b)^2&=a^2-2ab+b^2\\[1em]
	(a-b)(a+b)&=(a+b)(a-b)=a^2-b^2\\[1em]
	(x+a)(x+b)&=x^2+(a+b)x+ab \text{appelée \emph{l'identité somme-produit}}
\end{align}
\end{center}
Ces identités ne sont pas \enquote{inventées}.
On peut les retrouver en développant les expressions,
\begin{align*}
	\bm{(a+b)^2}&=(a+b)(a+b)=a^2+ab+ba+b^2=\bm{a^2+2ab+b^2}\\[1em]
	\bm{(a-b)^2}&=(a-b)(a-b)=a^2-ab-ba+b^2=\bm{a^2-2ab+b^2}\\[1em]
	\bm{(a+b)\cdot (a-b)}&=a^2-ab+ba+b^2=\bm{a^2-b^2}\\[1em]
	\bm{(x+a)(x+b)}&=x^2+ax+bx+ab=\bm{x^2+(a+b)x+ab}
\end{align*}
L'identité \enquote{somme-produit} est la plus générale puisqu'elle permet de retrouver toutes les autres comme des cas particuliers. 

Historiquement, elle joue un rôle important, car elle constitue une des motivations pour résoudre des équations du second degré comme nous le verrons par la suite. 

\section{Factorisation}
\begin{center}
Factoriser c'est transformer une {\bfseries somme} en un {\bfseries produit.}
\[\underbrace{(\cdots)\underset{\text{ou } -}{+}(\cdots)\underset{\text{ou } -}{+}(\cdots)}_{\text{ somme }}\longrightarrow \underbrace{(\cdots)\cdot (\cdots)}_{\text{produit}}\]
\end{center}

Nous verrons trois techniques différentes de factorisation qu'il faut appliquer par enchaînement. 
\subsection{Mise en évidence de facteurs communs}
Lorsque des termes ont des facteurs communs on applique la mise en évidence. Par exemple, 
\[3x+5x^2=3\tikzmarknode{1}{\uline{x}}+5x\cdot \tikzmarknode{2}{\uline{x}}= \tikzmarknode{3}{\uline{x}}(3+5x) \quad \text{on a mis le facteur } x \text{ en évidence}\]
\begin{tikzpicture}[remember picture, overlay, font=\sffamily]
\draw[blue,->] ([yshift=-1pt]1.south) -- ++(0,-8pt) coordinate (aux1) -| ([yshift=-1pt]3.south) ;
\draw[blue,-] (aux1 -|2) -- ([yshift=-1pt]3.south -|2);
\end{tikzpicture}
\[6x+9y=\tikzmarknode{1}{\uline{3}}\cdot 2 x + \tikzmarknode{2}{\uline{3}}\cdot 3y= \tikzmarknode{3}{\uline{3}}(2x+3y)\quad \text{on a mis le facteur } 3 \text{ en évidence}
\]
\begin{tikzpicture}[remember picture, overlay, font=\sffamily]
\draw[blue,->] ([yshift=-1pt]1.south) -- ++(0,-8pt) coordinate (aux1) -| ([yshift=-1pt]3.south);
\draw[blue,-] (aux1 -|2) -- ([yshift=-1pt]3.south -|2);
\end{tikzpicture}
\[
	4xz+6yx^2=\tikzmarknode{1}{\uline{2x}}\cdot 2 z + \tikzmarknode{2}{\uline{2x}}\cdot 3xy= \tikzmarknode{3}{\uline{2x}}(2z+3xy)\quad \text{on a mis le facteur } 2x \text{ en évidence}
\]
\begin{tikzpicture}[remember picture, overlay, font=\sffamily]
\draw[blue,->] ([yshift=-1pt]1.south) -- ++(0,-8pt) coordinate (aux1) -| ([yshift=-1pt]3.south) ;
\draw[blue,-] (aux1 -|2) -- ([yshift=-1pt]3.south -|2);
\end{tikzpicture}

Afin de mettre en évidence un facteur commun, on identifie les facteurs communs dans les coefficients et la partie littérale des termes de l'expression littérale. 

Afin de mettre en évidence \enquote{le plus possible} possible de facteurs communs, on identifie la partie littérale commune à tous les termes et on détermine le pgcd des coefficients.

Par exemple, 
\[24x^2y-36x^2yz+48x^2yz^2=\tikzmarknode{11}{\uuline{6}}\cdot 4\tikzmarknode{1}{\uline{x^2y}}-\tikzmarknode{22}{\uuline{6}}\cdot 6\tikzmarknode{2}{\uline{x^2y}}\cdot z+\tikzmarknode{33}{\uuline{6}}\cdot 8\tikzmarknode{3}{\uline{x^2y}}\cdot z^2=\tikzmarknode{44}{\uuline{6}}\tikzmarknode{4}{\uline{x^2y}}(4-6z+8z^2).\]	
\begin{tikzpicture}[remember picture, overlay, font=\sffamily]
\draw[blue,->] ([yshift=-1pt]1.south) -- ++(0,-8pt) coordinate (aux1) -| ([yshift=-1pt]4.south) ;
\draw[blue,-] (aux1 -|2) -- ([yshift=-1pt]3.south -|2);
\draw[blue,-] (aux1 -|3) -- ([yshift=-1pt]4.south -|3);
\draw[blue,->] ([yshift=1pt]11.north) -- ++(0,8pt) coordinate (aux1) -| ([yshift=1pt]44.north) ;
\draw[blue,-] (aux1 -|22) -- ([yshift=1pt]33.north -|22);
\draw[blue,-] (aux1 -|33) -- ([yshift=1pt]44.north -|33);
\end{tikzpicture}

\begin{rem}
Lorsqu'on vous demande de factoriser, il faudra toujours mettre en évidence le plus de facteurs communs possibles, c'est-à-dire qu'il faut répéter le processus jusqu'à ce que les termes dans la parenthèse n'ait plus aucun facteur commun.
\end{rem}
\subsection{Utilisation des identités remarquables}
On utilise également les identités remarquables pour factoriser une expression. 
\begin{center}
\begingroup
\renewcommand*{\arraystretch}{1.5}
\begin{tabular}{lcl}
forme développée && forme factorisée\\
\cline{1-1}\cline{3-3}
$a^2+2ab+b^2$&&$(a+b)^2$\\
$a^2-2ab+b^2$&$\longrightarrow$&$(a-b)^2$\\
$a^2-b^2$&&$(a-b)(a+b) \text{ ou } (a+b)(a-b)$\\
$c^2x^2+(a+b)cx+ab$&&$(cx+a)(cx+b)$\\
\end{tabular}
\endgroup
\end{center}
\begin{rem}
	Nous avons rajouté un coefficient $c$ dans l'identité \enquote{somme-produit} pour ne pas oublier que le coefficient influence également le terme \enquote{$(a+b)cx$}.
\end{rem}
Voici la procédure à suivre pour factoriser avec des identités remarquables

\tikzstyle{startstop} = [rectangle, rounded corners, minimum width=3cm, minimum height=1cm,text centered, draw=black,
execute at begin node={\begin{varwidth}{15em}},
execute at end node={\end{varwidth}}]
\tikzstyle{io} = [trapezium, trapezium left angle=70, trapezium right angle=110, minimum width=3cm, minimum height=1cm, text centered, draw=black, fill=blue!30]
\tikzstyle{process} = [rectangle, minimum width=3cm, minimum height=1cm, text centered, draw=black, fill=orange!30]
\tikzstyle{decision} = [diamond, minimum width=3cm, minimum height=1cm, text centered, draw=black, fill=green!30]
\tikzstyle{arrow} = [thick,->,>=stealth]
\begin{center}
\begin{tikzpicture}[scale=0.8,node distance=1.5cm]
	\node (e1) [startstop] {\centerline{Identifier si l'expression} \centerline{peut être}\centerline{une identité remarquable}};
\node (e21) [startstop, below of=e1,  xshift=-6cm, yshift=-1cm] {\centerline{Différence}\centerline{de deux carrés\,?} \centerline{$a^2-b^2$}};
\node (e22) [startstop, below of=e1,yshift=-1cm] {\centerline{Somme}\centerline{de deux carrés}\centerline{avec un double produit\,?} \centerline{$a^2\pm2\cdot \ldots +b^2$}};
\node (e23) [startstop, below of=e1,  xshift=6cm, yshift=-1cm] {\centerline{Présence} \centerline{d'un seul carré~?}};
\node (e34) [startstop, below of=e23, yshift=-1cm, xshift=0.5cm] {\centerline{Essayer de reconstituer} \centerline{l'identité \enquote{somme-produit}}\centerline{$\bm{(cx+a)(cx+b)}$}};
\node (e31) [startstop, below of=e21, yshift=-1cm,xshift=-0.5cm] {\centerline{$\bm{(a-b)(a+b)}$}};
\node (e32) [startstop, below of=e22, xshift=-1.8cm, yshift=-1cm] {\centerline{$a^2+2ab+b^2$\,?}\centerline{$\bm{(a+b)^2}$}};
\node (e33) [startstop, below of=e22, xshift=1.8cm, yshift=-1cm] {\centerline{$a^2-2ab+b^2$\,?}\centerline{$\bm{(a-b)^2}$}};
\draw[arrow] (e1)--(e21);
\draw[arrow] (e1)--(e22);
\draw[arrow] (e1)--(e23);
\draw[arrow] (e21)--(e31);
\draw[arrow] (e22)--(e32);
\draw[arrow] (e22)--(e33);
\draw[arrow] (e23)--(e34);
\end{tikzpicture}
\end{center}
Les diagrammes suivant permettent de rendre compte du processus à suivre lorsqu'il faut factoriser une identité remarquable.

\tikzset{
    % Set the overall layout of the tree
    level 1/.style={level distance=3.5cm, sibling distance=3.5cm},
    level 2/.style={level distance=3.5cm, sibling distance=2cm},
    % Define styles for bags and leafs
    bag/.style={text width=4em, text centered},
    end/.style={circle, minimum width=3pt,fill, inner sep=0pt}
}
\begin{minipage}[t]{0.5\textwidth}{
\vspace{0pt}
\begin{tikzpicture}[scale=0.8,grow=right, sloped,
    ebox/.style={below=1mm, minimum width=10mm, minimum height=6mm, draw}
]
\node[xshift=3cm,above=1.2cm] {{\bfseries Identités 1 et 2}};
\node[xshift=3cm,above=0.7cm] {$\bm{a^2\pm2ab+b^2}$};
\node[xshift=3cm, above=0.2cm] {si trois termes};
\node[xshift=3cm] {dont somme de deux carrés};
\node[bag] {$a^2$};
	child{
\node(1)[bag,below=0.5cm] {\emptybox};
\node[below=0.9cm,xshift=3cm] {$\leftrightarrow$ identifier les carrés $\leftrightarrow$}; 
\node(11)[bag,below=0.5cm, xshift=6cm] {\emptybox};
child{
	\node(2a)[below=2.5cm] {$a$};
	child{
		\node(2)[bag,below=3cm] {\emptybox};
		\node[below=3.4cm,xshift=3cm] {$\leftrightarrow$ déterminer $a$ et $b$ $\leftrightarrow$}; 
		\node(22)[bag,below=3cm, xshift=6cm] {\emptybox};
	}
	\node(2b)[bag,below=2.5cm,xshift=6cm] {$b$};
}
child{
	\node(3ab)[below=4.5cm,xshift=3cm]{vérifier que le terme restant vaut $2ab$};
	\node(3)[bag,below=5cm,xshift=3cm]{\emptybox};
	\node[bag,below=6.5cm,xshift=3cm]{$(a\pm b)^2$};
}
}
\node[bag,xshift=6cm] {$b^2$};
\draw[->] (1)--(2a);
\draw[->] (11)--(2b);
\draw[->] (2)--(3ab);
\draw[->] (22)--(3ab);
\end{tikzpicture}

}
\end{minipage}
\begin{minipage}[t]{0.5\textwidth}{
\vspace{0pt}
\begin{tikzpicture}[scale=0.8,grow=right, sloped,
    ebox/.style={below=1mm, minimum width=10mm, minimum height=6mm, draw}
]
\node[xshift=3cm,above=0.7cm] {{\bfseries Identité 3}};
\node[xshift=3cm,above=0.2cm] {\bm{$a^2-b^2$}};
\node[xshift=3cm] {si différence de deux carrés};
\node[bag] {$a^2$};
	child{
\node(1)[bag,below=0.5cm] {\emptybox};
\node[below=0.9cm,xshift=3cm] {$\leftrightarrow$ identifier les carrés $\leftrightarrow$}; 
\node(11)[bag,below=0.5cm, xshift=6cm] {\emptybox};
child{
	\node(2a)[below=2.5cm] {$a$};
	child{
		\node(2)[bag,below=3cm] {\emptybox};
		\node[below=3.4cm,xshift=3cm] {$\leftrightarrow$ déterminer $a$ et $b$ $\leftrightarrow$}; 
		\node(22)[bag,below=3cm, xshift=6cm] {\emptybox};
	}
	\node(2b)[bag,below=2.5cm,xshift=6cm] {$b$};
}
child{
	\node[below=4cm,xshift=3cm]{$(a-b)(a+b)$};
}
}
\node[bag,xshift=6cm] {$b^2$};
\draw[->] (1)--(2a);
\draw[->] (11)--(2b);
\end{tikzpicture}

}
\end{minipage}

Par exemple, 

\begin{minipage}[t]{0.5\textwidth}{
\vspace{0pt}
\begin{tikzpicture}[scale=0.8,grow=right, sloped,
    ebox/.style={below=1mm, minimum width=10mm, minimum height=6mm, draw}
]
\node[xshift=3cm] {$\bm{9x^2+6xy+y^2}$};
\node[bag] {$a^2$};
	child{
	\node(1)[bag,below=0.5cm] {\boxt{$9x^2$}};
\node[below=0.9cm,xshift=3cm] {$\leftrightarrow$ identifier les carrés $\leftrightarrow$}; 
\node(11)[bag,below=0.5cm, xshift=6cm] {\boxt{$y^2$}};
child{
	\node(2a)[below=2.5cm] {$a$};
	child{
		\node(2)[bag,below=3cm] {\boxt{$3x$}};
		\node[below=3.4cm,xshift=3cm] {$\leftrightarrow$ déterminer $a$ et $b$ $\leftrightarrow$}; 
		\node(22)[bag,below=3cm, xshift=6cm] {\boxt{$y$}};
	}
	\node(2b)[bag,below=2.5cm,xshift=6cm] {$b$};
}
child{
	\node(3ab)[below=4.5cm,xshift=3cm]{vérifier que le terme restant vaut $2ab$};
	\node(3)[bag,below=5cm,xshift=2.5cm]{\fbox{$2\cdot 3x\cdot y=6xy$\checkmark}};
	\node[below=6cm,xshift=3cm]{$\bm{(3x+ y)^2}$};
}
}
\node[bag,xshift=6cm] {$b^2$};
\draw[->] (1)--(2a);
\draw[->] (11)--(2b);
\draw[->] (2)--(3ab);
\draw[->] (22)--(3ab);
\end{tikzpicture}

}
\end{minipage}
\begin{minipage}[t]{0.5\textwidth}{
\vspace{0pt}
\begin{tikzpicture}[scale=0.8,grow=right, sloped,
    ebox/.style={below=1mm, minimum width=10mm, minimum height=6mm, draw}
]
\node[xshift=3cm] {$\bm{25x^2y^2-49z^2}$};
\node[bag] {$a^2$};
	child{
		\node(1)[bag,below=0.5cm] {\boxt{$x^2y^2$}};
\node[below=0.9cm,xshift=3cm] {$\leftrightarrow$ identifier les carrés $\leftrightarrow$}; 
\node(11)[bag,below=0.5cm, xshift=6cm] {\boxt{$49z^2$}};
child{
	\node(2a)[below=2.5cm] {$a$};
	child{
		\node(2)[bag,below=3cm] {\boxt{$xy$}};
		\node[below=3.4cm,xshift=3cm] {$\leftrightarrow$ déterminer $a$ et $b$ $\leftrightarrow$}; 
		\node(22)[bag,below=3cm, xshift=6cm] {\boxt{$7z$}};
	}
	\node(2b)[bag,below=2.5cm,xshift=6cm] {$b$};
}
child{
	\node[below=4cm,xshift=3cm]{$\bm{(xy-7z)(xy+7z)}$};
}
}
\node[bag,xshift=6cm] {$b^2$};
\draw[->] (1)--(2a);
\draw[->] (11)--(2b);
\end{tikzpicture}

}
\end{minipage}

Pour l'identité \enquote{somme-produit}, une difficulté est de reconnaître le terme \enquote{somme} et le terme \enquote{produit}. Voici comment procéder.

\vspace{-0.5cm}
\begin{minipage}[t]{0.5\textwidth}{
\vspace{0pt}
\begin{tikzpicture}[scale=0.7,grow=right, sloped,
    ebox/.style={below=1mm, minimum width=10mm, minimum height=6mm, draw}
]
\node[xshift=3cm,above=0.7cm] {{\bfseries Identité \enquote{somme-produit}}};
\node[xshift=3cm,above=0.2cm] {\bm{$c^2x^2+(a+b)cx+ab$}};
\node[xshift=3cm] {si un seul carré};
\node[bag] {$c^2 x^2$};
	child{
\node(1)[bag,below=0.5cm] {\emptybox};
\node(1t)[below=0.9cm,xshift=3cm] {identifier le carré}; 
\node(2t)[below=1.3cm,xshift=3cm] {et déterminer $cx$}; 
child{
	\node(2a)[below=2.5cm] {$cx$};
	child{
		\node(2)[bag,below=3cm] {\emptybox};
		\node[below=3.4cm,xshift=3cm] {vérifier si un autre terme}; 
		\node[below=3.8cm,xshift=3cm] {est divisible par $cx$}; 
		\node[below=5cm] {\checkmark on continue};
			\node[below=5cm, xshift=5cm] {\XSolidBrush l'identité \enquote{somme-produit}};
		\node[below=6.5cm]{le terme divisible est};
		\node[below=6.9cm]{{\bfseries un candidat pour}};
		\node[below=7.4cm]{le terme \enquote{somme}};
		\node[below=7.9cm]{\emptybox};
		\node[below=9.2cm]{le terme divisé};
		\node[below=9.7cm]{\emptybox};
		\node[below=6.5cm, xshift=4.5cm]{le terme non divisible est};
		\node[below=6.9cm, xshift=4.5cm]{{\bfseries un candidat pour}};
		\node[below=7.4cm,xshift=4.5cm]{le terme \enquote{produit}};
		\node[below=7.9cm,xshift=4.5cm]{\emptybox};
		\node[below=11cm,xshift=2.5cm]{reconstituer les};
		\node[below=11.4cm,xshift=2.5cm]{termes $a$ et $b$};
		\node[below=12.2cm,xshift=4.5cm]{ $b$};
		\node[below=12.7cm,xshift=4.5cm]{\emptybox};
		\node[below=12.2cm]{ $a$};
		\node[below=12.7cm]{\emptybox};
		\node[below=13.7cm,xshift=2.5cm]{vérifier};
		\node[below=14.5cm]{$(a+b)cx$};
		\node[below=15cm]{\emptybox};
		\node[below=14.5cm,xshift=4.5cm]{$ab$};
		\node[below=15cm,xshift=4.5cm]{\emptybox};
	\node[below=16.1cm,xshift=2.3cm]{$\bm{(cx+a)(cx+b)}$};
	}
}
}

\draw[->] (1)--(2a);
\draw[<->] (1)--(1t);
\draw[->] (2t)--(2a);
%\draw[->] (2)--(3ab);
%\draw[->] (22)--(3ab);
\end{tikzpicture}

}
\end{minipage}
\begin{minipage}[t]{0.5\textwidth}
{
	t
}
\end{minipage}

\subsection{Mise en groupements -- enchaînement des techniques}
Nous avons appliqué ces techniques à un terme de l'expression.
Toutefois, il se peut que l'on applique la mise en évidence de facteurs communs ou la factorisation par une identité remarquable à une partie d'une expression. 

On commence par mettre en groupements des termes qui pourraient être factorisés et on applique successivement les méthodes vues ci-dessus.
Par exemple,
\begin{tasks}
	\task \vspace{-2em}\begin{align*}
	x^3-3x^2+7x-21&=(x^3-3x^2)+(7x-21) &&\rightarrow\text{ groupements}\\
		      &=x^2(x-3)+7(x-3)&&\rightarrow \text{mise en évidence}\\
		      &=x^2\uline{(x-3)}+7\uline{(x-3)}&&\rightarrow \text{facteur commun}\\
		      &=(x-3)(x^2+7)&&\rightarrow \text{mise en évidence}
\end{align*}

\task	
\vspace{-2em}\begin{align*}
	25x^2-20xy+4y^2-9&=(25x^2-20xy+4y^2)+(-9)&&\rightarrow \text{ groupements}\\
		      &=(5x-2y)^2-9&&\rightarrow \text{identité remarquable}\\
		      &=(5x-2y)^2-(3)^2&& \rightarrow \text{constitution de l'identité}\\
		      &=(5x-2y-3)(5x-2y+3)&&\rightarrow \text{3e identité remarquable}\\
\end{align*}
		\end{tasks}
		\vspace{-1cm}
\begin{rem}
	Lorsqu'on demande de factoriser un polynôme, on veut que le résultat soit factorisé au maximum. 
\end{rem}
Comme avec la décomposition d'entiers en produit de facteurs premiers, on commence par une première décomposition et on applique à nouveau la décomposition à chaque facteur
	\[
	48=6\cdot 8=(2\cdot 3)\cdot (4\cdot 2)=2\cdot 3\cdot (2\cdot 2)\cdot 2=2^4\cdot 3	
\]
Voici encore deux exemples où on enchaîne la mise en évidence de facteurs communs puis la factorisation à l'aide des identités remarquables.
\begin{tasks}
	\task 
	\vspace{-2em}
	\begin{align*}
		2a^4+2a^3-40a^2&=\underset{\text{repérage du facteur commun}}{\underbrace{\bm{2a^2}\cdot a^2+\bm{2a^2}\cdot a+\bm{2a^2}\cdot (-20)}}\\
			       &=\underset{\text{mise en évidence}}{\underbrace{2a^2(a^2+a-20}}\\										 &=2a^2\underset{\text{repérage de l'identité}}{\underbrace{(a^2+a-20)}}\\
			       &=2a^2\underset{\text{factorisation}}{\underbrace{(a-4)(a+5)}}
	\end{align*}
	\task 
	\vspace{-2em}
	\begin{align*}
	28m^2-7&=\underset{\text{mise en évidence}}{\underbrace{7(4m^2-1)}}\\
	       &=7\underset{\text{repérage de l'identité}}{\underbrace{((2m)^2-(1)^2)}}\\
	       &=7\underset{\text{factorisation}}{\underbrace{(2m-1)(2m+1)}}
	\end{align*}
\end{tasks}


{\bfseries Factoriser une expression n'est pas simple. Afin de reconnaître les identités remarquables et les groupements possibles il faut s'entraîner. De plus certaines expressions ne peuvent pas être factorisées. Entrainez-vous, entraînez-vous, entraînez-vous\,!}
\begin{rem}
	Il n'est pas toujours possible de factoriser une expression. Tous les polynômes du second degré de la forme $ax^2+bx+c$ avec $\Delta=b^2-4ac<0$ ne peuvent pas être factorisées sur les nombres réels. En particulier, $x^2+1$ ne peut pas être factorisé et tous les polynômes de la forme
	\[a^2+b^2\]
	ne peuvent pas être factorisés. 
\end{rem}
\end{document}
