\documentclass[a4paper,12pt]{report}
\usepackage{../courslatex}
\usepackage{theorie}

\setcounter{chapter}{1}

\renewcommand{\titreChapitre}{Ch. \thechapter\,: Ensembles et intervalles réels}
\begin{document}
\chapter{Ensembles et intervalles réels}
\thispagestyle{fancy}
\section{Notations ensemblistes}
Un ensemble représente une collection d'objets qui sont appelés les éléments de l'ensemble. On note un ensemble avec des accolades. 
Par exemple 
\[\{1,2,3,4,5\} \text{ est l'ensemble qui contient les éléments } 1,2,3,4 \text{ et } 5\]

\begin{notation}	
Soient $A$ et $B$ deux ensembles.
On note 
\begin{itemize}
\item[{\PencilRightDown}] l'ensemble vide, l'ensemble qui ne contient aucun élément, $\{\}=\emptyset$,\; 
 \item[{\PencilRightDown}] $x\in A$ pour $x$ appartient à (est un élément de) $A$\,;
 \item[{\PencilRightDown}] $x\not\in A$ pour $x$ n'appartient pas à (n'est pas un élément de) $A$\,;
	\item[{\PencilRightDown}] $B\subset A$ pour dire que $B$ est inclus dans $A$, c'est-à-dire que tous les éléments de $B$ sont aussi des éléments de $A$. On dit que $B$ est un \emph{sous-ensemble} de $A$.
	\item[{\PencilRightDown}] $B\not \subset A$ pour dire que $B$ n'est pas inclus dans $A$, c'est-à-dire qu'il existe au moins un élément de $B$ qui n'est pas un élément de $A$.  
\end{itemize}
\end{notation}

Voici trois façons de décrire un ensemble.
Par extension, par compréhension ou par une fonction. 

\begin{description}[leftmargin=2cm]
	\item[\emph{Par extension}] On donne tous les éléments de l'ensemble séparés par des points-virgules.
		Les points de suspension sont autorisés lorsque la règle qui définit l'appartenance à l'ensemble est claire. 
		Par exemple 
		\[A=\{1;2;3;4\} \text{ ou } B=\{0;2;4;6;8;\ldots\}\text{ (pour les nombres pairs)}\]
	\item[\emph{Par compréhension}] On donne une ou plusieurs conditions d'appartenance à l'ensemble après avoir précisé où sélectionner les éléments.
		Par exemple
		\[B=\{x\in \N \mid x \text{ est impair }\} \text{ ou } C=\{x\in \R_{+}\mid x\leq 3\} \]
	\item[\emph{Par une fonction}] On fait appel à une fonction $f:D\rightarrow E$ et on définit l'ensemble $H=\{f(x) \mid x\in D\}$. 
		Par exemple
		\[I=\{2x\mid x\in \N\} \text{ ou encore } J=\{x^2\mid x\in \Q\}\]
\end{description}

Les ensembles peuvent paraître abstraits, mais nous avons un outil qui permet de représenter les relations entre deux ou plusieurs ensembles.

\begin{defi}[Diagramme de Venn]
\emph{Un diagramme de Venn} est une représentation d'un ou plusieurs ensembles par des lignes simples fermées dans lesquelles on représente les éléments en fonction de leurs appartenances. 
Chaque élément ne peut occuper qu'une seule position, celle qui correspond à sa caractérisation la plus précise.
\end{defi}

Voici un exemple avec les ensembles de nombres vus au chapitre précédents.
Chaque nombre se situe dans le plus petit ensemble auquel il appartient. 
Nous verrons d'autres exemples dans les séries.
\begin{center}
	\def\natEl{(0,0) ellipse (2cm and 1cm)}
	\def\intEl{(0.5,0) ellipse (3cm and 2cm)}
	\def\ratEl{(1,0) ellipse (4cm and 3cm)}
	\def\realEl{(1.5,0) ellipse (5cm and 4cm)}
\begin{tikzpicture}
        \node (A) at (1.5,0) {$\N$};
        \node (B) at (3,0) {$\Z$};
        \node (C) at (4.5,0) {$\Q$};
        \node (D) at (6,0) {$\R$};
        \node (A) at (0.3,0.2) {$0$};
        \node (A) at (-0.5,-0.2) {$10$};
	\node (A) at (-1.3,0.15) {$\dfrac{4}{2}$};
	\node (A) at (1,-0.5) {$\sqrt{9}$};
        \node (A) at (1.2,-1.3) {$-2$};
	\node (A) at (1.1,1.5) {$\sqrt[3]{-1}$};
        \node (A) at (3,1.9) {$-3,5$};
        \node (A) at (2.3,-2.3) {$2,764$};
	\node (A) at (3.9,1.1) {$\dfrac{1}{5}$};
	\node (A) at (3.4,-1.5) {$9,\overline{51}$};
	\node (A) at (5.2,1.5) {$\sqrt{2}$};
        \node (A) at (5.5,-1) {$\pi$};	
	\draw \natEl;
	\draw \intEl;
	\draw \ratEl;
	\draw \realEl;
\end{tikzpicture}
\end{center}

\section{Opérations sur les ensembles}
On définit trois opérations sur les ensembles.
Soient $A$ et $B$ deux ensembles.
\begin{description}[leftmargin=2cm]
	\item[\emph{L'union de $A$ et de $B$}] notée $A\cup B$ est l'ensemble formé en réunissant les éléments de $A$ et de $B$.
		\[A\cup B=\{x \mid x\in A \text{ ou } x \in B\}\]
\def\firstcircle{(0,0) circle (1cm)}
\def\thirdcircle{(0:1cm) circle (1cm)}
\begin{center}
\begin{tikzpicture}
    \begin{scope}
	\fill[pattern=north west lines, pattern color=blue] \firstcircle;
        \fill[pattern=north west lines, pattern color=blue] \thirdcircle;
        %\fill[red] \firstcircle;
        %\fill[blue] \thirdcircle;
    \end{scope}
        \draw \firstcircle node[left] {$A$};
        \draw \thirdcircle node [right] {$B$};
\end{tikzpicture}
\end{center}
	\item[\emph{L'intersection de $A$ et de $B$}] notée $A\cap B$ est l'ensemble formé de tous les éléments qui appartiennent à $A$ et à $B$.
		\[A\cap B =\{x\mid x\in A \text{ et } x\in B\}\]
\begin{center}
\begin{tikzpicture}
    \begin{scope}
        \clip \firstcircle;
        \fill[pattern=north west lines, pattern color=blue] \thirdcircle;
        %\fill[red] \firstcircle;
        %\fill[blue] \thirdcircle;
    \end{scope}
        \draw \firstcircle node[left] {$A$};
        \draw \thirdcircle node [right] {$B$};
\end{tikzpicture}
\end{center}

	\item[\emph{La différence de $A$ et de $B$}] notée $A\setminus B$ est l'ensemble formé de tous les éléments de $A$ qui n'appartiennent pas à $B$.
		\[A\setminus B =\{x\mid x\in A \text{ et } x\not \in B\}\]
\begin{center}
\begin{tikzpicture}
    \begin{scope}
        \clip \firstcircle;
        \fill[pattern=north west lines, pattern color=blue, even odd rule] \thirdcircle \firstcircle;
        %\fill[red] \firstcircle;
        %\fill[blue] \thirdcircle;
    \end{scope}
        \draw \firstcircle node[left] {$A$};
        \draw \thirdcircle node [right] {$B$};
\end{tikzpicture}
\end{center}
\end{description}

Pour des ensembles de nombres, on introduit les notations suivantes\,:
\begin{notation}
	Soit $A$ un ensemble de nombres, on note\,
	\begin{itemize}
	\item[{\PencilRightDown}]	 $A^*=A\setminus \{0\}$ les éléments de $A$ différents de zéro.
	\item[{\PencilRightDown}] $A_+=A_{\geq 0}=\{a\in A \mid a\geq 0\}$ les éléments de $A$ positifs ou nul\,;
	\item[{\PencilRightDown}] $A_-=A_{\leq 0}=\{a\in A \mid a\leq 0\}$ les éléments de $A$ négatifs ou nul.
	\end{itemize}
\end{notation}

\section{Intervalles réels}

On définit pour des ensembles réels la notion suivante.
\begin{defi}[Intervalles réels]
	Soient $a,b\in \R$.
	\begin{description}[leftmargin=2cm]
		\item[\emph{Intervalle fermé}] L'intervalle $[a;b]$ s'appelle l'intervalle fermé entre $a$ et $b$. 
			\[\interval{a}{b}=\{x\in \R \mid a\leq x \leq b\}\]

	\item[\emph{Intervalle ouvert}] L'intervalle $]a;b[$ s'appelle l'intervalle ouvert entre $a$ et $b$. 
			\[\interval[open]{a}{b}=\{x\in \R \mid a < x < b\}\]

		\item[\emph{Intervalle semi-ouvert à gauche}] L'intervalle $]a;b]$ s'appelle l'intervalle semi-ouvert à gauche entre $a$ et $b$. 
			\[\interval[open left]{a}{b}=\{x\in \R \mid a< x \leq b\}\]

	\item[\emph{Intervalle semi-ouvert à droite}] L'intervalle $[a;b[$ s'appelle l'intervalle semi-ouvert à droite entre $a$ et $b$. 
			\[\interval[open right]{a}{b}=\{x\in \R \mid a\leq x < b\}\]
 \end{description}
 Les nombres $a$ et $b$ s'appelle \emph{les bornes} de l'intervalle.
\end{defi}
Avec ces intervalles, nous introduisons le symbole infini noté $\infty$.
Nous utiliserons ce symbole avec $-\infty$ et $+\infty$ comme indiqué sur la droite réelle ci-dessous. 

On peut utiliser les symboles $-\infty$ et $+\infty$ comme des bornes d'intervalles. 
L'infini n'est pas contenu dans l'intervalle et on utilise la notation semi-ouvert à la borne $-\infty$ ou $+\infty$. 
\[\interval[open right]{a}{+\infty}=\{x\in \R \mid a\leq x\} \quad \text{ et } \quad \interval[open left]{-\infty}{a}=\{x\in \R\mid x\leq a\}\]
Avec ces notations
\[\R=\interval[open]{-\infty}{+\infty} \]
On peut donner une représentation de chacun de ces intervalles sur la droite réelle de la manière suivante. 
Par exemple, l'intervalle $\interval[open left]{-2}{3}$\,:
\begin{center}
	\begin{tikzpicture}
    % Draw the real line
    \draw[thick] (-5,0) -- (5,0); % Adjust the length as needed

    % Add minus infinity symbol on the left
    \node at (-5,0) [below] {$-\infty$};

    % Add plus infinity symbol on the right
    \node at (5,0) [below] {$+\infty$};

    % Add arrowheads to indicate extension to infinity
    \draw[thick,->] (-5,0) -- (-5.5,0);
    \draw[thick,->] (5,0) -- (5.5,0);
    \draw[very thick] (-2,0) -- (3,0);
    \draw[>={Bracket[width=6mm,line width=1pt,length=1.5mm]},->] (-2, 0) -- (3, 0);
    \draw[>={Bracket[width=6mm,line width=1pt,length=1.5mm]},->] (-2.01, 0) -- (-2, 0);
    % Optionally, add some tick marks and labels for real numbers
    \foreach \x in {-2,0,3} {
        \draw (\x,0.1) -- (\x,-0.1); % Tick marks
        \node at (\x, -0.5) {\x}; % Labels for the tick marks
    }
\end{tikzpicture}
\end{center}

Pour terminer ce chapitre, voici un exemple d'opération avec les intervalles réels. 
Déterminons graphiquement l'intervalle $]-2;+\infty [ \cap [-3;1]$. 
On commence par aligner les représentations graphiques des ensemble les unes sous les autres. 
On utilise la définition de l'opération \enquote{intersection} pour déterminer le résultat. 
\begin{center}
	\begin{tikzpicture}
    % Draw the real line
    \draw[thick] (-5,0) -- (5,0); % Adjust the length as needed

    % Add minus infinity symbol on the left
    \node at (-5,0) [below] {$-\infty$};

    % Add plus infinity symbol on the right
    \node at (5,0) [below] {$+\infty$};

    \node at (6,0) [right] {$\interval[open]{-2}{+\infty}$};
    % Add arrowheads to indicate extension to infinity
    \draw[thick,->] (-5,0) -- (-5.5,0);
    \draw[thick,-] (5,0) -- (5.5,0);
    \draw[>={Latex[width=3mm]},->, very thick] (-2,0) -- (5.5,0); 
    \draw[>={Bracket[width=6mm,line width=1pt,length=1.5mm]},->, thick] (-2.01, 0) -- (-2, 0);
    % Optionally, add some tick marks and labels for real numbers
    \foreach \x in {-2} {
        \draw (\x,0.1) -- (\x,-0.1); % Tick marks
        \node at (\x, -0.5) {\x}; % Labels for the tick marks
    }
  \begin{scope}[shift={(0,-1.5)}]
    \draw[thick] (-5,0) -- (5,0); % Adjust the length as needed

    % Add minus infinity symbol on the left
    \node at (-5,0) [below] {$-\infty$};

    % Add plus infinity symbol on the right
    \node at (5,0) [below] {$+\infty$};

    \node at (6,0) [right] {$\interval{-3}{1}$};

    % Add arrowheads to indicate extension to infinity
    \draw[thick,->] (-5,0) -- (-5.5,0);
    \draw[thick,->] (5,0) -- (5.5,0);
    \draw[>={Bracket[width=6mm,line width=1pt,length=1.5mm]},<->, very thick] (-3, 0) -- (1, 0);
    % Optionally, add some tick marks and labels for real numbers
    \foreach \x in {-3,1} {
        \draw (\x,0.1) -- (\x,-0.1); % Tick marks
        \node at (\x, -0.5) {\x}; % Labels for the tick marks
    }
  \end{scope}
  \begin{scope}[shift={(0,-3)}]
    \draw[thick] (-5,0) -- (5,0); % Adjust the length as needed

    % Add minus infinity symbol on the left
    \node at (-5,0) [below] {$-\infty$};

    % Add plus infinity symbol on the right
    \node at (5,0) [below] {$+\infty$};

    \node at (6,0) [right] {$\interval[open]{-2}{+\infty}\cap \interval{-3}{1}=\interval[open left]{-2}{1}$};
    % Add arrowheads to indicate extension to infinity
    \draw[thick,->] (-5,0) -- (-5.5,0);
    \draw[thick,->] (5,0) -- (5.5,0);

    \draw[>={Bracket[width=6mm,line width=1pt,length=1.5mm]},->, thick] (-2.01, 0) -- (-2, 0);
    \draw[>={Bracket[width=6mm,line width=1pt,length=1.5mm]},->, very thick] (-2, 0) -- (1, 0);
    % Optionally, add some tick marks and labels for real numbers
    \foreach \x in {-2,1} {
        \draw (\x,0.1) -- (\x,-0.1); % Tick marks
        \node at (\x, -0.5) {\x}; % Labels for the tick marks
    }
  \end{scope}
  \draw[thick, dashed] (-2,0) -- (-2,-3);
  \draw[thick, dashed] (1,-1.5) -- (1,-3);
\end{tikzpicture}
\end{center}
\newpage
\vspace{1cm}
\textLigne{Notes personnelles ou complément}
\end{document}
