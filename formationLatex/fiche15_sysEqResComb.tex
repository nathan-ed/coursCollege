\documentclass[a4paper,12pt]{report}
\usepackage[fiche,niveau={1MA1},nfiche={12}, annee={Emilie-Gourd, 2024--2025}, auteur={ns},theme={Équations irrationnelles}]{packages/bandeaux}
\usepackage{packages/boites}
\usepackage{packages/fontes}
\usepackage{packages/courslatex}

\begin{document}
Remarquons que l'équation basique $x=-1$ a une solution évidente $-1$. Lorsque l'on élève l'équation au carré, on obtient $x^2=1$ qui a deux solutions, les nombres $-1$ et $1$. 

\begin{boiteicone}
Lorsqu'on élève une équation au carré, de nouvelles solutions apparaissent qui ne sont pas forcément des solutions de l'équation initiale.	
\end{boiteicone}

Ainsi, l'opération \og{} élever au carré \fg{} n'est pas une opération d'équivalence. On n'écrit pas \og{}$\iff$\fg{}, mais \og{}$\implies$\fg{}, car la nouvelle équation n'a pas le même ensemble de solutions.

\[x=-1 \implies \quad x^2=1 \quad \text{( et non pas } \iff )\]

L'opération \og{} élever à une puissance \fg{} est utile pour résoudre des équations qui contiennent des inconnues sous des radicaux ($\sqrt[n]{\phantom{test}}$).

\medskip
\centering
\fbox{
\begin{tabular}{cc}
    Inconnue sous un radical & Pas d'inconnue sous un radical \\
    $\sqrt{x+1}=x+2$ & $\sqrt{3}x=1+x$
\end{tabular}
}

\raggedright
\medskip

Une équation du premier degré avec une inconnue sous un radical peut avoir une infinité de solution, deux, une ou aucune solution. {\bfseries Toutes les solutions de l'équation initiale sont des solutions de l'équation élevée au carré, mais toutes les solutions de l'équations élevée au carré ne sont pas des solutions de l'équations initiales.} Il faut vérifier chaque solution.
\[1=-x+\sqrt{4 x + 16 }\]

On résout ce type d'équation de la manière suivante:
\begin{align*}
&\phantom{\iff} 1=-x+\sqrt{4 x + 16 }&&\text{isoler la racine carrée}\\
&\iff x + 1=\sqrt{4 x + 16 }&&\text{élever au carré}\\&\implies \left(x + 1\right)^2=4 x + 16 &&\text{développer}\\&\iff x^2 + 2 x + 1=4 x + 16 && \text{comparer à zéro}\\&\iff x^2-2 x-15 =0&& \text{résoudre l'équation (ici par factorisation)}\\&\iff \left (x + 3 \right)\left (x-5 \right)=0&& \text{appliquer le théorème du produit nul}\\&\iff x + 3 =0 \quad\text{ou} \quad x-5 =0&& \text{résoudre les équations}\\&\iff x=-3\quad \text{ou} \quad x=5
\end{align*}

On vérifie à présent les solutions obtenues.
      \\
      Pour $x=-3$
      \begin{align*}
      1 &\stackrel{?}{=} -\left(-3\right)+\sqrt{4\cdot \left(-3\right) + 16 } \\
      1  &\neq 5
      \end{align*}
      donc $-3$ n'est pas solution de l'équation.\\
      Pour $x=5$
      \begin{align*}
      1 &\stackrel{?}{=} -\left(5\right)+\sqrt{4\cdot \left(5\right) + 16 } \\
      1 &= 1      
\end{align*}

      donc $5$ est solution de l'équation.\\
      Ainsi, l'ensemble des solutions de l'équation est $S=\{5\}$.

      \newpage

      \begin{boiteExT}[$x=6+\sqrt{-7x+30}$]
\vspace{20cm}
\end{boiteExT}

Exercice(s) correspondant(s)~: série 12, exercice 1. 
\end{document}


