\documentclass[a4paper,12pt]{report}
\usepackage[fiche,niveau={1MA1},nfiche={1}, annee={Emilie-Gourd, 2024--2025}, auteur={ns},theme={Factorisation -- Enchaîner les techniques}]{packages/bandeaux}
\usepackage{packages/boites}
\usepackage{packages/fontes}
\usepackage{packages/courslatex}

\begin{document}
On considère l'identité suivante : $a x^2+b x+c=a\left(x-x_1\right)\left(x-x_2\right)$, dans laquelle $x_1=\frac{-b+\sqrt{\Delta}}{2 a}, \quad x_2=\frac{-b-\sqrt{\Delta}}{2 a} \quad\left(\right.$ avec $\left.\Delta=b^2-4 a c \geq 0\right)$.
a) $\left(^*\right)$ Démontrer cette identité en utilisant la complétion du carré.
b) Mémoriser cette identité.
c) Utiliser cette identité pour factoriser les polynômes suivants :
(1) $3 x^2+24 x+48$
(3) $6 x^2+7 x-20$
(2) $2 x^2-6 x+2$
(4) $-x^2+4 x-2$



\end{document}


