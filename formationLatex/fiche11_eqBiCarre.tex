\documentclass[a4paper,12pt]{report}
\usepackage[fiche,niveau={1MA1},nfiche={11}, annee={Emilie-Gourd, 2024--2025}, auteur={ns},theme={Équations bicarrées}]{packages/bandeaux}
\usepackage{packages/boites}
\usepackage{packages/fontes}
\usepackage{packages/courslatex}

\begin{document}
Jusqu'à présent, nous avons vu 
\begin{boiteExT}
\vspace{5cm}
\end{boiteExT}
Nous allons étudier comment résoudre une équation de la forme

\[ax^4+bx^2+c=0\]

On commence par remarquer que 
\[ax^4+bx^2+c=0 \iff a(x^2)^2+bx^2+c=0 \iff ay^2+by+c=0 \text{ avec } y=x^2\]

On applique la formule du deuxième degré à l'équation en $y$. L'équation aura 
\begin{tasks}(3)
	\task[] $0$ solution réelle;
	\task[] $1$ solution réelle;
	\task[] $2$ solutions réelle.
\end{tasks}
On détermine ensuite les solutions de l'équation en déterminant les valeurs de $x=\pm\sqrt{y}$ pour $y\geq 0$.
\medskip

Seulement les solutions positives pour l'équation en $y$ donneront des solutions pour l'équation en $x$, car la racine carrée n'est pas définie sur les nombres négatifs.

\begin{boiteExT}[$x^4-20x^2+91=0$]
\vspace{10cm}
\end{boiteExT}
\newpage 
\begin{boiteExT}[$x^4+2x^2-1=0$]
\vspace{15cm}
\end{boiteExT}

\begin{exo}
	Résoudre à l'aide de la méthode présentée les équations suivantes (calculatrice autorisée). 
\begin{tasks}(2)
	\task $9 x^4-37 x^2+4=0$
	\task $4 x^4+91 x^2-225=0$
	\task $16 x^4+40 x^2+9=0$
	\task $x^4-2 x^2-6=0$
\end{tasks}
\end{exo}

\end{document}


