\documentclass[a4paper,12pt]{report}
\usepackage[serie,niveau={1MA1},nfiche={}, annee={Emilie-Gourd, 2024--2025}, auteur={ns},theme={Série 13}]{packages/bandeaux}
\usepackage{packages/boites}
\usepackage{packages/fontes}
\usepackage{packages/courslatex}

\begin{document}
\begin{exo}[1]
Le couple $\left(3 ; \dfrac{3}{2}\right)$ est-il solution du système $  \left\{
    \begin{aligned}
      & 7x - 12y = 3 \\
      & -5x + 8y = 31
    \end{aligned}
  \right.$ ?	
\end{exo}

\begin{exo}[1]
	Donner un système de deux équations à deux inconnues dont l'ensemble des solutions est \[S=\{(-2 ; 3)\}.\]
\end{exo}
\begin{exo}[2]
	Résoudre les systèmes d'équations suivants avec la méthode de votre choix. 
	\begin{tasks}(3)
		
\task
$
  \left\{
    \begin{aligned}
      & x - 2y = -5 \\
      & 7x + 10y = 1
    \end{aligned}
  \right.
$

\task
$
  \left\{
    \begin{aligned}
      & 5x + 5y = 5 \\
      & 3x - 7y = -2
    \end{aligned}
  \right.
$

\task
$
  \left\{
    \begin{aligned}
      & 5x + 6y = -2 \\
      & 10x + 3y = -7
    \end{aligned}
  \right.
$

\task
$
  \left\{
    \begin{aligned}
      & 5x + 4y = 13 \\
      & 2x - 7y = 31
    \end{aligned}
    \right.$

\task
$
  \left\{
    \begin{aligned}
      & \dfrac{x+5}{2} - \dfrac{3 - y}{5} = \dfrac{5}{2} \\
      & x + 7 + \dfrac{y - 6}{4} = 7 \cdot \dfrac{5}{2}
    \end{aligned}
  \right.
$

\task
$
  \left\{
    \begin{aligned}
      & \dfrac{x-3}{2} - \dfrac{5}{2} = \dfrac{2y - 21}{2} + 1 \\
      & \dfrac{x + 2}{3} + 3 = \dfrac{3 - y}{5} - \dfrac{10}{3}
    \end{aligned}
  \right.
$
\end{tasks}
\end{exo}

%\begin{exo}[]{
%\begin{minipage}{0.60\linewidth}
%Ce paquet a la forme d'un prisme droit à base carrée. 
%Pour le ficeler, selon la manière $A$, il faut une ficelle de $220\un{cm}$, alors que $180 \un{cm}$ suffisent pour le ficeler selon la manière $B$. 
%À chaque fois, on compte $20\un{cm}$ pour le noeud. 
%Trouve les dimensions de ce paquet.
%\end{minipage}
%\begin{minipage}{0.25\linewidth}
%	\begin{center}
%	\includegraphics[scale=1]{"../media/exos11e_paquet"}
%	\end{center}
%\end{minipage}
%}
%\end{exo}

\begin{exo}[1]
	Traduire chacune des ces situations par un système de deux équations et déterminer les solutions.
\begin{enumerate}
	\item La somme de deux nombrres est $100$. La différence de ces deux nombres est $68$. Quels sont ces nombres?
	\item Entendu de bon matin à la terrasse d'un café:
		\begin{itemize}
			\item "Deux chocolats et trois croissants: Fr. $8,90$."
			\item "Trois chocolats et cinq croissants: Fr. $13,80$."
			\end{itemize}
			Quel est le prix d'un chocolat? Et celui d'un croissant?
		\item 350 spectateurs ont assisté à un spectacle. 
			Au parterre, la place revient à Fr. $20.-$; à la galerie, elle revient à Fr. $30.-$.

			Le montant de la recette des entrées est de Fr. $7850.-$.

			Combien y avait-il de spectateurs au parterre? Et à la galerie?
\end{enumerate}
\end{exo}
\begin{exo}[1]
		Un groupe de vingt-quatre personnes fait un stage de deux jours dans une école de voile. Deux activités sont au programme: la planche à voile ou le catamaran. Le premier jour, dix personnes choisissent la planche à voile et les autres le catamaran. La facture totale de ce premier jour s'élève à $560$ francs. Le deuxième jour, ils sont douze à choisir la planche à voile et les autres font du catamaran. La facture du deuxième jour s'élèves à $540$ francs. 

	Quel est le prix par personne d'une journée de planche à voile et celui d'une journée de catamaran?
\end{exo}
\begin{exo}[2]
Un confiseur répartit des truffes dans des cornets de $200$ g. S'il avait réparti ses truffes dans des cornets de $150$ g, il y aurait eu $12$ cornets de plus. 

Quelle quantité de truffes a-t-il préparée?

\end{exo}

\begin{exo}[2]
Céline regarde avec envie un pull et une robe présentés dans la vitrine d’une boutique.
Malheureusement, le prix total de ces deux vêtements est de $137.50$ francs et dépasse
son budget. Quelques temps après, le prix du pull baisse de $20\%$ et celui de la robe de $30\%$. Céline calcule rapidement la dépense totale et constate que le prix total a baissé de
$35$ francs, ce qui lui permet d’acheter ces deux vêtements.
Quels étaient les prix du pull et de la robe avant la baisse?
\end{exo}
\begin{exo}[1]
Pour organiser une sortie de fin d’année, un collège loue des cars. Il y a des grands cars de
56 places et des petits cars de 44 places. Il y a quatre grands cars de plus que de petits.
624 élèves participent à la sortie et tous les cars sont remplis.
Combien le collège a-t-il loué de cars de chaque catégorie?
\end{exo}
\begin{exo}[1]
	Un terrain rectangulaire a un périmètre de $150$ m. Si l’on augmente sa largeur de $5$ m et
si l’on diminue sa longueur de $3$ m, alors son aire augmente de $120$ $\text{m}^2$.
Quelles sont les dimensions de ce rectangle?
\end{exo}

\end{document}


