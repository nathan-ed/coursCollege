\documentclass[a4paper,12pt]{report}
\usepackage[fiche,niveau={1MA1},nfiche={12}, annee={Emilie-Gourd, 2024--2025}, auteur={ns},theme={Résolution par substitution et combinaison linéaire}]{packages/bandeaux}
\usepackage{packages/boites}
\usepackage{packages/fontes}
\usepackage{packages/courslatex}

\begin{document}
\section*{Par substitution}
Pour résoudre un système d'équations par subsitution, on {\bfseries isole une inconnue dans une équation du système} puis on {\bfseries substitue l'expression obtenue dans une autre équation du système} afin \og{}d'éliminer\fg{} l'inconnue en question de l'équation. 
Afin d'appliquer cette méthode, on n'utilise pas {\bfseries PE3}.
\begin{boiteExT}[$\systeme{-3x+y=9,4x-3y=-17}$]
\vspace{15cm}
\end{boiteExT}
\newpage
\section*{Par combinaison linéaire}
Pour résoudre un système d'équations par combinaison linéaire, on applique {\bfseries PE3} pour éliminer une inconnue dans une équation du système.  
\begin{boiteExT}[$\systeme{5x-4y=8,2x+5y=1}$]
\vspace{15cm}
\end{boiteExT}
\begin{boiteicone}
	On réécrit toutes les équations à chaque étape.
\end{boiteicone}

\end{document}


