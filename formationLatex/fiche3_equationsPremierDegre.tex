\documentclass[a4paper,12pt]{report}
\usepackage[fiche,niveau={1MA1},nfiche={3}, annee={Emilie-Gourd, 2024--2025}, auteur={ns},theme={Équations du premier degré à une inconnue}]{packages/bandeaux}
\usepackage{packages/boites}
\usepackage{packages/fontes}
\usepackage{packages/courslatex}
\usepackage{packages/mathenv}

\begin{document}

Une équation du premier degré à une inconnue est de la forme $ax+b=cx+d$ avec $a,b,c,d\in \R$.
\begin{boiteExT}[Résolution d'une équation du premier degré]
\vspace{7cm}

\end{boiteExT}
Ce type d'équation peut avoir une unique solution, aucune solution ou une infinité de solutions~:
\begin{itemize}
	\item[] si $a\neq c$ alors l'équation admet une unique solution;
	\item[] si $a=c$ est $b\neq d$ alors $S=\emptyset$;
	\item[] si $a=c$ et $b=d$ alors $S=\R$.
\end{itemize}
\begin{boiteExT}[Les trois types d'ensemble de solutions]
\vspace{10cm}

\end{boiteExT}
\newpage
On résout, comme exemple, quelques équations.
\begin{boiteExT}[Équation avec des fractions]
\vspace{10cm}

\end{boiteExT}
\begin{boiteExT}[Équation avec de la distributivité]
\vspace{10cm}

\end{boiteExT}
\end{document}
