\documentclass[a4paper,12pt]{report}
\usepackage[serie,niveau={1MA1},nfiche={}, annee={Emilie-Gourd, 2024--2025}, auteur={ns},theme={Série 12}]{packages/bandeaux}
\usepackage{packages/boites}
\usepackage{packages/fontes}
\usepackage{packages/courslatex}

\begin{document}

\begin{exo}[3]
Résoudre les équations suivantes dans $\mathbb{R}$
\begin{tasks}(2)
\task $\sqrt{(x-1)(3 x-6)}=x-2$
\task $\sqrt{2 x+7}=\sqrt{x}+2$
\task $4 x-1=\sqrt{7 x^2-2 x+8}$
\task $\sqrt{x+8}-\sqrt{x+3}=5 \sqrt{x}$
\task $\sqrt{x+8}+\sqrt{x+3}=5 \sqrt{x}$
\task $\sqrt{7 x-27}=\sqrt{2 x+1}+\sqrt{3 x+4}$
\end{tasks}
\end{exo}

\begin{exo}[2]
Résoudre dans $\mathbb{R}$ l'équation suivante : $x^6+4 x^3-32=0$, de deux façons :
\begin{tasks}
\task par un changement de variable approprié;
\task par factorisation directe (identités remarquables).
\end{tasks}
\end{exo}

\textLigne{Entraînement individuel}

\begin{exo}
	Résoudre les équations suivantes dans $\mathbb{R}$\,:
	\begin{tasks}(2)
\task $x^2-10 x+16=0$
\task $7 x^3+9=3 x^2+21 x$
\task $(x-4)(x+5)-2 x(x+5)=0$
\task $x^2=8 x$
\task $(x+1)(x+2)=(x+2)(x+3)$
\task $(x-8)(4 x-3)+x^2-8 x=0$
\task $(2 x+3)^2=8-x(2-3 x)$
\task $(x-3)^2-2 x=3 x^2-1$
\task $-(-1-4 x)^2=1-(5 x-1)^2$
\task $4 x^2+8 x+1=6$
	\end{tasks}
\end{exo}
\begin{exo}
Résoudre les équations dans $\mathbb{R}$.
\begin{tasks}(2)	
\task $x^2-5=8(2 x+6)-(x-5)^2$
\task $x^3+2 x^2=3 x+6$
\task $x^3+9 x^2-2 x-18=0$
\task $\left(x^2-2 x\right)^2-1=0$
\end{tasks}
\end{exo}
\end{document}


