\newboolean{themeBool} % Boolean to indicate if solutions should be shown
\newboolean{uuid} % Boolean to indicate if UUID should be shown
\newboolean{link} % Boolean to indicate if a link should be shown
\newcounter{num} % Counter to keep track of exercise numbers

% liens vers des ressources extérieures :
\newcommand{\pathcorrige}{https://raw.githubusercontent.com/smaxx73/Exercices/main/pdf/pdf/} % Path to corrected exercises
\newcommand{\pathnotebook}{https://github.com/smaxx73/Exercices/blob/main/notebook/} % Path to notebooks
\newcommand{\pathexercice}{https://raw.githubusercontent.com/smaxx73/Exercices/main/pdf/pdf_sansol/} % Path to exercises without solutions

% Define the title of the exercise
\newcommand{\titre}[1]{%
	\def\TitreExo{#1} % Set the title for later use
}
% Define the title of the exercise
\newcommand{\piments}[1]{%
	\def\Piments{#1} % Set the title for later use
}



% Define the content of the exercise
\newcommand{\contenu}[1]{%
	\def\Contenu{#1} % Set the content for later use
}

% Define the number of points for the exercise
\newcommand{\pts}[1]{%
	\def\Points{#1} % Set the number of points for later use
}

% Define the year of the exercise
\newcommand{\annee}[1]{
	\def\Annee{#1} % Set the year for later use
}

% Define the correction (solution) of the exercise
\newcommand{\correction}[1]{%
	\def\Solution{#1} % Set the solution for later use
}

% Directly use code as it is
\newcommand{\code}[1]{#1} % Simple passthrough command for code

% Directly use text as it is
\newcommand{\texte}[1]{#1} % Simple passthrough command for text

% Define the theme (currently unused)
\newcommand{\theme}[1]{\def\Theme{#1}} % Placeholder for theme

% Define the author (currently unused)
\newcommand{\auteur}[1]{} % Placeholder for author

% Define the organization (currently unused)
\newcommand{\organisation}[1]{} % Placeholder for organization

% Command for question text
\newcommand{\question}[1]{#1} % Simple passthrough command for question
% Define the year of the exercise
\newcommand{\niveau}[1]{
	\def\Niveau{#1} % Set the year for later use
}

\newcommand{\source}[1]{
	\def\Source{#1} % Set the year for later use
}

\newcommand{\type}[1]{
	\def\Type{#1} % Set the year for later use
}

% Command for providing a response, shown only if the solution boolean is true
\newcommand{\reponse}[1]{%
	\ifthenelse{\boolean{solution}} % Check if 'solution' is true
	{%
		\begin{solutionbox} % Begin solution box environment
			\begin{footnotesize} #1 \end{footnotesize} % Display the response in small font size
		\end{solutionbox}}{} % End solution box if 'solution' is true
} 

% Move to the next exercise by incrementing the counter
\newcommand{\nextexo}{%
	\addtocounter{num}{1} % Increment exercise counter
	\vspace{1em} % Add vertical space
}

% Command to insert an exercise, taking several arguments including content and whether to show UUID or link
\newcommand{\insertexo}[5]{% contenu, solution, uuid, lien solution, numerotation, theme
	\input{\path/src_eval/#1} % Input the exercise content from the provided path
	\setboolean{themeBool}{#2} % Set the theme boolean
	\setboolean{uuid}{#3} % Set the uuid boolean
	\setboolean{link}{#4} % Set the link boolean
	
	\begin{question}[\Points] % Begin question environment with the number of points
	\ifthenelse{\boolean{themeBool}} % Check if themeBool is true
		{\Theme} % Include theme if true
		{} % Else, do nothing
	\ifthenelse{\boolean{uuid}} % Check if uuid is true
		{#1\\} % Include uuid if true
		{} % Else, do nothing
	\Contenu % Insert the content of the exercise
	\end{question}	
	\begin{solution} % Begin solution environment
	\Solution % Insert the solution
	\end{solution}
}


% Command to list exercises from a given list
\newcommand{\listexo}[1]{% liste of exercises
	\foreach \ex in #1 { % Iterate over each exercise in the list
		\nextexo % Move to the next exercise
		\insertexo{\ex}{\themeBool}{\uuid}{\link}{\thenum} % Insert the current exercise with its properties
	}
}

% Counter for QR codes
\newcounter{qrcodenum} % Define a counter to track QR codes

% Command to create a list of QR codes for exercises
\newcommand{\listeqrcode}[2]{% liste d'exercices, initialisation du compteur
\setcounter{qrcodenum}{#2} % Set the initial value for the QR code counter
\noindent
\foreach \exo in #1{ % Iterate over each exercise in the list
	% Create a minipage for each exercise with its number, title, and QR code
	\begin{minipage}{0.24\textwidth}  % Adjust the width to have 4 QR codes per line
		\centering
		\href{\pathcorrige\exo.pdf}{\texttt{Ex \theqrcodenum : \exo \vspace{1em}}}  % Display the number and title of the exercise with a hyperlink
		{\qrcode{\pathcorrige\exo.pdf}}  % Generate the QR code for the exercise
		\par\vspace{2em}  % Add vertical space between QR codes
	\end{minipage}
	% Increment the QR code counter
	\stepcounter{qrcodenum}
}
}
