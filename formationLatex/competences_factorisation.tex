\documentclass[a4paper,12pt]{report}
\usepackage[competences,niveau={1MA1},nfiche={}, annee={Emilie-Gourd, 2024--2025}, auteur={ns},theme={Compétences pour l'évaluation du 11.11.2024}]{packages/bandeaux}
\usepackage{packages/boites}
\usepackage{packages/fontes}
\usepackage{packages/courslatex}
\setlist[itemize]{align=left,labelsep=1em,leftmargin=*,itemsep=5pt,parsep=5pt,topsep=5pt,rightmargin=0cm}

\begin{document}
Je suis capable de

{\bfseries Factorisation}
\begin{itemize}
	\item Enchaîner les différentes techniques de factorisation (mise en évidence, identités remarquables et mise en groupement) pour factoriser au maximum une expression.
\end{itemize}
{\bfseries Équations du premier degré à une inconnue}
\begin{itemize}
	\item Résoudre une équation du premier degré à une inconnue avec des coefficients réels (en particulier des nombres irrationnels et des fractions);
	\item Réciter les définitions et les principes liés aux équations (équation, solution, équivalente, deux principes d'équivalence);
	\item Déterminer le type de l'ensemble de solution d'une équation du premier dégré.
	\item Résoudre des problèmes en identifiant l'inconnue, en posant puis en résolvant une équation;
	\item Résoudre une équation avec un paramètre.
\end{itemize}
{\bfseries Équations du second degré à une inconnue}
\begin{itemize}
	\item Réciter et appliquer le théorème du produit nul;
	\item Résoudre une équation du second degré en factorisant, puis en appliquant le théorème du produit nul;
	\item Résoudre une équation en appliquant la méthode de complétion du carré;
	\item Résoudre un problème nécessitant de poser une équation du second degré.
\end{itemize}
{\bfseries Équations de degré supérieur à deux avec une inconnue}
\begin{itemize}
	\item Expliquer le lien entre la factorisation et la résolution d'une équation (théorème du produit nul);
	\item Résoudre une équation du degré supérieur à deux en factorisant une expression puis en appliquant les méthodes de résolution du premier et second degré. 
\end{itemize}

En particulier, les exercices des séries 6, 7, 8, 9 et 10 ainsi que les fiches 1 à 7 sont au champ de l'évaluation.
\end{document}
