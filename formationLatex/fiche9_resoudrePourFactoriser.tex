\documentclass[a4paper,12pt]{report}
\usepackage[fiche,niveau={1MA1},nfiche={9}, annee={Emilie-Gourd, 2024--2025}, auteur={ns},theme={Lien entre les racines d'un polynome et sa forme factorisée}]{packages/bandeaux}
\usepackage{packages/boites}
\usepackage{packages/fontes}
\usepackage{packages/courslatex}

\begin{document}
Le théorème du produit nul nous a permis d'expliciter les liens entre les solutions d'une équation et sa forme factorisée.
Jusqu'à maintenant, nous avons utilisé la forme factorisée d'une équation pour déduire les solutions de cette équation.
Nous allons maintenant passer par la résolution d'une équation pour déterminer la forme factorisée d'un trinôme. 
\begin{boiteExT}[Rappels]
	\vspace{8cm}

\end{boiteExT}
Connaître les racines d'un trinôme n'est pas suffisant pour le factoriser.
\begin{boiteExT}[Deux trinômes différents avec des mêmes racines]
	\vspace{4.5cm}

\end{boiteExT}
On doit également fixer un des coefficients pour assurer l'unicité de l'expression.

\begin{boiteExT}[Déterminer l'expression réduite du polynôme ayant $r_1$ et $r_2$ pour racines et dont le coefficient dominant vaut $4$]
	\vspace{5cm}	
\end{boiteExT}
Si besoin, on commence par déterminer les racines $(r_1, r_2, \ldots, r_n)$ d'un polynôme puis on multiplie le produit $(x-r_1)\cdots (x-r_n)$ par le coefficient nécessaire pour obtenir le polynôme souhaité. 
\begin{boiteExT}[Factoriser $2x^2-3x+1$]
	\vspace{8cm}
\end{boiteExT}
\begin{exo}
	Déterminer tous les polynômes de degré 2 ayant $-3$ et $4$ comme racines.
\end{exo}
\begin{exo}
	Déterminer tous les polynômes de degré 2 ayant comme racines $1+\sqrt{3}$ et $1-\sqrt{3}$.
\end{exo}
\begin{exo}
	Déterminer tous les polynômes de degré 3 ayant
\begin{tasks}
	\task comme racines $0$ et $-1$.
	\task comme racines $0$ et $-1$, mais aucun autre nombre.
\end{tasks}
\end{exo}
\end{document}


