\documentclass[a4paper,12pt]{report}
\usepackage[fiche,niveau={1MA1},nfiche={8}, annee={Emilie-Gourd, 2024--2025}, auteur={ns},theme={Formule du deuxième degré}]{packages/bandeaux}
\usepackage{packages/boites}
\usepackage{packages/fontes}
\usepackage{packages/courslatex}

\begin{document}
On a démontré la formule suivante pour résoudre une équation du second degré.

\begin{boiteExT}
	\vspace{16cm}
\end{boiteExT}
\begin{boiteExT}[$2x^2-x+3=0$]
	\vspace{5cm}
\end{boiteExT}
\begin{boiteExT}[$2x^2-x-6=0$]
	\vspace{5cm}
\end{boiteExT}
\begin{boiteExT}[$2x^2-x+3=3x^2+x+2$]
	\vspace{7cm}
\end{boiteExT}

Résoudre les équations dans $\mathbb{R}$, à l'aide de la formule du deuxième degré~:
\begin{tasks}
\task $3 x^2-4 x-2=0$
\task $6-3 x+x^2=2+3x$
\task $2 x^2-x+3=2$
\end{tasks}
\end{document}


