\documentclass[a4paper,12pt]{report}
\usepackage[serie,niveau={1MA1},nfiche={}, annee={Emilie-Gourd, 2024--2025}, auteur={ns},theme={Série 14}]{packages/bandeaux}
\usepackage{packages/boites}
\usepackage{packages/fontes}
\usepackage{packages/courslatex}

\begin{document}
\begin{exo}[2]
Résoudre les systèmes d'équations suivants dans $\mathbb{R}^2$ et $\mathbb{R}^3$.
\begin{tasks}(2)
\task
  $\left\{
    \begin{aligned}
      & \dfrac{x}{3} = \dfrac{y}{2} \\
      & 12x + 3y + 14 = 0
    \end{aligned}
  \right.$
\task
$
  \left\{
    \begin{aligned}
      & 2x - y + 3z = 4 \\
      & 3x + y - 4z = 7 \\
      & x + 2y - 5z = 1
    \end{aligned}
  \right.
$
\end{tasks}
\end{exo}
\begin{exo}[3]
Résoudre les systèmes d'équations suivants dans $\mathbb{R}^2$ et $\mathbb{R}^3$.
\begin{tasks}(3)
\task
$
  \left\{
    \begin{aligned}
      & \dfrac{x+y}{2} = \dfrac{7x - 5y}{6} + \dfrac{x+4}{4} \\
      & \dfrac{x - 6y}{2} = \dfrac{x - 2y}{7} + 4
    \end{aligned}
  \right.
$

\task
$
  \left\{
    \begin{aligned}
      & \dfrac{x-3}{y-5} = \dfrac{4}{3} \\
      & \dfrac{x+5}{y+2} = \dfrac{6}{5}
    \end{aligned}
  \right.
$

\task
$
  \left\{
    \begin{aligned}
      & 2x - y = x - 3y - 2 \\
      & 5 - x + \dfrac{3}{2}(x + y) = x + 2y + \dfrac{13}{2}
    \end{aligned}
  \right.
$

\task
$
  \left\{
    \begin{aligned}
      & 4x + 3y + 6z = 41 \\
      & 8x + 5y = 31 \\
      & 7y = 21
    \end{aligned}
  \right.
$

\task
$
  \left\{
    \begin{aligned}
      & 6x + 4y + 8z = 6 \\
      & 3x + y - 2z = 1 \\
      & 3x + 2y - 4z = 1
    \end{aligned}
  \right.
$

\task
$
  \left\{
    \begin{aligned}
      & x - y - z = 6 \\
      & x - 2y - 3z = 10 \\
      & 5x + 6y + z = 2
    \end{aligned}
  \right.
$

\end{tasks}
\end{exo}
\begin{exo}[2]
Ecrire un système d'équations permettant de résoudre chacun des problèmes.
\begin{tasks}
	\task Un nombre de trois chiffres est tel que le produit de ses chiffres divisé par leur somme donne 32 tiers; le nombre lui-même divisé par la même somme donne 48; enfin, le chiffre des dizaines dépasse celui des unités d'autant qu'il est dépassé par celui des centaines. Quel est ce nombre ?
	\task Si d'un nombre de quatre chiffres on soustrait le nombre qu'on obtient en écrivant les chiffres dans l'ordre inverse, on trouve 4725. Le produit des chiffres est 672, le produit des chiffres du milieu 28 et le chiffre des milliers est supérieur de 5 à celui des unités. Quel est ce nombre?
\end{tasks}
\end{exo}
\begin{exo}[1]
On a payé une somme globale de $29'280$ francs pour l'achat des trois séries de meubles suivantes:
\begin{itemize}
	\item $20$ canapés, copie Directoire;
	\item $18$ fauteuils, copie Louis XV;
	\item $16$ chaises, copie Empire.
\end{itemize}
Sachant que 13 canapés valent autant que 21 fauteuils et que 3 fauteuils ont la même valeur que 8 chaises, on demande les prix d'un canapé d'un fauteuil et d'une chaise.
\end{exo}
\begin{exo}[2]
Trois frères, Albrecht, Blaise et Carl ont acheté une maison 2 millions de francs. Albrecht dit qu'il pourrait payer la somme entière si Blaise lui donnait les cinq huitièmes de ce qu'il a. Blaise dit qu'il payerait tout si Carl lui donnait les huit neuvièmes de ce qu'il a. Enfin Carl dit que pour acquitter seul le prix, il lui manque le tiers de ce qu'a Albrecht plus les trois seizièmes de ce que possède Blaise. Combien chacun a-t-il ?
\end{exo}
\end{document}


