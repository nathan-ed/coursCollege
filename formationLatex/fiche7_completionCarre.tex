\documentclass[a4paper,12pt]{report}
\usepackage[fiche,niveau={1MA1},nfiche={7}, annee={Emilie-Gourd, 2024--2025}, auteur={ns},theme={Equations -- Complétion du carré}]{packages/bandeaux}
\usepackage{packages/boites}
\usepackage{packages/fontes}
\usepackage{packages/courslatex}

\begin{document}
La complétion du carré est une technique qui permet de résoudre une équation du second en se ramenant aux techniques de factorisation connues.
\begin{boiteExT}[$x^2-8x+4=0$]
	\vspace{10cm}	

\end{boiteExT}

\begin{boiteExT}[$2x^2+3x-2=0$]
	\vspace{10cm}	

\end{boiteExT}
\newpage
Voici les étapes à suivre pour l'appliquer sur l'équation générale 
\[ax^2+bx+c=0.\]
\begin{boiteExT}
	\vspace{20cm}
\end{boiteExT}
En suivant ces étapes, on remarque que l'on vient de démontrer la formule de résolution d'équation du second degré. On appellera cette formule dans le cours la \emph{formule du deuxième degré}.
Résoudre les équations suivantes à l'aide de la méthode de complétion du carré.
\begin{tasks}(2)
 \task $3 x^2+24 x+48=0$
 \task $6 x^2+7 x-20=0$
 \task $2 x^2-6 x+2=0$
 \task $-x^2+4 x-2=0$
\end{tasks}



\end{document}


