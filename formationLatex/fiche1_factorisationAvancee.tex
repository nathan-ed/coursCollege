\documentclass[a4paper,12pt]{report}
\usepackage[fiche,niveau={1MA1},nfiche={1}, annee={Emilie-Gourd, 2024--2025}, auteur={ns},theme={Factorisation -- Enchaîner les techniques},top,nobackground]{packages/bandeaux}
\usepackage{packages/boites}
\usepackage{packages/fontes}
\usepackage{packages/courslatex}

\begin{document}

 Pour factoriser il faut 
  \begin{enumerate}
	  \item mettre en évidence le plus grand facteur possible;
	  \item vérifier si on se trouve face à une identité remaquable;
	  \item appliquer la technique des groupements pour révéler une factorisation plus poussée;
	\item réduire les expressions obtenues;
	\item recommencer les étapes ci-dessus.
\end{enumerate}
Factoriser une expression est un problème difficile, il n'y a pas de marche à suivre précise, mais il faut maîtriser les techniques suivantes afin reconnaître les situations dans lesquelles il est nécessaire de les appliquer.
\begin{description}
	\item[Mise en évidence] On détermine le pgcd des coefficients et la partie littérale commune de chacun des termes.
\item[Identités remarquables] On applique les schémas vus dans la théorie pour nous aider à choisir de quelle identité il s'agit.
\item[Technique des groupements] On essaye de faire apparaître un facteur commun à plusieurs termes, mais pas à tous en même temps.
\end{description}

\begin{boiteExT}[Exemple de factorisation avec groupements]
\vspace{3cm}

\end{boiteExT}
Les groupements peuvent être égaux au signe près. On multiplie par $1=(-1)^2$ pour obtenir un changement de signe du groupement.
%\[(x-y)(3x-4)=(-1)^2(x-y)(3x-4)=(-1)(x-y)(-1)(3x-4)=(y-x)(4-3x)\]
\begin{boiteExT}[Retrouver un groupement au signe près]
\vspace{4cm}

\end{boiteExT}
Les techniques s'enchaînent. Voici quelques exemples.

\begin{boiteExT}[Identité remarquable avec groupements]
	\vspace{4cm}	

\end{boiteExT}

\begin{boiteExT}[Identité remarquable pour faire apparaître un groupement]
	\vspace{6cm}	

\end{boiteExT}

\begin{boiteExT}[Groupement pour faire apparaître une identité]
	\vspace{6cm}	

\end{boiteExT}

\begin{exo}
Factoriser le plus possible les expressions suivantes.
  \begin{tasks}(2)
  	\task $(7x-1)^2-(5x+2)^2$
	\task $(4x-1)^2-9(3-x)^2$
	\task $2x^2-4x+2-3(x-1)(2x+1)$
	\task $25x^2+(5x-3)(2x+7)-9+(6-10x)(x-3)$
  \end{tasks}
\end{exo}


\end{document}
