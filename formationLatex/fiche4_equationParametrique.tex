\documentclass[a4paper,12pt]{report}
\usepackage[fiche,niveau={1MA1},nfiche={4}, annee={Emilie-Gourd, 2024--2025}, auteur={ns},theme={Équations -- Premier degré à une inconune avec un paramètre}]{packages/bandeaux}
\usepackage{packages/boites}
\usepackage{packages/fontes}
\usepackage{packages/courslatex}

\begin{document}

\begin{boiteExT}[Une première équation avec un paramètre (discussion)]
	\vspace{8cm}

\end{boiteExT}


\begin{boiteExT}[Reconnaître une équation avec un paramètre]
	\vspace{6cm}

\end{boiteExT}

\begin{boiteExT}[Exemple 1 $(k+2)x=8$]
	\vspace{6cm}

\end{boiteExT}


\begin{boiteExT}[Exemple 2 $a(x-3)=5a$]
	\vspace{6cm}

\end{boiteExT}

\begin{boiteExT}[Exemple 3 $(p-1)x+3=3x$] 
	\vspace{6cm}

\end{boiteExT}

\begin{boiteExT}[Exemple 4 $(m+3)x+7=mx+10$]
	\vspace{6cm}

\end{boiteExT}
\begin{exo}
Résoudre dans $\R$ pour $x$.
\begin{tasks}(3)
	\task $4 m x-8-11 m=3+2 m$
	\task $5x-3p=2x-7$
	\task $9r^2x+2r=4x-5$
\end{tasks}
\end{exo}
\end{document}


