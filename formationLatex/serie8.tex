\documentclass[a4paper,12pt]{report}
\usepackage[serie,niveau={1MA1},nfiche={}, annee={Emilie-Gourd, 2024--2025}, auteur={ns},theme={Série 8}]{packages/bandeaux}
\usepackage{packages/boites}
\usepackage{packages/fontes}
\usepackage{packages/courslatex}

\begin{document}
{\bfseries Lundi $\longrightarrow$ automatismes et ex. 1,2,3  \hfill Jeudi $\longrightarrow$ ex. 4,5,6,7}

\textLigne{Exercices}
\begin{exo}[1]
Parmi les égalités suivantes, lesquelles sont toujours vraies? lesquelles toujours fausses ? lesquelles parfois vraies parfois fausses ?
	\begin{tasks}(4)
\task $5+5=5^2$
\task $x+x=x^2$
\task $x+x=2 x$
\task $(x+1)^3=x^3+1^3$
\task $0 \cdot x=1$
\task $x^2 \cdot x^2 \cdot x^2=3 x^2$
\task $(x+1)^2=x^2+2 x+1$
\task $0 \cdot x=0$
	\end{tasks}
\end{exo}
\begin{exo}[1]
	Répondre par vrai ou faux en justifiant.
\begin{tasks}
	\task  Le nombre $-8$ est-il solution de l'équation : $x^2=32-4 x$ ?
\task Le nombre $0$ est-il solution de l'équation : $x^2+12 x+12=3 x^3-3 x^2-x+12$ ?
\task Le nombre $-\dfrac{1}{2}$ est-il solution de l'équation : $x(x-2)=x^2-1$ ?
\task Le nombre $\dfrac{1}{2}$ est-il solution de l'équation : $x(x-2)=x^2-1$ ?
\end{tasks}
\end{exo}
\begin{exo}[2]
On considère l'équation : $x^3-4=15 x$.
	\begin{tasks}
\task Un entier naturel est solution de cette équation; trouver lequel et justifier à l'aide de la définition du mot solution.
\task Montrer que le nombre irrationnel $\sqrt{3}-2$ est aussi solution de cette équation.
	\end{tasks}
\end{exo}
\begin{exo}[2]
Observer les écritures suivantes pour trouver comment les réduire sans développer les carrés.
	\begin{tasks}(3)
\task $(2 x-y+1)^2-(2 x+y+1)^2$
\task*(2) $(2 x+y)^2+2(2 x+y)(2 x-y)+(2 x-y)^2$
\task $\left(\dfrac{1}{2} x-\dfrac{1}{2} y\right)^2-\left(\dfrac{1}{2} x+\dfrac{1}{2} y\right)^2$
\task*(2) $\left(x^2-2\right)^2-2\left(x^2-2\right)\left(x^2+x+1\right)+\left(x^2+x+1\right)^2$
	\end{tasks}
\end{exo}
\begin{exo}[3]
Déterminer le nombre $a$ pour que l'équation ait la solution demandée.
	\begin{tasks}(2)
\task $a x+1=2 x+5$; solution: $S=\{-2\}$;
\task $1-a x=4 x+2$; solution : $S=\left\{\dfrac{1}{3}\right\}$
\task $3=a \cdot\left(-\dfrac{1}{2} x+3\right)$; solution : $S=\{-1\}$
\task $7-2 x=x+a x$; solution: $S=\{3\}$.
	\end{tasks}
\end{exo}
\begin{exo}[1]
Compléter les équations b), c) et d) pour obtenir des équations équivalentes à l'équation $A$.
	\begin{tasks}(4)
\task $x=\dfrac{2}{5} y-2$
\task $5 x=\ldots$
\task $x+2=\ldots$
\task $\dfrac{5}{2} x=\ldots$
	\end{tasks}
\end{exo}
\begin{exo}[2]
Traduire chaque phrase par une équation, puis résoudre.
	\begin{tasks}
		\task \enquote{Le triple du nombre $x$ vaut 2 de plus que $x$.}
		\task \enquote{La somme de $x$ et de 3 vaut 2 de moins que le double de $x$.}
	\task \enquote{Le double d'un nombre dépasse ses deux tiers de 10.}
\task \enquote{Si l'on soustrait le dixième de $x$ au quart de $x$ on obtient 2 de moins que $x$.}
\task \enquote{Si l'on retranche 5 du triple de $x$, on obtient la moitié de la somme de 3 et de $x$.}
	\end{tasks}
\end{exo}

\textLigne{Automatismes}
Résoudre les équations suivantes dans $\mathbb{R}$.

En écrivant les principes d'équivalence pour chaque étape.
	\begin{tasks}(4)
\task $4 x=9 x$
\task $6 x+3=5 x$
\task $4 x-5=3 x+2$
\task $8 x=9 x+3$
\end{tasks}
De tête.
\begin{tasks}(4)
\task $4 x-7=10 x-7$
\task $5 x+1=5 x-1$
\task $5+2 x=4 x-5$
\task $7 x-8=-x$
\task $3 x-1=3 x-1$
\task $13 x-1=5 x-2$
\task $6 x+4=2 x+10$
\task $2-5 x=11-3 x$
\end{tasks}

Par écrit.
	\begin{tasks}(3)
\task $\dfrac{x}{2}+\dfrac{x}{3}=\dfrac{5}{6}$
\task $\dfrac{x}{3}+\dfrac{4}{5}=\dfrac{5}{6}$
\task $\dfrac{x-15}{5}-\dfrac{4-3 x}{4}=15$
\task $x+\dfrac{x}{6}-\dfrac{x}{3}=3$
\task $\dfrac{2 x}{3}-\dfrac{4 x}{9}=\dfrac{3}{5} \cdot \dfrac{x}{2}$
\task $\dfrac{x+1}{4}-\dfrac{x-1}{3}=0$
\task $\dfrac{x+3}{5}+\dfrac{x+3}{4}=\dfrac{9}{5}$
\task $\dfrac{x}{4}-\dfrac{x}{8}=\dfrac{3}{24} x-1$
	\end{tasks}
	\begin{comment}
\begin{exo}[1]
Résoudre quatre fois de suite l'équation $\dfrac{x}{2}-3 x=\dfrac{5}{4}+x$, en utilisant la méthode proposée :
	\begin{tasks}
\task Votre manière de faire. 

(Dans les méthodes b), c) et d), simplifier au fur et à mesure l'expression obtenue.)
\task $\left[P E_2\right]$, en multipliant par 4 ; puis $\left[P E_1\right]$, en ajoutant $-4 x$; puis $\left[P E_2\right]$, en multipliant par $-\dfrac{1}{14}$.
\task $\left[P E_1\right]$, en ajoutant $-x$; puis $\left[P E_2\right]$, en multipliant par 2 ; puis $\left[P E_2\right]$, en multipliant par $-\dfrac{1}{7}$.
\task $\left[P E_1\right]$, en ajoutant $\dfrac{5}{2} x$; puis $\left[P E_1\right]$, en ajoutant $-\dfrac{5}{4} ;$ puis $\left[P E_2\right]$, en multipliant par $\dfrac{2}{7}$.
	\end{tasks}
\end{exo}

	\end{comment}
\end{document}
