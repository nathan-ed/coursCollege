\documentclass{beamer}

\input{./_preambules/preambule_mathalea.tex}

\ExplSyntaxOn

% Define competences property list
\prop_new:N \g_competences_prop
\prop_new:N \g_competences_prop_index
\prop_new:N \g_competences_prop_year_chapter

% Simplify competence addition syntax
\NewDocumentCommand{\AddCompetence}{mmmm} {
    \prop_gput:Nnn \g_competences_prop_index { #1 } { #4 }
 % Define a unique clist variable for temporary storage
\clist_clear_new:N \l_my_temp_clist
    % Retrieve the current list of competences for the Year:Chapter key
    \prop_get:NnNTF \g_competences_prop_year_chapter { #2:#3 } \l_my_temp_clist
        { \clist_put_right:Nn \l_my_temp_clist { #4 } } % Append to the existing clist
        { \clist_set:Nn \l_my_temp_clist { #4 } } % Create a new clist
    % Update the property list with the updated clist
    \prop_put:NnV \g_competences_prop_year_chapter { #2:#3 } \l_my_temp_clist
}
% Retrieve all competences for a specific Year:Chapter and print them
\NewDocumentCommand{\ListCompetencesYearChapter}{mm} {
    % Define a unique clist variable for retrieval
	\begin{itemize}
\clist_clear_new:N \l_my_temp_clists
    \prop_get:NnNT \g_competences_prop_year_chapter { #1:#2 } \l_my_temp_clist {
        \clist_map_inline:Nn \l_my_temp_clist { \item ##1 }
    }
\end{itemize}
}

% Load competences from file only once
\AtBeginDocument{
    \file_input:n { competences.tex }
}

% Define a command to retrieve a specific competence
\NewExpandableDocumentCommand{\GetCompetence}{ m } {
    \prop_item:Nn \g_competences_prop_index { #1 }
}


\ExplSyntaxOff


\usepackage{fontawesome}
\title{Exercices de début de cours (edc)}
\subtitle{Automatismes et autres compétences -- 1$^{\text{re}}$ année du collège}
\author{ns}
\date{}
\setbeamerfont{frametitle}{size=\normalsize}
% define a new counter
\newcounter{exercice}
\setcounter{exercice}{0}
\begin{document}

%%%%%%%%%%%%%%%%%%%%%%%%%%%%%%%%%%%%%%%%%%%%%%%%%%%%%%%%%%%%%%%%%%%%%%%%%%%%%%%
%%%%%%%%%%%%%%%%%%%%%%%%%%%%%%%%%%%%%%%%%%%%%%%%%%%%%%%%%%%%%%%%%%%%%%%%%%%%%%%
%%%%%%%%%%%%%%%%%%%%%%%%%%%%%%%%%%%%%%%%%%%%%%%%%%%%%%%%%%%%%%%%%%%%%%%%%%%%%%%
\begin{frame}
\titlepage
\end{frame}

\stepcounter{exercice}
\begin{frame}{\GetCompetence{FactTri}, edc \theexercice}
	% increment the counter
 \onslide<1->{ Utiliser la méthode de résolution du deuxième degré pour factoriser l'expression suivante. 
	 \ListCompetencesYearChapter{1M}{5}
Dans la forme factorisée, tous les coefficients doivent être entiers.
 \[15x^2+32x-7\]}
 \onslide<2->{$\Delta=32^2-4\cdot 15\cdot (-7)=1444$. Les racines du polynôme sont 
	 \[S=\left\{\dfrac{1}{5};-\dfrac{7}{3}\right\}.\]

	 La forme factorisée est 
 \[(5x-1)(3x+7)\]}
\end{frame}

\stepcounter{exercice}
\begin{frame}{ \GetCompetence{CombLin}, edc \theexercice}
	\onslide<1->{
Résoudre les systèmes d'équations suivants par combinaison linéaire :
\begin{multicols}{2}
\begin{enumerate}
	\item  $\begin{cases}\begin{aligned}-x-2y &=1\\-6x-4y &=-26\end{aligned}\end{cases}$
	\item  $\begin{cases}\begin{aligned}-4x+6y &=-44\\-3x+4y &=-32\end{aligned}\end{cases}$
\end{enumerate}
\end{multicols}}
\onslide<2->{Les solutions sont :
	\begin{multicols}{2}
\begin{enumerate}
\item $S=\left\{\left(7;-4\right)\right\}$.
\item $S=\left\{\left(8;-2\right)\right\}$.
\end{enumerate}
\end{multicols}}
\end{frame}
\end{document}
