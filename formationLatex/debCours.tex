\documentclass{beamer}

\input{./_preambules/preambule_mathalea.tex}
\usepackage{fontawesome}
\title{Début de cours}
\subtitle{Automatismes et autres compétences}
\author{ns}

\begin{document}

%%%%%%%%%%%%%%%%%%%%%%%%%%%%%%%%%%%%%%%%%%%%%%%%%%%%%%%%%%%%%%%%%%%%%%%%%%%%%%%
%%%%%%%%%%%%%%%%%%%%%%%%%%%%%%%%%%%%%%%%%%%%%%%%%%%%%%%%%%%%%%%%%%%%%%%%%%%%%%%
%%%%%%%%%%%%%%%%%%%%%%%%%%%%%%%%%%%%%%%%%%%%%%%%%%%%%%%%%%%%%%%%%%%%%%%%%%%%%%%
\begin{frame}
\titlepage
\end{frame}

\begin{frame}{Résoudre pour factoriser, exercice 1}
 \onslide<1->{Utiliser la méthode de résolution pour factoriser l'expression suivante, dans la forme factorisée, tous les coefficients doivent être entiers.
 \[15x^2+32x-7\]}
 \onslide<2->{La forme factorisée est 
 \[(5x-1)(3x+7)\]}
\end{frame}

\begin{frame}{Système d'équations par combinaison linéaire exercice 1}
	\onslide<1->{
Résoudre les systèmes d'équations suivants par combinaison linéaire :
\vfill
\begin{multicols}{2}
\begin{enumerate}
	\item  $\begin{cases}\begin{aligned}-x-2y &=1\\-6x-4y &=-26\end{aligned}\end{cases}$
	\item  $\begin{cases}\begin{aligned}-4x+6y &=-44\\-3x+4y &=-32\end{aligned}\end{cases}$
\end{enumerate}
\end{multicols}}
\onslide<2->{Les solutions sont :
	\begin{multicols}{2}
\begin{enumerate}
\item $S=\left\{\left(8;-2\right)\right\}$.
\item $S=\left\{\left(7;-4\right)\right\}$.
\end{enumerate}
\end{multicols}}
\end{frame}
\end{document}
