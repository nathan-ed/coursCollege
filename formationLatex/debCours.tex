\documentclass{beamer}

%%% Essential packages
\RequirePackage{amsmath}       % For mathematical environments
\RequirePackage{amssymb}       % For additional math symbols
\RequirePackage{graphicx}      % For including images
\RequirePackage{tikz}          % For creating diagrams
\RequirePackage{pgfplots}      % For creating plots
\RequirePackage{etoolbox}      % For conditional processing
\RequirePackage{babel}         % For language support
\RequirePackage{booktabs}      % For table rules
\RequirePackage{color}         % For colored text
\RequirePackage{colortbl}      % For colored tables
\RequirePackage{cancel}        % For canceling terms in equations
\RequirePackage{diagbox}       % For diagonal boxes in tables
\RequirePackage{eurosym}       % For euro symbol
\RequirePackage{fancybox}      % For fancy boxes
%\RequirePackage{gensymb}       % For degree and other symbols
\RequirePackage{longtable}     % For long tables
\RequirePackage{mathtools}     % For advanced math tools
\RequirePackage{multicol}      % For multiple column layouts
\RequirePackage{multirow}      % For multi-row tables
\RequirePackage{numprint}      % For number formatting
\RequirePackage{pgf}           % For PGF graphics
\RequirePackage{pifont}        % For ding symbols
\RequirePackage{pst-eucl}      % For geometric constructions in PSTricks
\RequirePackage{pst-grad}      % For gradient fills
\RequirePackage{pst-node}      % For node connections in PSTricks
\RequirePackage{pst-plot}      % For plotting in PSTricks
\RequirePackage{siunitx}       % For SI units
\RequirePackage{stmaryrd}      % For additional math symbols
\RequirePackage{tabularx}      % For advanced tables
\RequirePackage{tkz-euclide}   % For geometric constructions with TikZ
\RequirePackage{tkz-tab}       % For function tables
\RequirePackage{xcolor}        % For advanced color features
\RequirePackage{xlop}          % For arithmetic operations in LaTeX
\RequirePackage{wrapfig}       % For wrapping text around figures

%%% Additional TikZ libraries
\usetikzlibrary{arrows}
\usetikzlibrary{arrows.meta}
\usetikzlibrary{backgrounds}
\usetikzlibrary{calc}
\usetikzlibrary{decorations.pathmorphing}
\usetikzlibrary{decorations.pathreplacing}
\usetikzlibrary{patterns}
\usetikzlibrary{shapes}
\usetikzlibrary{shapes.geometric}
\usetikzlibrary{shapes.misc}


\usepackage{fontawesome}
\title{Début de cours}
\subtitle{Automatismes et autres compétences}
\author{ns}

\begin{document}

%%%%%%%%%%%%%%%%%%%%%%%%%%%%%%%%%%%%%%%%%%%%%%%%%%%%%%%%%%%%%%%%%%%%%%%%%%%%%%%
%%%%%%%%%%%%%%%%%%%%%%%%%%%%%%%%%%%%%%%%%%%%%%%%%%%%%%%%%%%%%%%%%%%%%%%%%%%%%%%
%%%%%%%%%%%%%%%%%%%%%%%%%%%%%%%%%%%%%%%%%%%%%%%%%%%%%%%%%%%%%%%%%%%%%%%%%%%%%%%
\begin{frame}
\titlepage
\end{frame}

\begin{frame}{Résoudre pour factoriser, exercice 1}
 \onslide<1->{Utiliser la méthode de résolution pour factoriser l'expression suivante, dans la forme factorisée, tous les coefficients doivent être entiers.
 \[15x^2+32x-7\]}
 \onslide<2->{La forme factorisée est 
 \[(5x-1)(3x+7)\]}
\end{frame}

\begin{frame}{Système d'équations par combinaison linéaire exercice 1}
	\onslide<1->{
Résoudre les systèmes d'équations suivants par combinaison linéaire :
\vfill
\begin{multicols}{2}
\begin{enumerate}
	\item  $\begin{cases}\begin{aligned}-x-2y &=1\\-6x-4y &=-26\end{aligned}\end{cases}$
	\item  $\begin{cases}\begin{aligned}-4x+6y &=-44\\-3x+4y &=-32\end{aligned}\end{cases}$
\end{enumerate}
\end{multicols}}
\onslide<2->{Les solutions sont :
	\begin{multicols}{2}
\begin{enumerate}
\item $S=\left\{\left(8;-2\right)\right\}$.
\item $S=\left\{\left(7;-4\right)\right\}$.
\end{enumerate}
\end{multicols}}
\end{frame}
\end{document}
