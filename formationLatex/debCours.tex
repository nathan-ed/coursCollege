\documentclass{beamer}

%%% Essential packages
\RequirePackage{amsmath}       % For mathematical environments
\RequirePackage{amssymb}       % For additional math symbols
\RequirePackage{graphicx}      % For including images
\RequirePackage{tikz}          % For creating diagrams
\RequirePackage{pgfplots}      % For creating plots
\RequirePackage{etoolbox}      % For conditional processing
\RequirePackage{babel}         % For language support
\RequirePackage{booktabs}      % For table rules
\RequirePackage{color}         % For colored text
\RequirePackage{colortbl}      % For colored tables
\RequirePackage{cancel}        % For canceling terms in equations
\RequirePackage{diagbox}       % For diagonal boxes in tables
\RequirePackage{eurosym}       % For euro symbol
\RequirePackage{fancybox}      % For fancy boxes
%\RequirePackage{gensymb}       % For degree and other symbols
\RequirePackage{longtable}     % For long tables
\RequirePackage{mathtools}     % For advanced math tools
\RequirePackage{multicol}      % For multiple column layouts
\RequirePackage{multirow}      % For multi-row tables
\RequirePackage{numprint}      % For number formatting
\RequirePackage{pgf}           % For PGF graphics
\RequirePackage{pifont}        % For ding symbols
\RequirePackage{pst-eucl}      % For geometric constructions in PSTricks
\RequirePackage{pst-grad}      % For gradient fills
\RequirePackage{pst-node}      % For node connections in PSTricks
\RequirePackage{pst-plot}      % For plotting in PSTricks
\RequirePackage{siunitx}       % For SI units
\RequirePackage{stmaryrd}      % For additional math symbols
\RequirePackage{tabularx}      % For advanced tables
\RequirePackage{tkz-euclide}   % For geometric constructions with TikZ
\RequirePackage{tkz-tab}       % For function tables
\RequirePackage{xcolor}        % For advanced color features
\RequirePackage{xlop}          % For arithmetic operations in LaTeX
\RequirePackage{wrapfig}       % For wrapping text around figures

%%% Additional TikZ libraries
\usetikzlibrary{arrows}
\usetikzlibrary{arrows.meta}
\usetikzlibrary{backgrounds}
\usetikzlibrary{calc}
\usetikzlibrary{decorations.pathmorphing}
\usetikzlibrary{decorations.pathreplacing}
\usetikzlibrary{patterns}
\usetikzlibrary{shapes}
\usetikzlibrary{shapes.geometric}
\usetikzlibrary{shapes.misc}



\ExplSyntaxOn

% Define competences property list
\prop_new:N \g_competences_prop
\prop_new:N \g_competences_prop_index
\prop_new:N \g_competences_prop_year_chapter
\prop_new:N \g_competences_prop_year_chapter_uuid

% Simplify competence addition syntax
\NewDocumentCommand{\AddCompetence}{mmmm} {
    \prop_gput:Nnn \g_competences_prop_index { #1 } { #4 }
 % Define a unique clist variable for temporary storage
\clist_clear_new:N \l_my_temp_clist
    % Retrieve the current list of competences for the Year:Chapter key
    \prop_get:NnNTF \g_competences_prop_year_chapter { #2:#3 } \l_my_temp_clist
        { \clist_put_right:Nn \l_my_temp_clist { #4 } } % Append to the existing clist
        { \clist_set:Nn \l_my_temp_clist { #4 } } % Create a new clist
    % Update the property list with the updated clist
    \prop_put:NnV \g_competences_prop_year_chapter { #2:#3 } \l_my_temp_clist

    \clist_clear_new:N \l_temp_clist_with_uuid
    \prop_get:NnNTF \g_competences_prop_year_chapter_uuid { #2:#3 } \l_temp_clist_with_uuid
    { \clist_put_right:Nn \l_temp_clist_with_uuid {[#1] #4 } }
	{ \clist_set:Nn \l_temp_clist_with_uuid {[#1] #4 } }
    \prop_put:NnV \g_competences_prop_year_chapter_uuid { #2:#3 } \l_temp_clist_with_uuid
}
% Retrieve all competences for a specific Year:Chapter and print them
\NewDocumentCommand{\ListCompetencesYearChapter}{mm} {
    % Define a unique clist variable for retrieval
	\begin{itemize}
\clist_clear_new:N \l_my_temp_clists
    \prop_get:NnNT \g_competences_prop_year_chapter { #1:#2 } \l_my_temp_clist {
        \clist_map_inline:Nn \l_my_temp_clist { \item ##1 }
    }
\end{itemize}
}

% Retrieve all competences for a specific Year:Chapter and print them
\NewDocumentCommand{\ListCompetencesYearChapterUUID}{mm} {
    % Define a unique clist variable for retrieval
	\begin{description}[ align=left,
  labelwidth=3em,
  leftmargin=!]
\clist_clear_new:N \l_my_temp_clists
    \prop_get:NnNT \g_competences_prop_year_chapter_uuid { #1:#2 } \l_temp_clist_with_uuid {
	    \clist_map_inline:Nn \l_temp_clist_with_uuid { \item##1 }
    }
\end{description}
}

% Load competences from file only once
\AtBeginDocument{
    \file_input:n { competences/competences.tex }
}

% Define a command to retrieve a specific competence
\NewExpandableDocumentCommand{\GetCompetence}{ m } {
    \prop_item:Nn \g_competences_prop_index { #1 }
}


\ExplSyntaxOff


\usepackage{fontawesome}
\title{Exercices de début de cours (edc)}
\subtitle{Automatismes et autres compétences -- 1$^{\text{re}}$ année du collège}
\author{ns}
\date{}
\setbeamerfont{frametitle}{size=\normalsize}
% define a new counter
\newcounter{exercice}
\setcounter{exercice}{0}
\begin{document}

%%%%%%%%%%%%%%%%%%%%%%%%%%%%%%%%%%%%%%%%%%%%%%%%%%%%%%%%%%%%%%%%%%%%%%%%%%%%%%%
%%%%%%%%%%%%%%%%%%%%%%%%%%%%%%%%%%%%%%%%%%%%%%%%%%%%%%%%%%%%%%%%%%%%%%%%%%%%%%%
%%%%%%%%%%%%%%%%%%%%%%%%%%%%%%%%%%%%%%%%%%%%%%%%%%%%%%%%%%%%%%%%%%%%%%%%%%%%%%%
\begin{frame}
\titlepage
\end{frame}

\stepcounter{exercice}
\begin{frame}{\GetCompetence{FactTri}, edc \theexercice}
	% increment the counter
 \onslide<1->{ Utiliser la méthode de résolution du deuxième degré pour factoriser l'expression suivante. 
Dans la forme factorisée, tous les coefficients doivent être entiers.
 \[15x^2+32x-7\]}
 \onslide<2->{$\Delta=32^2-4\cdot 15\cdot (-7)=1444$. Les racines du polynôme sont 
	 \[S=\left\{\dfrac{1}{5};-\dfrac{7}{3}\right\}.\]

	 La forme factorisée est 
 \[(5x-1)(3x+7)\]}
\end{frame}

\stepcounter{exercice}
\begin{frame}{ \GetCompetence{CombLin}, edc \theexercice}
	\onslide<1->{
Résoudre les systèmes d'équations suivants par combinaison linéaire :
\begin{multicols}{2}
\begin{enumerate}
	\item  $\begin{cases}\begin{aligned}-x-2y &=1\\-6x-4y &=-26\end{aligned}\end{cases}$
	\item  $\begin{cases}\begin{aligned}-4x+6y &=-44\\-3x+4y &=-32\end{aligned}\end{cases}$
\end{enumerate}
\end{multicols}}
\onslide<2->{Les solutions sont :
	\begin{multicols}{2}
\begin{enumerate}
\item $S=\left\{\left(7;-4\right)\right\}$.
\item $S=\left\{\left(8;-2\right)\right\}$.
\end{enumerate}
\end{multicols}}
\end{frame}

\stepcounter{exercice}
\begin{frame}{\GetCompetence{3f3}, edc \theexercice}
	\onslide<1->{Résoudre le système d'équations suivant dans $\mathbb{R}^3$ :
	\[\begin{cases}\begin{aligned}x+2y-z &=2\\x-y+2z &=5\\2x+2y+2z &=12\end{aligned}\end{cases}\]}
	\onslide<2->{La solution est $S=\left\{\left(1;2;3\right)\right\}$.}	
\end{frame}

\stepcounter{exercice}
\begin{frame}{\GetCompetence{3f3}, edc \theexercice}
	\onslide<1->{Résoudre le système d'équations suivant dans $\mathbb{R}^3$ :
	\[\begin{cases}\begin{aligned}3x+4y-2z &=10\\5x-2y+7z &=1\\3x+3y+z &=6\end{aligned}\end{cases}\]}
	\onslide<2->{La solution est $S=\left\{\left(\dfrac{96}{47};\dfrac{17}{47};-\dfrac{57}{47}\right)\right\}$.}	
\end{frame}


\stepcounter{exercice}
\begin{frame}{\GetCompetence{3f3}, edc \theexercice}
	\onslide<1->{Résoudre le système d'équations suivant dans $\mathbb{R}^3$ :
	\[\begin{cases}\begin{aligned}10x+3y+2z &=8\\4x+2y-5z &=-8\\3x-7y+2z &=6\end{aligned}\end{cases}\]}
	\onslide<2->{La solution est $S=\left\{\left(\dfrac{202}{447};-\dfrac{52}{447};\dfrac{856}{447}\right)\right\}$.}	
\end{frame}
\end{document}
