\documentclass[a4paper,12pt]{report}
\usepackage[serie,niveau={1MA1},nfiche={}, annee={Emilie-Gourd, 2024--2025}, auteur={ns},theme={Séries 16 à 21 sur les fonctions}]{packages/bandeaux}
\usepackage{packages/boites}
\usepackage{packages/fontes}
\usepackage{packages/courslatex}
\input{pics/cuboid}
\RequirePackage{icomma}
%chemin vers les sources d'exercices
\renewcommand{\path}{./}

\newboolean{solution} % Boolean to indicate if solutions should be shown
\newboolean{uuid} % Boolean to indicate if UUID should be shown
\newboolean{link} % Boolean to indicate if a link should be shown
\newcounter{num} % Counter to keep track of exercise numbers

% liens vers des ressources extérieures :
\newcommand{\pathcorrige}{https://raw.githubusercontent.com/smaxx73/Exercices/main/pdf/pdf/} % Path to corrected exercises
\newcommand{\pathnotebook}{https://github.com/smaxx73/Exercices/blob/main/notebook/} % Path to notebooks
\newcommand{\pathexercice}{https://raw.githubusercontent.com/smaxx73/Exercices/main/pdf/pdf_sansol/} % Path to exercises without solutions

% Define the title of the exercise
\newcommand{\titre}[1]{%
	\def\TitreExo{#1} % Set the title for later use
}

% Define the content of the exercise
\newcommand{\contenu}[1]{%
	\def\Contenu{#1} % Set the content for later use
}

% Define the number of points for the exercise
\newcommand{\pts}[1]{%
	\def\Points{#1} % Set the number of points for later use
}

% Define the number of points for the exercise
\newcommand{\piments}[1]{%
	\def\Piments{#1} % Set the number of points for later use
}

% Define the year of the exercise
\newcommand{\annee}[1]{
	\def\Annee{#1} % Set the year for later use
}

% Define the correction (solution) of the exercise
\newcommand{\correction}[1]{%
	\def\Solution{#1} % Set the solution for later use
}

% Directly use code as it is
\newcommand{\code}[1]{#1} % Simple passthrough command for code

% Directly use text as it is
\newcommand{\texte}[1]{#1} % Simple passthrough command for text

% Define the theme (currently unused)
\newcommand{\theme}[1]{} % Placeholder for theme

% Define the author (currently unused)
\newcommand{\auteur}[1]{} % Placeholder for author

% Define the organization (currently unused)
\newcommand{\organisation}[1]{} % Placeholder for organization

% Command for question text
\newcommand{\question}[1]{#1} % Simple passthrough command for question

% Command for providing a response, shown only if the solution boolean is true
\newcommand{\reponse}[1]{%
	\ifthenelse{\boolean{solution}} % Check if 'solution' is true
	{%
		\begin{solutionbox} % Begin solution box environment
			\begin{footnotesize} #1 \end{footnotesize} % Display the response in small font size
		\end{solutionbox}}{} % End solution box if 'solution' is true
} 

% Move to the next exercise by incrementing the counter
\newcommand{\nextexo}{%
	\addtocounter{num}{1} % Increment exercise counter
	\vspace{1em} % Add vertical space
}


\newcommand{\insertexo}[4]{% contenu, uuid, type (exo/cor/both), upside-down solution
    \input{\path src/#1} % Input the exercise content from the provided path
    \setboolean{uuid}{#2} % Set the uuid boolean
    
    % Start grouping to tightly control scope
    \begingroup
    % Print the exercise if #3 equals "exo" or "both"
    \ifthenelse{\equal{#3}{exo} \OR \equal{#3}{both}}{%
        \noindent % Ensure no extra indentation
        \begin{exo}[\Piments] % Begin question environment with the number of points
            \ifthenelse{\boolean{uuid}}{#1}{} % Include uuid if true
            \Contenu % Insert the content of the exercise
        \end{exo}
    }{} % Otherwise, do nothing
    
    % Print the correction if #3 equals "cor" or "both"
    \ifthenelse{\equal{#3}{cor} \OR \equal{#3}{both}}{%
        \noindent % Ensure no extra indentation
        \ifthenelse{\equal{#4}{true}}{%
            \begin{core-upside} % Use core-upside for upside-down solution
                {\footnotesize \Solution} % Insert the solution
            \end{core-upside}
        }{%
            \begin{core} % Use normal core for regular solution
                \Solution % Insert the solution
            \end{core}
        }%
    }{} % Otherwise, do nothing
    
    \endgroup % End the group
}

% Command to list exercises from a given list
\newcommand{\listexo}[1]{% liste of exercises
	\foreach \ex in #1 { % Iterate over each exercise in the list
		\nextexo % Move to the next exercise
		\insertexo{\ex}{\solution}{\uuid}{\link}{\thenum} % Insert the current exercise with its properties
	}
}

% Counter for QR codes
\newcounter{qrcodenum} % Define a counter to track QR codes

% Command to create a list of QR codes for exercises
\newcommand{\listeqrcode}[2]{% liste d'exercices, initialisation du compteur
\setcounter{qrcodenum}{#2} % Set the initial value for the QR code counter
\noindent
\foreach \exo in #1{ % Iterate over each exercise in the list
	% Create a minipage for each exercise with its number, title, and QR code
	\begin{minipage}{0.24\textwidth}  % Adjust the width to have 4 QR codes per line
		\centering
		\href{\pathcorrige\exo.pdf}{\texttt{Ex \theqrcodenum : \exo \vspace{1em}}}  % Display the number and title of the exercise with a hyperlink
		{\qrcode{\pathcorrige\exo.pdf}}  % Generate the QR code for the exercise
		\par\vspace{2em}  % Add vertical space between QR codes
	\end{minipage}
	% Increment the QR code counter
	\stepcounter{qrcodenum}
}
}


% Custom label using TikZ
\newcommand{\tikzlabel}[1]{
  \begin{tikzpicture}[remember picture,overlay]
    \node[draw, circle, line width=0.4pt, inner sep=2pt, anchor=east] {#1};
  \end{tikzpicture}
}
\begin{document}
\textLigne{Série 16 -- Généralités}
\insertexo{1M-pzws2}{false}{exo}{true}
\insertexo{1M-jj4kc}{false}{exo}{true}
\insertexo{1M-7r267}{false}{exo}{true}
\insertexo{1M-3j85f}{false}{exo}{true}
\insertexo{1M-jxb8v}{false}{exo}{true}
\insertexo{1M-v42bf}{false}{exo}{true}
\insertexo{1M-ctcch}{false}{exo}{true}
\insertexo{1M-cm4fj}{false}{exo}{true}
\insertexo{1M-umzs2}{false}{exo}{true}
\insertexo{1M-gs1br}{false}{exo}{true}


\textLigne{Série 17 -- Fonctions affines et droites}
\insertexo{1M-7aj4p}{false}{exo}{true}
\insertexo{1M-azp4k}{false}{exo}{true}
\insertexo{1M-we4mn}{false}{exo}{true}
\insertexo{1M-v45ts}{false}{exo}{true}
\insertexo{1M-5yfr4}{false}{exo}{true}
\insertexo{1M-ffeqn}{false}{exo}{true}
\insertexo{1M-qxjb5}{false}{exo}{true}
\begin{multicols}{2}
\insertexo{1M-f9zpd}{false}{exo}{true}
\insertexo{1M-2z37j}{false}{exo}{true}
\end{multicols}
\insertexo{1M-9ppsy}{false}{exo}{true}




\textLigne{Série 18 -- Fonctions affines et droites suite}
\insertexo{1M-trcsj}{false}{exo}{true}
\insertexo{1M-96gye}{false}{exo}{true}
\insertexo{1M-bxak6}{false}{exo}{true}
\insertexo{1M-t85gc}{false}{exo}{true}
\insertexo{1M-v4s75}{false}{exo}{true}
\insertexo{1M-3v9h8}{false}{exo}{true}


\textLigne{Série 19 -- Fonctions quadratiques et paraboles}
\insertexo{1M-habv6}{false}{exo}{true}
\insertexo{1M-y3kxz}{false}{exo}{true}

\textLigne{Série 20 -- Fonctions quadratiques et paraboles suite}
\insertexo{1M-vkggb}{false}{exo}{true}
\insertexo{1M-jyun7}{false}{exo}{true}

\textLigne{Série 21 -- Fonction inverse et fonction racine carrée}
\insertexo{1M-kjxge}{false}{exo}{true}
\insertexo{1M-u9her}{false}{exo}{true}
\insertexo{1M-uc1n8}{false}{exo}{true}


\textLigne{Exercices supplémentaires}
\insertexo{1M-sbyj3}{false}{exo}{true}
\insertexo{1M-k6m9k}{false}{exo}{true}
\insertexo{1M-xwye2}{false}{exo}{true}
\insertexo{1M-h9s4d}{false}{exo}{true}
\insertexo{1M-b4vhc}{false}{exo}{true}
\insertexo{1M-jr16y}{false}{exo}{true}
\medskip

\textLigne{Entraînement MathALEA -- Généralités}
\medskip

\qrwithlabel{Lire une image}{https://coopmaths.fr/alea/?uuid=6c6b3&id=1F1-2&uuid=b8946&id=1F1-11&lang=fr-CH&v=eleve&es=1011010&title=Lire+une+image+\%C3\%A0+partir+du+graphique}
\qrwithlabel{Utiliser le vocabulaire lié aux fonctions}{https://coopmaths.fr/alea/?uuid=0eecd&id=1F1-5&n=3&d=10&s=3&cd=0&uuid=4daef&id=1F1-6&n=5&d=10&s=5&cd=1&uuid=b92da&id=1F1-4&lang=fr-CH&v=eleve&es=1011010&title=Vocabulaire+des+fonctions}
\qrwithlabel{Calculer une image ou une préimage}{https://coopmaths.fr/alea/?uuid=c9382&id=1F1-7&n=3&d=10&s=3&s2=1&s3=5&cd=1&uuid=8a78e&id=1F1-8&n=2&d=10&s=1&s2=1&s3=4&cd=1&uuid=ba520&id=1F1-3&n=3&d=10&s=1-2&s2=2&s3=5&cd=1&lang=fr-CH&v=eleve&es=1011010&title=Calculs+d\%27images+et+de+pr\%C3\%A9images}
\qrwithlabel{Compléter un tableau de valeurs}{https://coopmaths.fr/alea/?uuid=afb2f&id=1F1-10&n=3&d=10&s=5&cd=1&lang=fr-CH&v=eleve&es=0011000&title=Compl\%C3\%A9ter+un+tableau+de+valeurs}

%\qrwithlabel{Déterminer si un point appartient au graphe d'une fonction}{}
\medskip

\textLigne{Entraînement MathALEA -- Fonctions affines et droite}
\medskip

\qrwithlabel{Déterminer la pente de la droite passant par deux points}{https://coopmaths.fr/alea/?uuid=1ea16&id=1F2-1&lang=fr-CH&v=eleve&es=0011010&title=D\%C3\%A9terminer+la+pente+d\%27une+droite+passant+par+deux+points}
\qrwithlabel{Déterminer l'équation d'une droite représentée graphiquement}{https://coopmaths.fr/alea/?uuid=41e6f&id=1F2-6&n=3&d=10&s=2&cd=1&lang=fr-CH&v=eleve&es=0011010&title=D\%C3\%A9terminer+l\%27\%C3\%A9quation+d\%27une+droite+repr\%C3\%A9sent\%C3\%A9e+graphiquement}
\qrwithlabel{Déterminer l'équation d'une droite connaissant sa pente et un point}{https://coopmaths.fr/alea/?uuid=d1da3&id=1F2-5&lang=fr-CH&v=eleve&es=0011000&title=D\%C3\%A9terminer+l\%27\%C3\%A9quation+connaissant+la+pente+et+un+point}
\qrwithlabel{Déterminer l'équation d'une droite passant par deux points}{https://coopmaths.fr/alea/?uuid=0cee9&id=1F2-2&uuid=1bb30&id=1F2-3&lang=fr-CH&v=eleve&es=1011010&title=D\%C3\%A9terminer+l\%27\%C3\%A9quation+d\%27une+droite+\%C3\%A0+partir+de+deux+points}

\qrwithlabel{Points alignés}{https://coopmaths.fr/alea/?uuid=b1777&id=1F2-7&n=3&d=10&s=1&cd=1&lang=fr-CH&v=eleve&es=0011000&title=Points+align\%C3\%A9s}
\qrwithlabel{Point d'intersection de deux droites}{https://coopmaths.fr/alea/?uuid=1b828&id=1F2-9&n=1&d=10&s=1&cd=1&uuid=4b211&id=1F2-11&uuid=8ff15&id=1F2-10&n=1&d=10&s=2&s2=1&cd=1&lang=fr-CH&v=eleve&es=1011000&title=Points+d\%27intersection}
\qrwithlabel{Droites parallèles et perpendiculaires}{https://coopmaths.fr/alea/?uuid=2db22&id=1F2-12&n=2&d=10&s=3&s2=3&cd=1&lang=fr-CH&v=eleve&es=0011000&title=Droites+parall\%C3\%A8les+et+perpendiculaires}
\medskip

\textLigne{Entraînement MathALEA -- Fonctions quadratiques et paraboles}
\medskip

\qrwithlabel{Lecture graphique}{https://coopmaths.fr/alea/?uuid=a896e&id=1F3-1&n=3&d=10&s=4&cd=1&lang=fr-CH&v=eleve&es=0011000&title=Lecture+graphique}
\qrwithlabel{Forme canonique}{https://coopmaths.fr/alea/?uuid=60504&id=1F3-2&n=2&d=10&cd=1&lang=fr-CH&v=eleve&es=0011000&title=Forme+canonique}
\qrwithlabel{Sens de variation}{https://coopmaths.fr/alea/?uuid=16f97&id=1F3-4&n=4&d=10&s=1-3-7-5&cd=1&lang=fr-CH&v=eleve&es=0011000&title=Sens+de+variation}
%\qrwithlabel{Points d'intersection}{}
\qrwithlabel{Retrouver l'équation d'une parabole}{https://coopmaths.fr/alea/?uuid=392b3&id=1F3-6&lang=fr-CH&v=eleve&es=0011000&title=\%C3\%89quation+de+la+parabole+connaissant+le+sommet+et+un+point}

\qrwithlabel{Étude complète d'une parabole}{https://coopmaths.fr/alea/?uuid=5ea23&id=1F3-3&lang=fr-CH&v=eleve&es=0011000&title=\%C3\%89tude+compl\%C3\%A8te+d\%27une+parabole}
\medskip


%\textLigne{Entraînement MathALEA -- Fonction inverse et fonction racine carrée}
%\medskip
%\qrwithlabel{Déterminer le domaine de définition d'une fonction}{}

\newpage

\textLigne{Élements de réponses}
\textLigne{Série 16 -- Généralités}
\insertexo{1M-pzws2}{false}{cor}{true}
\insertexo{1M-jj4kc}{false}{cor}{true}
\insertexo{1M-7r267}{false}{cor}{true}
\insertexo{1M-3j85f}{false}{cor}{true}
\insertexo{1M-jxb8v}{false}{cor}{true}
\insertexo{1M-v42bf}{false}{cor}{true}
\insertexo{1M-ctcch}{false}{cor}{true}
\insertexo{1M-cm4fj}{false}{cor}{true}
\insertexo{1M-umzs2}{false}{cor}{true}
\insertexo{1M-gs1br}{false}{cor}{true}


\textLigne{Série 17 -- Fonctions affines et droites}
\insertexo{1M-7aj4p}{false}{cor}{true}
\insertexo{1M-azp4k}{false}{cor}{true}
\insertexo{1M-we4mn}{false}{cor}{true}
\insertexo{1M-v45ts}{false}{cor}{true}
\insertexo{1M-5yfr4}{false}{cor}{true}
\insertexo{1M-ffeqn}{false}{cor}{true}
\insertexo{1M-qxjb5}{false}{cor}{true}
\insertexo{1M-f9zpd}{false}{cor}{true}
\insertexo{1M-2z37j}{false}{cor}{true}
\insertexo{1M-9ppsy}{false}{cor}{true}




\textLigne{Série 18 -- Fonctions affines et droites suite}
\insertexo{1M-trcsj}{false}{cor}{true}
\insertexo{1M-96gye}{false}{cor}{true}
\insertexo{1M-bxak6}{false}{cor}{true}
\insertexo{1M-t85gc}{false}{cor}{true}
\insertexo{1M-v4s75}{false}{cor}{true}
\insertexo{1M-3v9h8}{false}{cor}{true}


\textLigne{Série 19 -- Fonctions quadratiques et paraboles}
\insertexo{1M-habv6}{false}{cor}{true}
\insertexo{1M-y3kxz}{false}{cor}{true}

\textLigne{Série 20 -- Fonctions quadratiques et paraboles suite}
\insertexo{1M-vkggb}{false}{cor}{true}
\insertexo{1M-jyun7}{false}{cor}{true}

\textLigne{Série 21 -- Fonction inverse et fonction racine carrée}
\insertexo{1M-kjxge}{false}{cor}{true}
\insertexo{1M-u9her}{false}{cor}{true}
\insertexo{1M-uc1n8}{false}{cor}{true}


\textLigne{Exercices supplémentaires}
\insertexo{1M-sbyj3}{false}{cor}{true}
\insertexo{1M-k6m9k}{false}{cor}{true}
\insertexo{1M-xwye2}{false}{cor}{true}
\insertexo{1M-h9s4d}{false}{cor}{true}
\insertexo{1M-b4vhc}{false}{cor}{true}
\insertexo{1M-jr16y}{false}{cor}{true}

\newpage

{\bfseries Compétence travaillée par exercice :}

{\footnotesize
\textLigne{Généralités sur les fonctions}
\begin{enumerate}[label=\protect\tikzlabel{\arabic*}]
    \item Lire un graphique, calculer des images et des préimages.
    \item Lire un graphique et utiliser le vocabulaire.
    \item Modéliser un problème par une fonction.
    \item Déterminer l'expression d'une fonction à partir de couples.
    \item Calculer des images, des préimages et des zéros.
    \item Calculer des images et des préimages.
    \item Calculer des images.
    \item Calculer des images.
	\item Déterminer les zéros d'une fonction.
	\item Calculer des images et des préimages.
\end{enumerate}

\textLigne{Fonctions affines et droites}
\begin{enumerate}[label=\protect\tikzlabel{\arabic*},resume]
    \item Associer des tableaux de valeurs et des graphiques.
    \item Compléter des tableaux de valeurs à partir d'un graphique.
    \item Compléter des tableaux de valeurs à partir d'une expression algébrique.
    \item Vérifier l'appartenance d'un point à un graphique.
    \item Tracer des droites à partir de tableaux de valeurs.
    \item Identifier des fonctions constantes, linéaires et affines.
    \item Déterminer des pentes et des ordonnées à l'origine.
    \item Déterminer des équations de droites à partir de graphiques.
    \item Déterminer des équations de droites à partir de conditions.
    \item Déterminer des équations de droites à partir de deux points.
    \item Déterminer des fonctions linéaires à partir d'images.
    \item Déterminer des équations de droites parallèles et perpendiculaires.
    \item Déterminer des équations de droites parallèles.
    \item Déterminer des équations de droites à partir de conditions variées.
    \item Vérifier l'alignement de points.
    \item Déterminer des pentes, des ordonnées à l'origine et des points d'intersection.
\end{enumerate}

\textLigne{Fonctions quadratiques et paraboles}
\begin{enumerate}[label=\protect\tikzlabel{\arabic*},resume]
    \item Calculer des zéros, des sommets et tracer des paraboles.
    \item Étudier complètement des fonctions quadratiques et tracer des paraboles.
    \item Tracer des paraboles et calculer des points d'intersection.
    \item Calculer des points d'intersection de paraboles.
\end{enumerate}

\textLigne{Fonction inverse et fonction racine carrée}
\begin{enumerate}[label=\protect\tikzlabel{\arabic*},resume]
    \item Déterminer les domaines de définition et tracer des graphiques.
    \item Déterminer les domaines de définition, tracer des graphiques et trouver les intersections de fonctions.
    \item Déterminer les domaines de définition, tracer des graphiques et trouver les intersections de fonctions.
\end{enumerate}

\textLigne{Exercices supplémentaires}
\begin{enumerate}[label=\protect\tikzlabel{\arabic*},resume]
    \item Calculer des préimages.
    \item Déterminer des domaines de définition.
    \item Déterminer des pentes, des ordonnées à l'origine, des équations de droites parallèles et perpendiculaires.
    \item Déterminer des fonctions affines à partir de conditions variées.
    \item Déterminer des équations de droites à partir de conditions variées.
    \item Déterminer des équations de droites perpendiculaires et calculer des points d'intersection.
\end{enumerate}
}

\end{document}


