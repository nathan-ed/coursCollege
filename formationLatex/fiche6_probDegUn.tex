\documentclass[a4paper,12pt]{report}
\usepackage[fiche,niveau={1MA1},nfiche={6}, annee={Emilie-Gourd, 2024--2025}, auteur={ns},theme={Equations -- Problèmes du premier degré}]{packages/bandeaux}
\usepackage{packages/boites}
\usepackage{packages/fontes}
\usepackage{packages/courslatex}

\begin{document}
Les équations sont l'outil principal de résolution de problème en mathématiques. 

Voici les étapes à suivre pour résoudre un problème à l'aide d'une équation.

{\bfseries Partie 1\,: Déterminer l'équation}
\begin{enumerate}[1)]
	\item Lire le problème autant de fois que nécessaire pour bien comprendre et interpréter la situation.

		{\bfseries Faire un schéma si besoin};
	\item déterminer les deux quantités à comparer, elles deviendront les deux membres de l'équation;
	\item poser une inconnue qui permet d'exprimer les quantités à égaliser;
	\item écrire les deux membres de l'équation en fonction de l'inconnue. L'inconnue devrait apparaître dans au moins un des deux membres.
\end{enumerate}
{\bfseries Partie 2\,: Résoudre l'équation}
\begin{enumerate}[1),resume]
	\item Résoudre l'équation avec les principes d'équivalence.
\end{enumerate}
{\bfseries Partie 3\,: Interpréter le résultat et conclure}
\begin{enumerate}[1),resume]
	\item Vérifier la cohérence du résultat avec le choix de l'inconnue;
	\item Interpréter la valeur obtenue et répondre à la question posée dans l'énoncé.
\end{enumerate}
\begin{boiteExT}[Exemple 1]
	
\vspace{6.5cm}

\end{boiteExT}
\begin{boiteExT}[Exemple 2]
	
\vspace{6.5cm}

\end{boiteExT}
\newpage

Déterminer la mesure du côté du carré afin que les périmètres du carré et du triangle soient égaux.
\begin{center}
	\begin{tikzpicture}[scale=0.8]
    % Define the length of the side of the square and triangle
    \def\side{4}

    % Define points for the square
    \tkzDefPoint(0,0){A} % bottom-left of the square
    \tkzDefPoint(9,0){F}
    \tkzDefPoint(\side,0){B} % bottom-right of the square
    \tkzDefSquare(A,B) \tkzGetPoints{C}{D} % Top points of the square

    % Draw the square
    \tkzDrawPolygon(A,B,C,D)

    % Right angle marks for the square
    \tkzMarkRightAngle[size=0.3](A,B,C)
    \tkzMarkRightAngle[size=0.3](B,C,D)
    \tkzMarkRightAngle[size=0.3](C,D,A)
    \tkzMarkRightAngle[size=0.3](D,A,B)

    % Define points for the equilateral triangle starting from point B
    \tkzDefEquilateral(B,F) \tkzGetPoint{E} % E is the third point of the triangle
    \tkzDrawPolygon(B,F,E) % Draw the equilateral triangle

    % Mark equal sides of the triangle
    \tkzMarkSegments[mark=|](B,F B,E F,E)
    \tkzMarkSegments[mark=||](A,B B,C C,D D,A)

    % Draw the dimension line from A to E and label it 20 cm
    %\tkzDrawSegment[<->,blue](A,E)
    %\node[below] at ($(A)!0.5!(E)$) {20 cm};

\end{tikzpicture}
\end{center}

\begin{boiteExT}[Exemple 3]
	
\vspace{20cm}

\end{boiteExT}
\end{document}


