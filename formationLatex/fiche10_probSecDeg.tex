\documentclass[a4paper,12pt]{report}
\usepackage[fiche,niveau={1MA1},nfiche={10}, annee={Emilie-Gourd, 2024--2025}, auteur={ns},theme={Problèmes du second degré}]{packages/bandeaux}
\usepackage{packages/boites}
\usepackage{packages/fontes}
\usepackage{packages/courslatex}

\begin{document}
Pour résoudre un problème du second degré, on procède de la même manière que pour résoudre un problème du premier degré.
\begin{tasks}
	\task Lire et relire autant de fois que nécessaire la consigne.
	\task Déterminer l'inconnue ou les inconnues à poser.
	\task Écrire l'équation ou les équations qui permettent de mettre en relation les inconnues et les données.
	\task Résoudre les équations en appliquant la méthode de résolution adaptée.
	\task Interpréter le résultat et répondre à la question.
	\task Vérifier la cohérence du résultat.
\end{tasks}

\begin{boiteExT}[Le périmètre d'un rectangle vaut $96$ m et son aire $540$ m$^2$. Déterminer les mesures du rectangle.]

	\vspace{17cm}	

\end{boiteExT}
\end{document}


