\documentclass[a4paper,12pt]{report}
\usepackage[serie,niveau={1MA1},nfiche={}, annee={Emilie-Gourd, 2024--2025}, auteur={ns},theme={Activité -- Mise en équation}]{packages/bandeaux}
\usepackage{packages/boites}
\usepackage{packages/fontes}
\usepackage{packages/courslatex}

\begin{document}
Voici neuf énoncés et onze équations. Chaque énoncé correspond à une des équations proposées. Retrouver laquelle et justifier.
\vspace{5pt}

\begin{tabularx}{\textwidth}{|c|X|}
\hline 1 & Une balance à deux plateaux est en équilibre lorsque l'on place 10 cubes et une masse de 2 kg sur l'un des plateaux et 2 cubes et une masse de 30 kg sur l'autre. Quelle est la masse d'un cube? \\
\hline 2 & Une troupe théâtrale donne une représentation au collège. Pour payer le cachet des acteurs, chaque élève doit payer 30 Fr. Le jour de la représentation, 10 élèves sont absents, et chaque élève doit payer 2 Fr. de plus que prévu. Combien d'élèves assistent à cette représentation? \\
\hline 3 & Un père a 30 ans et son fils a 10 ans. Dans combien d'années l'âge du père sera-t-il le double de celui du fils? \\
\hline 4 & Pour aller de Grenoble au col du Lautaret, un touriste à vélo roule à la vitesse de $10 \mathrm{~km} / \mathrm{h}$. Il se repose 2 heures au col du Lautaret. Au retour, il roule à la vitesse de $30 \mathrm{~km} / \mathrm{h}$. Sachant que cet entraînement a duré 10 heures, quelle est la distance séparant Grenoble du col du Lautaret? \\
\hline 5 & Un rectangle est tel que sa longueur est le double de sa largeur. Si on augmente sa longueur de 30 m et si on diminue sa largeur de 10 m , son aire est multipliée par 2. Quelle est la largeur initiale du rectangle? \\
\hline 6 & Un pavé droit a une hauteur de 30 cm ; sa largeur est le dixième de sa longueur et lorsque l'on calcule l'aire totale et le volume de ce pavé, on trouve les mêmes nombres (l'un en $\mathrm{cm}^2$ et l'autre en $\mathrm{cm}^3$). Quelle est la largeur de ce pavé? \\
\hline 7 & L'eau de Javel est utilisée diluée. Une solution diluée à 2\% contient 2 cl d'eau de Javel et 98 cl d'eau pour former un litre (100 cl) de solution. Une solution diluée à $30 \%$ contient 30 cl d'eau de Javel pour former un litre (100 cl) de solution. Quelle quantité de solution à $30 \%$ faut-il ajouter à 1 litre d'une solution à $2 \%$ pour obtenir une solution à $10 \%$ ? \\
\hline 8 & On dispose d'une plaque de carton carrée de 10 cm de côté. On découpe quatre carrés (un dans chaque coin) et on plie de façon à obtenir une boîte sans couvercle. Quelle est la dimension des carrés à découper pour que le volume de la boîte soit égal à $30 \mathrm{~cm}^3$ ? \\
\hline 9 & Un arbre haut de 10 m et un poteau haut de 2 m sont situés en face l'un de l'autre sur chacune des rives d'une rivière large de 30 m . Au sommet de chacun d'eux est perché un oiseau. Ils se lancent tous deux à la même vitesse et au même instant sur une pauvre mouche qui les nargue à la surface de l'eau. Par un effet magique de Dame Nature, ils l'atteignent au même moment et se dfracassent le bec dans un contact plus que vigoureux, et de ce fait se retrouvent bredouilles. A quelle distance du pied de l'arbre se trouvait cette mouche mdfraculée? \\
\hline
\end{tabularx}
\vspace{5pt}

\begin{tasks}(2)
\task $\dfrac{x}{10}+2+\dfrac{x}{30}=10$
\task $10(x+2)=30 x$
\task $x(10-2 x)(10-2 x)=30$
\task $(30+x)=2(10+x)$
\task $30(x+10)=x(30+2)$
\task $\dfrac{30 x}{2}+\dfrac{10 x}{2}=\dfrac{1}{2}(x+10) 30$
\task $10 x+2=2 x+30$
\task $x^2+10^2=(30-x)^2+2^2$
\task $(2 x+30)(x-10)=2 x(2 x)$
\task $\dfrac{30}{100} x+\dfrac{2}{100}=(x+1) \dfrac{10}{100}$
\task $2(30 x+x \cdot 10 x+30 \cdot 10 x)=x \cdot 10 x \cdot 30$
\end{tasks}
\end{document}
