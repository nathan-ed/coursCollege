\documentclass[a4paper,12pt]{report}
\usepackage[fiche,niveau={1MA1},nfiche={12}, annee={Emilie-Gourd, 2024--2025}, auteur={ns},theme={Systèmes d'équations linéaires -- Généralités}]{packages/bandeaux}
\usepackage{packages/boites}
\usepackage{packages/fontes}
\usepackage{packages/courslatex}


\begin{document}
\begin{boiteicone}[][][\faIcon[light]{pencil-alt}]
	\begin{itemize}
		\item Un système d'équations est un ensemble d'équations portant sur les mêmes inconnues qui doivent être résolues simultanément. Un {\bfseries système d'équations linéaires} est un système dont toutes les équations sont du premier degré.
		\item Une solution d'un système d'équations à deux inconnues $x,y\in \mathbb{R}$ est un couple noté $(x_0 ; y_0)$ qui vérifie les deux équations simultanément .
\item Une solution d'un système d'équations à trois inconnues $x , y , z\in \mathbb{R}$ est un triplet noté $(x_0 ; y_0 ; z_0)$ qui vérifie les trois équations simultanément.
	\end{itemize}
\end{boiteicone}

Voici deux exemples : 

Le triplet $(1;3;1)$ est solution du système
\[\systeme{
  x - 2y + 2z = -3,
  3x + 2y + 5z = 14
},\] car $1-2\cdot 3+2\cdot 1=-3$ et $3\cdot 1+2\cdot 3+5\cdot 1 =14$.

Mais le triplet $(0;2;1)$ n'est pas solution même si $1\cdot 0-2\cdot 2+2\cdot 1=-3$, car $3\cdot 0+2\cdot 2+5\cdot 1=9\neq 14$. 

\begin{boiteicone}
Toutes les équations du système doivent être vérifiées simultanément. 
\end{boiteicone}

Résoudre une système c'est déterminer l'ensemble de toutes ses solutions. 

Pour résoudre un système d'équations, on cherche à \og{}éliminer une ou plusieurs inconnues\fg{} pour se ramener à un système avec moins d'inconnues.


\begin{boiteicone}[][][\faIcon[light]{pencil-alt}]
Deux systèmes sont dits équivalents s'ils ont le même ensemble de solutions.
\end{boiteicone}

En plus des deux principes d'équivalence déjà rencontrés lors de la résolution d'équations du premier degré, nous pouvons également appliquer le principe d'équivalence suivant~:
\begin{enumerate}
	\item[PE3] en additionnant à une équation d'un système un multiple non nul d'une autre équation du même système, on obtient un système équivalent.
\end{enumerate}

\begin{boiteExT}[$\systeme{3x+y=4,-2x+3y=10}$]
\vspace{7cm}
\end{boiteExT}
\end{document}
