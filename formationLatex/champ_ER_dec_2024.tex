\documentclass[a4paper,12pt]{report}
\usepackage[competences,niveau={1MA1},nfiche={}, annee={Emilie-Gourd, 2024--2025}, auteur={ns},theme={Champ pour l'épreuve regroupée}]{packages/bandeaux}
\usepackage{packages/boites}
\usepackage{packages/fontes}
\usepackage{packages/courslatex}
\usepackage{hyperref}
\setlist[itemize]{align=left,labelsep=1em,leftmargin=*,itemsep=5pt,parsep=5pt,topsep=5pt,rightmargin=0cm}

\RequirePackage{qrcode}
\newcommand{\qrwithlabel}[2]{%
    \begin{minipage}{0.23\linewidth} % Adjust width to fit 4 on a line
        \centering
        \textbf{#1} \\ % Description on top
	\smallskip
	\href{#2}
	{\qrcode[height=3cm]{#2}} % QR code with fixed height
    \end{minipage}%
}

\begin{document}
Je suis capable de

\begin{itemize}
\item     Connaitre le vocabulaire et les résultats du cours ;
\item  Savoir factoriser des polynômes à l'aide de la complétion du carré et de la formule du discriminant ;
\item  Savoir factoriser des polynômes par enchaînement des techniques de mise en évidence, factorisation à l'aide des identités remarquables et des groupements ;
\item  Savoir résoudre des équations du deuxième degré à coefficients réels à l'aide des principes d'équivalence, de la factorisation, de la complétion du carré et de la formule du discriminant ;
\item   Savoir résoudre des équations irrationnelles ;
\item   Savoir résoudre des équations bicarrées ;
\item   Savoir résoudre des systèmes d'équations 2x2 et 3x3 ;
\item    Savoir résoudre un problème en posant et résolvant un système d'équations.
\end{itemize}

En particulier, les exercices des séries 6 à 15 ainsi que les fiches 1 à 19 sont au champ de l'évaluation.

Pour commencer

\begin{center}
    \qrwithlabel{Identités remarquables}{https://coopmaths.fr/alea/?uuid=6e472&id=11FA1-11&n=5&d=10&s=1-4&s2=1&s3=2&s4=2&cd=1&uuid=6e472&id=11FA1-11&n=5&d=10&s=5-8&s2=1&s3=2&s4=2&cd=1&lang=fr-CH&v=eleve&es=1011001&title=Identit\%C3\%A9s+remarquables}
    \qrwithlabel{Factorisation avec identités}{https://coopmaths.fr/alea/?uuid=c4b73&id=11FA1-12&n=5&d=10&s=2-4&s2=1&s3=1&s4=2&s5=false&cd=1&uuid=c4b73&id=11FA1-12&n=5&d=10&s=2-4&s2=1&s3=2&s4=2&s5=false&cd=1&lang=fr-CH&v=eleve&es=1011001&title=Factoriser+avec+les+identit\%C3\%A9s+remarquables}
    \qrwithlabel{Groupements}{https://coopmaths.fr/alea/?uuid=9965d&id=11FA1-13&n=5&d=10&s=1-2-3-7&s2=1&s3=1&s4=1&s5=1&cd=1&uuid=9965d&id=11FA1-13&n=5&d=10&s=1-2-3-7&s2=3&s3=2&s4=1&s5=1&cd=1&lang=fr-CH&v=eleve&es=1011001&title=Factoriser+avec+les+groupements}
    \qrwithlabel{Complétion du carré}{https://coopmaths.fr/alea/?uuid=7f0dc&id=CalcLit4-1&n=3&d=10&s=3&s2=true&cd=1&uuid=7f0dc&id=CalcLit4-1&n=3&d=10&s=3&s2=false&cd=1&lang=fr-CH&v=eleve&es=0011001&title=Compl\%C3\%A9tion+du+carr\%C3\%A9}
\medskip


    \qrwithlabel{Résolution avec delta}{https://coopmaths.fr/alea/?uuid=1be55&id=1AL23-22&n=6&d=10&s=4&s2=4&s3=5&cd=1&lang=fr-CH&v=eleve&es=1011001&title=R\%C3\%A9soudre+avec+la+formule+du+deuxi\%C3\%A8me+degr\%C3\%A9}
    \qrwithlabel{Résoudre pour factoriser}{https://coopmaths.fr/alea/?uuid=334ca&id=1AL21-40&n=3&d=10&s=1&cd=1&lang=fr-CH&v=eleve&es=1011001&title=R\%C3\%A9soudre+pour+factoriser}
    \qrwithlabel{Équations bicarrées}{https://coopmaths.fr/alea/?uuid=89034&id=CalcLit4-3&n=3&d=10&s=1&s2=1&alea=zmq8&cd=1&uuid=89034&id=CalcLit4-3&n=3&d=10&s=3&s2=1&alea=nGa1&cd=1&lang=fr-CH&v=eleve&es=1011001&title=\%C3\%89quations+bicarr\%C3\%A9es}
    \qrwithlabel{Équations irrationnelles}{https://coopmaths.fr/alea/?uuid=5f5fa&id=CalcLit4-2&n=3&d=10&s=3&s2=true&s3=false&cd=1&uuid=5f5fa&id=CalcLit4-2&n=3&d=10&s=3&s2=false&s3=true&cd=1&lang=fr-CH&v=eleve&es=1011001&title=\%C3\%89quations+irrationnelles}
\medskip

    \qrwithlabel{Solutions d'un système}{https://coopmaths.fr/alea/?uuid=ccb71&id=2G34-3&lang=fr-CH&v=eleve&es=1011001&title=Tester+si+un+couple+est+solution+d\%27un+syst\%C3\%A8me+d\%27\%C3\%A9quations}
    \qrwithlabel{Résolution par substitution}{https://coopmaths.fr/alea/?uuid=521b6&id=11FA6-5&n=3&d=10&s=1&cd=1&uuid=521b6&id=11FA6-5&n=3&d=10&s=2&cd=1&lang=fr-CH&v=eleve&es=1011001&title=R\%C3\%A9soudre+un+syst\%C3\%A8me+par+substitution}
    \qrwithlabel{Résolution par combinaison linéaire}{https://coopmaths.fr/alea/?uuid=5179b&id=11FA6-6&n=3&d=10&s=1&s2=false&s3=false&cd=1&uuid=5179b&id=11FA6-6&n=3&d=10&s=2&s2=false&s3=false&cd=1&uuid=5179b&id=11FA6-6&n=3&d=10&s=1&s2=true&s3=false&cd=1&lang=fr-CH&v=eleve&es=1011001&title=R\%C3\%A9soudre+un+syst\%C3\%A8me+par+combinaison+lin\%C3\%A9aire}
    \qrwithlabel{Problèmes}{https://coopmaths.fr/alea/?uuid=6fbf9&id=2G34-9&n=2&d=10&s=3&cd=1&lang=fr-CH&v=eleve&es=1011001&title=Probl\%C3\%A8mes+avec+des+syst\%C3\%A8mes}
\end{center}
\newpage

\section*{Planification de révision - Répétition espacée}

\noindent Période : du 18 novembre au 18 décembre 2024, minimum 15-20 minutes par jour \\
Jour de repos : chaque 7ème jour (par exemple le dimanche) \\
Une fois que les exercices QR-codes sont maîtrisés, reprendre les exercices des séries correspondants à chaque sujet en fonction du champ.\\
Planning généré par ChatGPT et légèrement adapté.

\begin{longtable}{|c|p{8cm}|c|}
\hline
\textbf{Date} & \textbf{Révision du sujets à réviser} &\\
\hline
\endfirsthead
\hline
\textbf{Date} & \textbf{Révision du sujets à réviser} &\\
\hline
\endhead
18 novembre & Révision du sujet 1 & $\square$\\
\hline
19 novembre & Révision du sujet 2 & $\square$\\

\hline
20 novembre & Révision du sujet 3 & $\square$\\

\hline
21 novembre & Révision du sujet 4 & $\square$\\

\hline
22 novembre & Révision du sujet 5 & $\square$\\

\hline
23 novembre & Révision du sujet 6 & $\square$\\

\hline
24 novembre & \textbf{Repos} \\
\hline
25 novembre & Révision du sujet 7 & $\square$\\

\hline
26 novembre & Révision du sujet 8 & $\square$\\

\hline
27 novembre & Révision du sujet 9 & $\square$\\

\hline
28 novembre & Révision du sujet 10 & $\square$\\

\hline
29 novembre & Révision du sujet 11 & $\square$\\

\hline
30 novembre & Révision du sujet 12 & $\square$\\

\hline
1er décembre & \textbf{Repos} \\
\hline
2 décembre & Révision des sujets 1 et 2 & $\square$\\

\hline
3 décembre & Révision des sujets 3 et 4 & $\square$\\

\hline
4 décembre & Révision des sujets 5 et 6 & $\square$\\

\hline
5 décembre & Révision des sujets 7 et 8 & $\square$\\

\hline
6 décembre & Révision des sujets 9 et 10 & $\square$\\

\hline
7 décembre & \textbf{Repos} \\
\hline
8 décembre & Révision des sujets 11 et 12 & $\square$\\

\hline
9 décembre & Révision des sujets 1, 4, 7, 10 & $\square$\\

\hline
10 décembre & Révision des sujets 2, 5, 8, 11 & $\square$\\

\hline
11 décembre & Révision des sujets 3, 6, 9, 12 & $\square$\\

\hline
12 décembre & Révision ciblée : Points faibles identifiés & $\square$\\

\hline
13 décembre & Révision générale des sujets 1 à 6 & $\square$\\

\hline
14 décembre & Révision générale des sujets 7 à 12 & $\square$\\

\hline
15 décembre & \textbf{Repos (ou pas)} & $\square$\\

\hline
15, 16 ou 17 décembre & Simulations d'examen (tous les sujets) & $\square$\\

\hline
17 décembre & Révision légère et relaxation & $\square$\\

\hline
18 décembre & \textbf{Jour de l'examen} \\
\hline
\end{longtable}
\end{document}
