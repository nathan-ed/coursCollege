\documentclass[a4paper,12pt]{report}
\usepackage[fiche,niveau={1MA1},nfiche={7}, annee={Emilie-Gourd, 2024--2025}, auteur={ns},theme={Équations -- Théorème du produit nul}]{packages/bandeaux}
\usepackage{packages/boites}
\usepackage{packages/fontes}
\usepackage{packages/courslatex}

\begin{document}

\begin{boiteExT}[Théorème du produit nul]
	\vspace{10cm}	

\end{boiteExT}

Grâce à la factorisation et au théorème du produit nul, on peut résoudre une équation de dégré supérieur en résolvant plusieurs équations de dégré inférieur.

\begin{boiteExT}[Résoudre $x^3(x+2)^2=x^2(x+2)^2$]
	\vspace{10cm}

\end{boiteExT}

\begin{boiteExT}[Résoudre $x^2+2x-1=0$]
	En appliquant les étapes suivantes\,:

	\begin{tasks}
\task $\left[P E_1\right]:$ ajouter 2
\vspace{1.5cm}
\task factoriser le membre de gauche
\vspace{1.5cm}
\task $\left[P E_1\right]:$ ajouter $(-2)$
\vspace{1.5cm}
\task factoriser le membre de gauche
\vspace{1.5cm}
\task $\left[P N\right]$
\vspace{1.5cm}
	\end{tasks}
\end{boiteExT}

\begin{boiteExT}[Résoudre $(8x+1)^2-(2x-3)^2=0$]
	\vspace{11cm}	
\end{boiteExT}

\end{document}
