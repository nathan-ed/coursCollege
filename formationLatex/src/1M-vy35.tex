\titre{}
\theme{equations}
\auteur{Nathan Scheinmann}
\niveau{1M}
\source{muzy-17}
\type{serie}
\piments{3}
\pts{}
\annee{2425}

\contenu{
Déterminer le nombre $a$ pour que l'équation ait la solution demandée.
	\begin{tasks}(2)
\task $a x+1=2 x+5$; solution: $S=\{-2\}$;
\task $1-a x=4 x+2$; solution : $S=\left\{\dfrac{1}{3}\right\}$
\task $3=a \cdot\left(-\dfrac{1}{2} x+3\right)$; solution : $S=\{-1\}$
\task $7-2 x=x+a x$; solution: $S=\{3\}$.
	\end{tasks}
}
\correction{
	\begin{tasks}(4)
		\task $0$
		\task $\dfrac{4}{7}$
		\task $-7$
		\task $-\dfrac{2}{3}$
	\end{tasks}}

