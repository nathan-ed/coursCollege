\titre{musy-169}
\theme{fonctions}
\auteur{Nathan Scheinmann}
\niveau{1M}
\source{}
\type{serie}
\piments{2}
\pts{}
\annee{2425}

\contenu{
Voici des équations de droites :
\begin{tasks}(3)
\task $3 x+2 y-9=0$;
\task $3 x=4 y$;
\task $y=-x+3$;
\task $x=\frac{1}{2} y+3$;
\task $\frac{x}{2}+\frac{y}{5}=1$
\end{tasks}
\begin{tasks}
	\task[1)] Déterminer la pente et l'ordonnée à l'origine de chaque droite.
	\task[2)] Donner la représentation graphique de ces droites (un seul repère).
	\task[3)] Calculer les coordonnées des différents points d'intersections de ces droites.
\end{tasks}
}
\correction{

}

