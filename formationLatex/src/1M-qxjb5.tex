\titre{}
\theme{fonctions}
\auteur{Nathan Scheinmann}
\niveau{1M}
\source{ns}
\type{serie}
\piments{2}
\pts{}
\annee{2425}

\contenu{
Déterminer la pente et l'ordonnée à l'origine des fonctions affines suivantes. Puis la représenter graphiquement.
		\begin{alignat*}{3}
			&f_1: x \mapsto-2 x+3& \quad\quad\quad\quad  & f_4: x \mapsto-2 &\quad\quad\quad\quad & f_7: x \mapsto 3 x \\
			&f_2: x \mapsto 5-x&  & f_5: x \mapsto x & &f_8: x \mapsto 0 \\
			&f_3: x \mapsto \frac{5 x}{4} & & f_6: x \mapsto \frac{2 x-3}{4}  & & f_9: x \mapsto \frac{2-x}{3}
\end{alignat*}
}
\correction{

}

