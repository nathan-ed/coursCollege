\titre{undefined}
\theme{equations}
\auteur{Nathan Scheinmann}
\niveau{1M}
\source{MER11e}
\type{serie}
\piments{1}
\pts{}
\annee{2425}

\contenu{
		Un groupe de vingt-quatre personnes fait un stage de deux jours dans une école de voile. Deux activités sont au programme: la planche à voile ou le catamaran. Le premier jour, dix personnes choisissent la planche à voile et les autres le catamaran. La facture totale de ce premier jour s'élève à $560$ francs. Le deuxième jour, ils sont douze à choisir la planche à voile et les autres font du catamaran. La facture du deuxième jour s'élèves à $540$ francs. 

	Quel est le prix par personne d'une journée de planche à voile et celui d'une journée de catamaran?
}
\correction{
On pose les inconnues
\begin{align*}
	&x=\text{le prix d'une journée en planche à voile} &&y=\text{le prix d'une journée en catamaran}
\end{align*}
On obtient le système
$\begin{cases}
10x+14y=560\\
12x+12y=540
\end{cases}$
On résout le système (par exemple par combinaison) et on obtient que $x=17,50$ et $y=27,50$. Une journée en planche à voile coûte CHF $17,50$ et une journée en catamaran CHF $27,50$.
}

