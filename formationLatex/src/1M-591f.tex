\titre{undefined}
\theme{equations}
\auteur{Nathan Scheinmann}
\niveau{1M}
\source{undefined}
\type{serie}
\piments{2}
\pts{}
\annee{2425}

\contenu{
Céline regarde avec envie un pull et une robe présentés dans la vitrine d’une boutique.
Malheureusement, le prix total de ces deux vêtements est de $137.50$ francs et dépasse
son budget. Quelques temps après, le prix du pull baisse de $20\%$ et celui de la robe de $30\%$. Céline calcule rapidement la dépense totale et constate que le prix total a baissé de
$35$ francs, ce qui lui permet d’acheter ces deux vêtements.
Quels étaient les prix du pull et de la robe avant la baisse?
}
\correction{
	On pose les inconnues
\begin{align*}
	&x=\text{Le prix du pull avant rabais} &&y=\text{Le prix de la robe avant rabais}
\end{align*}
On obtient le système
$
\begin{cases}
x+y=137,50\\
x-\dfrac{20}{100}x+y-\dfrac{30}{100}y=137,50-35
\end{cases}
$
On réduit au maximum la deuxième équation:
\[x-\dfrac{20}{100}x+y-\dfrac{30}{100}y=137,50-32,50\iff \dfrac{8}{10}x+\frac{7}{10}y=102,50 \iff 8x+7y=1025\]
Puis on résout le système
$
\begin{cases}
x+y=137,50\\
8x+7y=1025
\end{cases}
$
(par exemple par combinaison) et on obtient que $x=62,5$ et $y=75$, ainsi le prix du pull avant le rabais est de 62,50 francs et le prix de la robe avant le rabais était de $75$ francs. 
}

