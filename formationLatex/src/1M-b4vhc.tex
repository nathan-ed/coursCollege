\titre{musy-177}
\theme{fonctions}
\auteur{Nathan Scheinmann}
\niveau{1M}
\source{}
\type{serie}
\piments{2}
\pts{}
\annee{2425}

\contenu{
 Déterminer l'expression algébrique de la fonction affine... :
 \begin{tasks}
\task $f_1$, de pente $1$ et de zéro $-2$ ;
\task $f_2$, telle l'image de $1$ soit $-4$ et la préimage de $-2$ soit $3$ ;
\task $f_3$, de pente $-\dfrac{1}{2}$, et passant par le point $C=(-1 ; 2)$;
\task $f_4$, horizontale et passant par le point $D=(-2 ; 2)$;
\task $f_5$, passant par l'origine et parallèle à la droite d'équation $3 x+4 \mathrm{y}-3=0$;
\task $f_6$, passant par le point $P=(-3 ; 0)$ et perpendiculaire à la droite $g(x)=3-4 x$.
\task $f_7$, coupant l'axe des abscisses en $-5$ , et celui des ordonnées en $\dfrac{3}{2}$.
\end{tasks}
}
\correction{
	\begin{tasks}(4)
	\task $f_1(x)=x+2$
	\task $f_2(x)=x-5$
	\task $f_3(x)=-\dfrac{1}{2} x+\dfrac{3}{2}$
	\task $f_4(x)=2$
	\task $f_5(x)=-\dfrac{3}{4} x$ 
	\task $f_6(x)=\dfrac{1}{4} x+\dfrac{3}{4}$
	\task $f_7(x)=\dfrac{3}{10} x+\dfrac{3}{2}$
	\end{tasks}
}

