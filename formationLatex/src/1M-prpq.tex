\titre{106-musy17}
\theme{equations}
\auteur{Nathan Scheinmann}
\niveau{1M}
\source{muzy-2017}
\type{serie}
\piments{1}
\pts{}
\annee{2425}

\contenu{
Un problème de Leonhard Euler (1707 - 1783).

Un père mourut en laissant quatre fils. Ceux-ci se partagèrent ses biens de la manière suivante : le premier prit la moitié de la fortune, moins 3000 livres; le deuxième en prit le tiers moins 1000 livres; le troisième prit exactement le quart des biens; le quatrième prit 600 livres plus le cinquième des biens. Quelle était la fortune totale, et quelle somme reçut chacun des enfants?
}
\correction{
$12000; 3000$
}

